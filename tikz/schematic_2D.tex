\begin{figure}
	\begin{center}
		\begin{tikzpicture}[scale=0.7]
		\begin{scope}
		\foreach \i in {0,...,3}
		{
			\draw (-1,\i) node[anchor=east] {$\i$} --(2,\i);
		}
		\filldraw[color=color0] (0,0) node[anchor=north,inner sep=.4cm, color=black] {$n_x$} circle (0.25cm); 
		\filldraw[color=color0] (1,0) node[anchor=north,inner sep=.4cm, color=black] {$n_y$} circle (0.25cm);
		
		\draw [decorate,decoration={brace,amplitude=10pt},xshift=0pt,yshift=0pt]
		(-1.2,3.5) -- (2.2,3.5) node [black,midway,xshift=0.0cm, yshift=15pt] 
		{$n=0$};
		\end{scope}
		\begin{scope}[xshift=5cm]
		\foreach \i in {0,...,3}
		{
			\draw (-1,\i) --(2,\i);
		}
		\draw[color=color1] (0,0) node[anchor=north,inner sep=.4cm, color=black] {$n_x$} circle (0.25cm); 
		\filldraw[color=color0] (1,0) node[anchor=north,inner sep=.4cm, color=black] {$n_y$} circle (0.25cm);
		\filldraw[color=color0] (0,3) circle (0.25cm);
		\end{scope}
		\begin{scope}[xshift=9cm]
		\foreach \i in {0,...,3}
		{
			\draw (-1,\i) --(2,\i);
		}
		\draw[color=color1] (0,0) node[anchor=north,inner sep=.4cm, color=black] {$n_x$} circle (0.25cm); 
		\draw[color=color1] (1,0) node[anchor=north,inner sep=.4cm, color=black] {$n_y$} circle (0.25cm);
		\filldraw[color=color0] (0,2) circle (0.25cm); 
		\filldraw[color=color0] (1,1) circle (0.25cm); 
		\draw [decorate,decoration={brace,amplitude=10pt},xshift=0pt,yshift=0pt]
		(-5.2,3.5) -- (10.2,3.5) node [black,midway,xshift=0.0cm, yshift=15pt] 
		{$n=3$};
		\end{scope}
		\begin{scope}[xshift=13cm]
		\foreach \i in {0,...,3}
		{
			\draw (-1,\i) --(2,\i);
		}
		\draw[color=color1] (0,0) node[anchor=north,inner sep=.4cm, color=black] {$n_x$} circle (0.25cm); 
		\draw[color=color1] (1,0) node[anchor=north,inner sep=.4cm, color=black] {$n_y$} circle (0.25cm);
		\filldraw[color=color0] (0,1) circle (0.25cm); 
		\filldraw[color=color0] (1,2) circle (0.25cm); 
		\end{scope}
				\begin{scope}[xshift=17cm]
		\foreach \i in {0,...,3}
		{
			\draw (-1,\i) --(2,\i);
		}
		\filldraw[color=color0] (0,0) node[anchor=north,inner sep=.4cm, color=black] {$n_x$} circle (0.25cm); 
		\draw[color=color1] (1,0) node[anchor=north,inner sep=.4cm, color=black] {$n_y$} circle (0.25cm);
		\filldraw[color=color0] (1,3) circle (0.25cm); 
		\end{scope}
		\end{tikzpicture}
	\end{center}

	\begin{center}
		\begin{tikzpicture}[scale=0.7]
		\begin{scope}
		\foreach \i in {0,...,3}
		{
			\draw (-1,\i) node[anchor=east] {$\i$} --(2,\i);
		}
		\draw[color=color1] (0,0) node[anchor=north,inner sep=.4cm, color=black] {$n_x$} circle (0.25cm); 
		\filldraw[color=color0] (1,0) node[anchor=north,inner sep=.4cm, color=black] {$n_y$} circle (0.25cm);
		\filldraw[color=color0] (0,1) circle (0.25cm);
		\draw [decorate,decoration={brace,amplitude=10pt},xshift=0pt,yshift=0pt]
		(-1.2,3.5) -- (6.2,3.5) node [black,midway,xshift=0.0cm, yshift=15pt] 
		{$n=1$};
		\end{scope}
		\begin{scope}[xshift=4cm]
		\foreach \i in {1,...,4}
		{
			\draw (-1,\i-1) --(2,\i-1);
		}
		\filldraw[color=color0] (0,0) node[anchor=north,inner sep=.4cm, color=black] {$n_x$} circle (0.25cm); 
		\draw[color=color1] (1,0) node[anchor=north,inner sep=.4cm, color=black] {$n_y$} circle (0.25cm);
		\filldraw[color=color0] (1,1) circle (0.25cm); 
		\end{scope}
		\begin{scope}[xshift=9cm]
		\foreach \i in {1,...,4}
		{
			\draw (-1,\i-1) --(2,\i-1);
		}
		\draw[color=color1] (0,0) node[anchor=north,inner sep=.4cm, color=black] {$n_x$} circle (0.25cm); 
		\filldraw[color=color0] (1,0) node[anchor=north,inner sep=.4cm, color=black] {$n_y$} circle (0.25cm);
		\filldraw[color=color0] (0,2) circle (0.25cm);
		\end{scope}
		\begin{scope}[xshift=13cm]
		\foreach \i in {1,...,4}
		{
			\draw (-1,\i-1) --(2,\i-1);
		}
		\draw[color=color1] (0,0) node[anchor=north,inner sep=.4cm, color=black] {$n_x$} circle (0.25cm); 
		\draw[color=color1] (1,0) node[anchor=north,inner sep=.4cm, color=black] {$n_y$} circle (0.25cm);
		\filldraw[color=color0] (0,1) circle (0.25cm); 
		\filldraw[color=color0] (1,1) circle (0.25cm);
		\draw [decorate,decoration={brace,amplitude=10pt},xshift=0pt,yshift=0pt]
		(-5.2,3.5) -- (6.2,3.5) node [black,midway,xshift=0.0cm, yshift=15pt] 
		{$n=2$};
		\end{scope}
		\begin{scope}[xshift=17cm]
		\foreach \i in {1,...,4}
		{
			\draw (-1,\i-1) --(2,\i-1);
		}
		\filldraw[color=color0] (0,0) node[anchor=north,inner sep=.4cm, color=black] {$n_x$} circle (0.25cm); 
		\draw[color=color1] (1,0) node[anchor=north,inner sep=.4cm,color=black] {$n_y$} circle (0.25cm); 
		\filldraw[color=color0] (1,2) circle (0.25cm);
		\end{scope}
		\end{tikzpicture}
	\end{center}
	\caption{The possible states of a two-dimensional quantum dot for $n=n_x+n_y=0,1,2,3$. By recalling that an electron can take spins $+1/2$ and $-1/2$, one can use this schematic to determine the maximal number of electrons in each shell and thus reveal the magic numbers.}
	\label{fig:hostates}
\end{figure}