\titleframe{Methods}

\note{The applied methods will be discussed briefly}

\mframe{Variational Monte Carlo (VMC)}{}{
	Exploit the variational principle in order to obtain the ground state energy
	\begin{equation}
	\begin{aligned}
	E_0 < E_{\text{VMC}} &= \frac{\int d\bs{R}\Psi_T(\bs{R})^*\hat{\mathcal{H}}\Psi_T(\bs{R})}{\int d\bs{R}\Psi_T(\bs{R})^*\Psi_T(\bs{R})},\\
	%&= \int d\bs{R}\underbrace{\frac{\Psi_T(\bs{R})^*\Psi_T(\bs{R})}{\int d\bs{R}\Psi_T(\bs{R})^*\Psi_T(\bs{R})}}_{P(\bs{R})}\cdot\underbrace{\frac{1}{\Psi_T(\bs{R})}\hat{\mathcal{H}}\Psi_T(\bs{R})}_{E_L(\bs{R})}\\
	&=\int d\bs{R}E_L(\bs{R})P(\bs{R}),
	\end{aligned}
	\end{equation}
	with
	\begin{equation}
	E_L(\bs{R})=\frac{1}{\Psi_T(\bs{R})}\hat{\mathcal{H}}\Psi_T(\bs{R})\quad\wedge\quad P(\bs{R})=\frac{\Psi_T(\bs{R})^*\Psi_T(\bs{R})}{\int d\bs{R}\Psi_T(\bs{R})^*\Psi_T(\bs{R})}
	\end{equation}
}

\note{
	\begin{itemize}
		\item Our work is based on VMC
		\item Exploits variational principle
		\item We start with rewriting the expression in terms of the local energy and the probability density function
	\end{itemize}
}

\mframe{Monte Carlo Integration}{}{
	We attempt to solve the integral by sampling from the probability density function $P(\bs{R})\propto\Psi_T(\bs{R})^*\Psi_T(\bs{R})$:
	\begin{equation}
	\begin{aligned}
	E_{\text{VMC}} &= \int d\bs{R} E_L(\bs{R})P(\bs{R}),\\
	&\approx\frac{1}{M}\sum_{i=1}^ME_L(\bs{R}_i).
	\end{aligned}
	\end{equation}
}

\note{
	\begin{itemize}
		\item The reason $\rightarrow$ On the form of a general expectation value
		\item Can be solved by Monte Carlo integration
		\item Only gives an energy
		\item Find the ground state energy by adjusting the trial wave fucntion with respect to minimizing the energy. Repeat exercise. When the energy has converged, we have a ground state energy estimate.
	\end{itemize}
}

\mframe{Trial Wave Function Ansatz}{}{
	
	The Slater-Jastrow function is the \textit{de facto} standard trial wave function for electronic structure systems,
	\begin{equation}
	\Psi_T(\bs{R})=|\hat{D}(\bs{R})|J(\bs{R}),
	\end{equation}
	where the Slater matrix,
	\begin{equation}
	\hat{D}(\bs{R})=
	\begin{pmatrix}
	\phi_1(\boldsymbol{r}_1) & \phi_2(\boldsymbol{r}_1) & \hdots & \phi_N(\boldsymbol{r}_1)\\
	\phi_1(\boldsymbol{r}_2) & \phi_2(\boldsymbol{r}_2) & \hdots & \phi_N(\boldsymbol{r}_2)\\
	\vdots & \vdots & \ddots & \vdots \\
	\phi_1(\boldsymbol{r}_N) & \phi_2(\boldsymbol{r}_N) & \hdots & \phi_N(\boldsymbol{r}_N)
	\end{pmatrix},
	\end{equation}
	contains all the single-particle functions.
}

\note{
	\begin{itemize}
		\item Arbitrary function $\rightarrow$ Few requirements $\rightarrow$ electron systems
		\item Standard Slater-Jastrow function
	\end{itemize}
}

\mframe{Single-particle Functions}{}{
	The Hermite functions,
	\begin{equation}
	\phi_n(\bs{r})\propto H_n(\sqrt{\omega}\bs{r})\exp(-\frac{1}{2}\alpha\omega|\bs{r}|^2),
	\end{equation}
	are often used as the single-particle functions for quantum dots. The Gaussian can be factorized out from the Slater determinant,
	\begin{equation}
	|\hat{D}(\bs{R};\alpha)|\propto\exp(-\frac{1}{2}\alpha\omega|\bs{R}|^2)
	\begin{vmatrix}
	H_1(\boldsymbol{r}_1) & H_2(\boldsymbol{r}_1) & \hdots & H_N(\boldsymbol{r}_1)\\
	H_1(\boldsymbol{r}_2) & H_2(\boldsymbol{r}_2) & \hdots & H_N(\boldsymbol{r}_2)\\
	\vdots & \vdots & \ddots & \vdots \\
	H_1(\boldsymbol{r}_N) & H_2(\boldsymbol{r}_N) & \hdots & H_N(\boldsymbol{r}_N)
	\end{vmatrix}.
	\end{equation}
}

\note{
	\begin{itemize}
		\item Hermite functions often used for circular quantum dots $\rightarrow$ quantities
		\item An important finding
		\item Slater determinant exchange correlation
	\end{itemize}
}

\mframe{Restricted Boltzmann Machine}{}{
	As suggested by \citet{carleo_solving_2017}, we use the marginal distribution of the visible units as the single-particle functions in the Slater determinant, and see if them can model the correlations 
	\begin{equation}
	\phi_n(\bs{r})\propto H_n(\sqrt{\omega}\bs{r})P(\bs{r};\bs{\theta})
	\end{equation}
	where $P(\bs{r})$ is the marginal distribution of the visible units.
	\begin{equation}
	|\hat{D}(\bs{r};\bs{\theta})|\propto P(\bs{r};\bs{\theta})
	\begin{vmatrix}
	H_1(\boldsymbol{r}_1) & H_2(\boldsymbol{r}_1) & \hdots & H_N(\boldsymbol{r}_1)\\
	H_1(\boldsymbol{r}_2) & H_2(\boldsymbol{r}_2) & \hdots & H_N(\boldsymbol{r}_2)\\
	\vdots & \vdots & \ddots & \vdots \\
	H_1(\boldsymbol{r}_N) & H_2(\boldsymbol{r}_N) & \hdots & H_N(\boldsymbol{r}_N)
	\end{vmatrix}
	\end{equation}
}

\note{
	\begin{itemize}
		\item Our contribution $\rightarrow$ Marginal distribution
		\item Gives us a wave function where less physical intuition is needed
		\item Interesting because many systems, for instance nuclear systems, have very complex wave functions. We struggle with investigating those systems as we do not have the needed physical intuition
	\end{itemize}
}


\mframe{Jastrow Factor}{}{
The Jastrow factor is added to account for the correlations

Simple Jastrow factor
\begin{equation}
J(\bs{r}; \bs{\beta}) = \exp(\sum_{i=1}^N\sum_{j>i}^N{\beta_{ij}r_{ij}}).
\end{equation}

Padé-Jastrow factor
\begin{equation}
J(\bs{r};\beta) = \exp(\sum_{i=1}^N\sum_{j>i}^N\frac{a_{ij}r_{ij}}{1+\beta r_{ij}}).
\end{equation}
}

\note{
	\begin{itemize}
		\item Two Jastrow factors investigated
		\item Interesting as we want to see how much physical intuition we need to get acceptable results
		\item PJ is a complication of the simple Jastrow
	\end{itemize}
}

\mframe{Our Trial Wave Function Ansätze}{}{
	\begin{itemize}
		\setlength\itemsep{3em}
		\item $\Psi_{\text{RBM}}(\bs{R})=|\hat{D}_{\text{RBM}}(\bs{R})|$
		\item $\Psi_{\text{RBM+SJ}}(\bs{R})=|\hat{D}_{\text{RBM}}(\bs{R})|J(\bs{R};\bs{\beta})$
		\item $\Psi_{\text{RBM+PJ}}(\bs{R})=|\hat{D}_{\text{RBM}}(\bs{R})|J(\bs{R};\beta)$
		\item $\Psi_{\text{VMC}}(\bs{R})=|\hat{D}_{\text{Gauss}}(\bs{R})|J(\bs{R};\beta)$
	\end{itemize}
}

\note{
	Present ansätze
}