\documentclass{beamer}

 \setlength{\TPHorizModule}{10mm}
 \setlength{\TPVertModule}{\TPHorizModule}
 \textblockorigin{1mm}{1mm} % start everything near the top-left corner
 \setlength{\parindent}{0pt}

\beamertemplatetransparentcovereddynamic
\beamertemplatesolidbuttons
\beamertemplateroundedblocks
\definecolor{azul}{RGB}{1,80,159}
\definecolor{azulclaro}{RGB}{230,230,255}	
\setbeamercolor{titlelike}{fg=azul}
\setbeamercolor{block title}{fg=white}
\setbeamercolor{block title}{bg=azul}
\setbeamercolor{block body}{bg=azulclaro}

%%%%%%%%%%%%%%%%%%%%%%%%%%%%%%%%%%%%%%%%%%%%%%%%%%%%%%%%%%%%%%%%%%%%%%%%%%%%%%%%%%%%%%%%%%%%%%%%%%%%%%%%%%%%%%%%%%%%%
% ORDINARY FRAME
%%%%%%%%%%%%%%%%%%%%%%%%%%%%%%%%%%%%%%%%%%%%%%%%%%%%%%%%%%%%%%%%%%%%%%%%%%%%%%%%%%%%%%%%%%%%%%%%%%%%%%%%%%%%%%%%%%%%%
\newcommand{\mframe}[1]{
\frame{

% PROGRESSION WHEEL 
\begin{textblock}{3}(0,8.5)
\begin{tikzpicture}[x=1mm,y=1mm]
	 \foreach \x in {1,2,...,\inserttotalframenumber} 
	 					\draw [gray!30,very thick] (\x/\inserttotalframenumber*360:3) -- (\x/\inserttotalframenumber*360:4);
   \foreach \x in {1,2,...,\insertframenumber} 
   					\draw[azul,very thick] (\x/\inserttotalframenumber*360:3) -- (\x/\inserttotalframenumber*360:4);
   \node at (0,0) [] {\tiny{\textbf{\insertframenumber}}};
	\end{tikzpicture}
\end{textblock}

% UIO LOGO
\begin{textblock}{3}(10.3,0)
	\centering
	\includegraphics[scale=.3]{../Images/UiO_Segl_pms485.eps}
\end{textblock}

#1}
}

%%%%%%%%%%%%%%%%%%%%%%%%%%%%%%%%%%%%%%%%%%%%%%%%%%%%%%%%%%%%%%%%%%%%%%%%%%%%%%%%%%%%%%%%%%%%%%%%%%%%%%%%%%%%%%%%%%%%%
% TITLED FRAME
%%%%%%%%%%%%%%%%%%%%%%%%%%%%%%%%%%%%%%%%%%%%%%%%%%%%%%%%%%%%%%%%%%%%%%%%%%%%%%%%%%%%%%%%%%%%%%%%%%%%%%%%%%%%%%%%%%%%%
\newcommand{\titleframe}[1]{
\mframe{
	\begin{center}
		#1
	\end{center}
}
}
\usepackage[utf8]{inputenc}
\usepackage[english]{babel}
%\usepackage{refcheck}
%\usepackage[square,sort,comma,numbers]{natbib}
\usepackage{csquotes}	% Supports babel
\usepackage{graphicx} %for å inkludere grafikk
\usepackage{verbatim} %for å inkludere filer med tegn LaTeX ikke liker
\usepackage{tabularx}
\usepackage{booktabs}
\usepackage{amsmath}
\usepackage{mathbbol}
\usepackage{float}		% fix object
\usepackage{pdflscape}	% landscape page
\usepackage{afterpage}	% avoid undesired blank page
\usepackage{color}
\usepackage{array} 		% For column alignment
\usepackage{pgfplots}	% Plotting directly in latex
\usepackage{listings}
\usepackage{hyperref}
\usepackage{amsmath}
\usepackage{tikz}
\usepackage{physics}	% Dirac notation
\usepackage{amssymb}
\usepackage{titlesec}
\usepackage{comment}
\usepackage{makecell}	% New line in cell
\usepackage{fancybox}	% Oval equation box
\usepackage{enumitem}	% Itemize settings
\usepackage{subfig}		% Sub figures
\usepackage{empheq}		% Beautiful boxes
\usepackage{multirow}	% Multirow in table
\usepackage{multicol}	% Multicolumn in table
\usepackage{varwidth}	% Rotate text in table
\usepackage{arydshln}   % Dashed lines in table
\usepackage{epigraph}	% Quote at beg. of chapter
\usepackage{framed}
\usepackage{xargs} 
\usepackage{copyrightbox}
\usepackage{mathpazo}
%\usepackage[outdir=../Images/]{epstopdf}
%\usepackage{eulervm}

\renewcommand{\floatpagefraction}{.8}%

% Colors
\definecolor{background}{rgb}{0.898,0.898,0.898}
\definecolor{color0}{rgb}{0.203921568627451,0.541176470588235,0.741176470588235}
\definecolor{color1}{rgb}{0.650980392156863,0.0235294117647059,0.156862745098039}
\definecolor{color2}{rgb}{0.47843137254902,0.407843137254902,0.650980392156863}
\definecolor{color3}{rgb}{0.274509803921569,0.470588235294118,0.129411764705882}
\definecolor{myblue}{rgb}{.8157, .9255, 1}

% Algorithm2e
\usepackage[linesnumbered,
			lined,
			commentsnumbered,
			ruled,
			vlined
]{algorithm2e}
\SetKwInOut{Parameter}{Parameter}
\SetKwInOut{Require}{Require}
\SetKwInOut{Data}{Data}

%biblatex
\usepackage[backend=bibtex,
			sorting=none,
			style=nature
]{biblatex}
\addbibresource{refs.bib}

%tikz
\setcounter{secnumdepth}{4}
\usetikzlibrary{through, shapes, calc, shapes, arrows, positioning, arrows.meta, shadows, er, spy, external} 
\tikzstyle{neuron}=[draw,circle,minimum size=20pt,inner sep=0pt, fill=white]
\tikzstyle{stateTransition}=[thick]
\tikzstyle{learned}=[text=red]

\tikzset{basic/.style={draw,fill=blue!20,text width=1em,text badly centered}}
\tikzset{input/.style={basic,circle,fill=myblue}}
\tikzset{weights/.style={basic,rectangle,fill=myblue!90}}
\tikzset{functions/.style={basic,circle,fill=myblue!80}}

%\tikzexternalize[prefix=../figures/]

%empheq color box
\newlength\mytemplen
\newsavebox\mytempbox

\makeatletter
\newcommand\mybluebox{%
	\@ifnextchar[%]
	{\@mybluebox}%
	{\@mybluebox[0pt]}}

\def\@mybluebox[#1]{%
	\@ifnextchar[%]
	{\@@mybluebox[#1]}%
	{\@@mybluebox[#1][0pt]}}

\def\@@mybluebox[#1][#2]#3{
	\sbox\mytempbox{#3}%
	\mytemplen\ht\mytempbox
	\advance\mytemplen #1\relax
	\ht\mytempbox\mytemplen
	\mytemplen\dp\mytempbox
	\advance\mytemplen #2\relax
	\dp\mytempbox\mytemplen
	\colorbox{myblue}{\hspace{1em}\usebox{\mytempbox}\hspace{1em}}}

%double quotes
\newcommand*\openquote{\makebox(25,-5){\scalebox{5}{``}}}
\newcommand*\closequote{\makebox(25,-22){\scalebox{5}{''}}}
\colorlet{shadecolor}{myblue}
\makeatletter
\newif\if@right
\def\shadequote{\@righttrue\shadequote@i}
\def\shadequote@i{\begin{snugshade}\begin{quote}\openquote}
		\def\endshadequote{%
			\if@right\hfill\fi\closequote\end{quote}\end{snugshade}}
\@namedef{shadequote*}{\@rightfalse\shadequote@i}
\@namedef{endshadequote*}{\endshadequote}
\makeatother

% ngg symbol
\newcommand{\ngg}[1]{%
	\begin{tikzpicture}[scale=#1]%
	\draw (2ex,4ex) -- (8ex,8ex);%
	\draw (2ex,12ex) -- (8ex,8ex);%
	\draw (6ex,4ex) -- (12ex,8ex);%
	\draw (6ex,12ex) -- (12ex,8ex);%
	\draw (5ex,3ex) -- (9ex,13ex);%
	\end{tikzpicture}%
}

%chapter heading
\titleformat{\chapter}[display]
{\normalsize \rmfamily \huge \color{black}}
{\normalsize \color{color0} \MakeUppercase { \chaptertitlename } \hspace{1 ex} { \fontsize{60}{60}\selectfont \color{color0} \sffamily \thechapter }} {10 pt}{\huge}

%month only
\newcommand*{\mdate}{%
	\ifcase \month\or January\or February\or March\or April\or May\or June\or July\or
	August\or September\or October\or November\or December\fi \space \number\year}

%specific reused text
\newcommand{\mtitle}{Studies of Quantum Dots using Machine Learning}
\newcommand{\mauthor}{Even Marius Nordhagen}
\newcommand{\massignn}{.}

\newcommand{\prtl}{\mathrm{\partial}} %reduce length of partial (less to write)
% \NewDocumentCommand{\prd}{m O{} O{}}{\frac{\prtl^{#3}{#2}}{\prtl{#1}^{#3}}}
\newcommand{\prdp}[2]{\left(\frac{\prtl}{\prtl #1}\right)^{#2}}
\newcommand{\vsp}{\vspace{0.15cm}} %small vertical space
\newcommand{\txtit}[1]{\textit{{#1}}} %italic text
\newcommand{\blds}[1]{\boldsymbol{{#1}}} % better bold in mathmode (from amsmath)
\newcommand{\bigO}{\mathcal{O}} %nice big O
\newcommand{\me}{\mathrm{e}} %straight e for exp
\newcommand{\md}{\mathrm{d}} %straight d for differential
\newcommand{\mRe}[1]{\mathrm{Re}\left({#1}\right)}%nice real
\newcommand{\munit}[1]{\;\ensuremath{\, \mathrm{#1}}} %straight units in math
\newcommand{\Rarr}{\Rightarrow} %reduce lenght of Rightarrow (less to write)
\newcommand{\rarr}{\rightarrow} %reduce lenght of rightarrow (less to write)
\newcommand{\ecp}[1]{\left< {#1} \right>} %expected value
\newcommand{\urw}{\uparrow} % up arrow
\newcommand{\drw}{\downarrow} % down arrow
\newcommand{\pt}[1]{\textbf{\txtit{#1}}\justify}
\newcommand{\infint}{\int\limits^{\infty}_{-\infty}}
\newcommand{\oinfint}{\int\limits^{\infty}_0}
\newcommand{\sint}{\int\limits^{2\pi}_0\int\limits^{\pi}_0\oinfint}
\newcommand{\arcsinh}[1]{\text{arcsinh}\left(#1\right)}
\newcommand{\I}{\scalebox{1.2}{$\mathds{1}$}}
\newcommand{\veps}{\varepsilon} %\varepsilon is to long :P
\newcommand{\cnj}[1]{{#1}^{*}}
\newcommand{\Arf}[1]{\Autoref{#1}}
\newcommand{\suml}[2]{\sum\limits^{#2}_{#1}}
\newcommand{\sumll}[1]{\sum\limits_{#1}}
\newcommand{\tsup}[1]{\textsuperscript{#1}}
\newcommand{\wtld}[1]{\widetilde{#1}}

\newcommand{\ufij}[3]{#1_{#2\rarr#3}}
\newcommand{\Ham}{\hat{H}}
\newcommand{\mb}[1]{\blds{#1}}
\newcommand{\psiTcnj}{\cnj{\Psi}_T(\mb{R};\mb{\alpha})}
\newcommand{\psiT}{{\Psi}_T(\mb{R};\mb{\alpha})}
\newcommand{\phiT}{{\Phi}_T(\mb{R};\mb{\alpha})}
\newcommand{\dinner}[2]{\bra{#1}#2\ket{#1}}
\newcommand{\pinner}{\dinner{\Psi_T}{}}
\newcommand{\langevin}{\prd{t}[r] = DF(r(t)) + \eta}
\newcommand{\rnew}{r^{\text{new}}}
\newcommand{\rold}{r^{\text{old}}}
\newcommand{\Fnew}{F^{\text{new}}}
\newcommand{\Fold}{F^{\text{old}}}
\newcommand{\FokkerPlanck}{\prd{t}[P] = \sum_i D\prd{x_i}\left(\prd{x_i} - \mb{F_i}\right)P}
\newcommand{\Kin}{\frac{1}{2}\sum_i\nabla^2_i}
\newcommand{\frij}{f(\blds{r}_i, \blds{r}_j)}
\newcommand{\fij}{f_{ij}}
\newcommand{\HO}{V(\blds{R}) - \Kin}
\newcommand{\HI}{\sum\limits_{i<j} \frij}
\newcommand{\EHF}{E\left[\Psi^{\text{HF}}\right]}
\newcommand{\HIinnerAS}[2]{\bra{\psi_{#1}\psi_{#2}}H_I\ket{\psi_{#1}\psi_{#2}} - \bra{\psi_{#1}\psi_{#2}}H_I\ket{\psi_{#2}\psi_{#1}}}

\newcommand{\ijnorm}[2]{\sqrt{\braket{#1}{#1}\braket{#2}{#2}}}
\newcommand{\twoDI}{I_{\text{2D}}}
\newcommand{\sumE}[3]{\suml{#1}{#2+#3} E^{#2#3}_{#1}}

\newcommand\CC{C\nolinebreak[4]\hspace{-0.01em}\raisebox{.3ex}{\relsize{-1.35}{\textbf{++}}}\;}
\newcommand{\pp}[1]{#1\nolinebreak[4]\hspace{-.01em}\raisebox{.25ex}{\relsize{-1.5}{\textbf{++}}}\;}

\newcommand{\psiDW}{\psi^{\text{DW}}}
\newcommand{\psiHO}{\psi^{\text{HO}}}
\newcommand{\hDW}{h^{\text{DW}}}
\newcommand{\VDW}{V^{\text{DW}}}
\newcommand{\VHO}{V^{\text{HO}}}
\newcommand{\UDW}{U^{\text{DW}}}

% Define new column alignment types
\newcolumntype{L}[1]{>{\raggedright\let\newline\\\arraybackslash\hspace{0pt}}m{#1}}
\newcolumntype{R}[1]{>{\raggedleft\let\newline\\\arraybackslash\hspace{0pt}}m{#1}}
\newcolumntype{C}{>{\centering\let\newline\\\arraybackslash\hspace{0pt}}X}

% OWN DEFS
\newcommand{\bs}[1]{\boldsymbol{{#1}}}
\newcommand{\detup}{\det(\hat{D}^{\uparrow})}
\newcommand{\detdn}{\det(\hat{D}^{\downarrow})}
\newcommand{\hatH}{\hat{\text{H}}}
\newcommand{\psin}{\Psi_n(\bs{x})}
\newcommand{\psinc}{\Psi_n^*(\bs{x})}
\newcommand{\VMC}{Variational Monte Carlo}
\newcommand{\hatT}{\hat{T}}
\newcommand{\detD}{|\hat{D}|}

\newcommand{\citet}[1]{\citeauthor{#1} \cite{#1}}


\author{Even Marius Nordhagen} 
\institute[UiO] { University of Oslo \\ 
                  \medskip
                  \textit{evenmn@fys.uio.no}
                }
\date{\today} 
\title{Studies of Quantum Dots using Machine Learning}

\begin{document}

\begin{frame}
\titlepage 
\end{frame}

\mframe{\frametitle{Outline}
\begin{itemize}
	\item Theory
	\begin{itemize}
		\item Quantum Mechanics
		\item Machine Learning
	\end{itemize}
	\item Methods
	\begin{itemize}
		\item Variational Monte Carlo
		\item Restricted Boltzmann Machines
	\end{itemize}
	\item Results
	\item Conclusion
	\item (Code)
\end{itemize}
}

\mframe{
	\begin{center}
		Theory
	\end{center}
}

\mframe{\frametitle{Quantum Mechanics}
The mechanics of a quantum system is described by the time-dependent Schrödinger equation,
\begin{empheq}[box={\mybluebox[5pt]}]{equation}
	i\hbar\frac{\partial}{\partial t}\ket{\Psi(\bs{r},t)}=\hat{\mathcal{H}}\ket{\Psi(\bs{r},t)}
\end{empheq}

}

\mframe{\frametitle{The Time-independent Schrödinger Equation}
For a stationary system, the Schrödinger equation reduces to
	\begin{equation}
	E_n=\frac{\int d\bs{r}\psinc\hat{\mathcal{H}}\psin}{\int d\bs{r}\psinc\psin}
	\end{equation}
which gives the energy of state $n$.
}


\mframe{\frametitle{Harmonic Oscillator}

\begin{equation}
\begin{aligned}
\hat{\mathcal{H}} &= -\frac{1}{2}\nabla^2 + \frac{1}{2} \omega^2r^2\\
&= \langle\hat{T}\rangle + \langle\hat{V}\rangle
\end{aligned}
\end{equation}

\begin{figure}
	\centering
	\begin{tikzpicture}
\begin{axis}[domain=-3.5:3.5, samples=50,no markers, hide axis,y=1cm,thick]
\addplot [gray] {0.5*x^2};
\node[label={180:{(-2,2)}},circle,fill,inner sep=2pt] at (axis cs:-2,2) {};
\end{axis}
\end{tikzpicture}
\end{figure}

}

\mframe{\frametitle{Circular Quantum Dots}
	\begin{equation}
	\begin{aligned}
	\hat{\mathcal{H}} &= -\frac{1}{2}\nabla_1^2 -\frac{1}{2}\nabla_2^2 + \frac{1}{2} \omega^2r_1^2 + \frac{1}{2} \omega^2r_2^2 + \frac{1}{r_{ij}}\\
	&= \langle\hat{T}_1\rangle + \langle\hat{T}_2\rangle + \langle\hat{V}_1\rangle + \langle\hat{V}_2\rangle + \langle\hat{V}_{\text{int}}\rangle
	\end{aligned}
	\end{equation}
	\begin{figure}
		\centering
		\begin{tikzpicture}[declare function={f(\x)=0.5*\x^2;}]

\newcommand\Ho{2};			% Particle 1 height
\newcommand\Xo{-2};			% Horizontal 1 position
\newcommand\Ht{3.125};			% Particle 2 height
\newcommand\Xt{2.5};			% Horizontal 2 position
\newcommand\s{1};			% Dashed spacing

\begin{axis}[domain=-3:3, samples=50,no markers, hide axis,y=1cm,thick]
\addplot [gray] {0.5*x^2};

% Particle 1
\node[circle,fill,inner sep=4pt, color=color0] at (axis cs:\Xo+0.3,\Ho) {};

\draw[dashed] (axis cs:\Xo,\Ho) -- (axis cs:\Xo-\s,\Ho);
\draw[<->] (axis cs:\Xo-\s,\Ho) -- (axis cs:\Xo-\s,0);
\node at (axis cs:\Xo-\s+0.5,\Ho/2) {$\langle\hat{V}_1\rangle$};

\draw[->] (axis cs:\Xo+0.5,\Ho-0.3) -- (axis cs:\Xo+1.2,\Ho-1.2);
\node at (axis cs:-0.7,1.3) {$\langle\hat{T}_1\rangle$};

% Particle 2
\node[circle,fill,inner sep=4pt, color=color0] at (axis cs:\Xt-0.3,\Ht) {};

\draw[dashed] (axis cs:\Xt,\Ht) -- (axis cs:\Xt+\s,\Ht);
\draw[<->] (axis cs:\Xt+\s,\Ht) -- (axis cs:\Xt+\s,0);
\node at (axis cs:\Xt+\s-0.5,\Ht/2) {$\langle\hat{V}_2\rangle$};

\draw[->] (axis cs:\Xt-0.5,\Ht-0.3) -- (axis cs:\Xt-1,\Ht-1.2);
\node at (axis cs:1,2.2) {$\langle\hat{T}_2\rangle$};

% Interaction
\draw[<->] (axis cs:\Xt-0.6,\Ht-0.1) -- (axis cs:\Xo+0.6,\Ho+0.1);
\node at (axis cs:0,3) {$\langle\hat{V}_{\text{int}}\rangle$};

\end{axis}
\end{tikzpicture}
	\end{figure}
	
}

\mframe{\frametitle{Machine Learning}
	
	\begin{shadequote}{}
		Machine learning is the science of getting computers to act without being explicitly programmed.
	\end{shadequote}
	
}

\mframe{\frametitle{Feed-forward Neural Network}
	\begin{figure}[scale=0.2]
		\centering
		\begin{tikzpicture}

% Define outputs
\node[] (center) {};
\node[input, above=0.3em of center] (o1) {$y_1$};
\node[input, below=0.3em of center] (o2) {$y_2$};

% Draw lines from output nodes
\node[right of=o1] (righto1) {};
\node[right of=o2] (righto2) {};
\path[draw,->] (o1) -- (righto1);
\path[draw,->] (o2) -- (righto2);

% Hidden nodes
\node[input,left=5em of center] (h3) {$h_3$};
\node[input,above of=h3] (h2) {$h_2$};
\node[input,above of=h2] (h1) {$h_1$};
\node[input,below of=h3] (h4) {$h_4$};
\node[input,below of=h4] (h5) {$h_5$};
\node[input,above of=h1] (b2) {$B_2$};

% Draw lines from hidden nodes
\path[draw,->] (h1) -- (o1);
\path[draw,->] (h2) -- (o1);
\path[draw,->] (h3) -- (o1);
\path[draw,->] (h4) -- (o1);
\path[draw,->] (h5) -- (o1);
\path[draw,->] (b2) -- (o1);

\path[draw,->] (h1) -- (o2);
\path[draw,->] (h2) -- (o2);
\path[draw,->] (h3) -- (o2);
\path[draw,->] (h4) -- (o2);
\path[draw,->] (h5) -- (o2);
\path[draw,->] (b2) -- (o2);

% Define place left of left
\node[input,left=5em of h3] (x2) {$x_2$};
\node[input,above of=x2] (x1) {$x_1$};
\node[input,below of=x2] (x3) {$x_3$};
\node[input,above of=x1] (b1) {$B_1$};

% Draw lines from input nodes
\path[draw,->] (x1) -- (h1);
\path[draw,->] (x1) -- (h2);
\path[draw,->] (x1) -- (h3);
\path[draw,->] (x1) -- (h4);
\path[draw,->] (x1) -- (h5);

\path[draw,->] (x2) -- (h1);
\path[draw,->] (x2) -- (h2);
\path[draw,->] (x2) -- (h3);
\path[draw,->] (x2) -- (h4);
\path[draw,->] (x2) -- (h5);

\path[draw,->] (x3) -- (h1);
\path[draw,->] (x3) -- (h2);
\path[draw,->] (x3) -- (h3);
\path[draw,->] (x3) -- (h4);
\path[draw,->] (x3) -- (h5);

\path[draw,->] (b1) -- (h1);
\path[draw,->] (b1) -- (h2);
\path[draw,->] (b1) -- (h3);
\path[draw,->] (b1) -- (h4);
\path[draw,->] (b1) -- (h5);

% Draw lines towards input nodes
\node[left of=x1] (leftx1) {};
\node[left of=x2] (leftx2) {};
\node[left of=x3] (leftx3) {};
\path[draw,->] (leftx1) -- (x1);
\path[draw,->] (leftx2) -- (x2);
\path[draw,->] (leftx3) -- (x3);

% Add some text
\node[below=6.1em of x2] {input};
\node[below=6em of h3] {hidden};
\node[below=6.8em of center] {output};
\end{tikzpicture}
	\end{figure}
}

\mframe{\frametitle{Restricted Boltzmann machines}
	\begin{figure}
		\centering
		\begin{tikzpicture}

% Define visible units
\node[input] (x2) {$x_2$};
\node[input] at (1,+3/2) (x1) {$x_1$};
\node[input] at (1,-3/2) (x3) {$x_3$};

% Define hidden units
\node[input] at (3,+3/2) (h1) {$h_1$};
\node[input] at (4,0) (h2) {$h_2$};
\node[input] at (3,-3/2) (h3) {$h_3$};

% Define biases
\node[input] at (0,7/2) (a) {$1$};
\node[input] at (4,7/2) (b) {$1$};

% Define paths
\path[draw,thick,-] (x1) -- (h1) node[midway,above] {$w_{11}$};
\path[draw,thick,-] (x1) -- (h2);
\path[draw,thick,-] (x1) -- (h3);

\path[draw,thick,-] (x2) -- (h1);
\path[draw,thick,-] (x2) -- (h2);
\path[draw,thick,-] (x2) -- (h3);

\path[draw,thick,-] (x3) -- (h1);
\path[draw,thick,-] (x3) -- (h2);
\path[draw,thick,-] (x3) -- (h3);

\draw[color2,thick,-] (b) to [out=270,in=90] (h1);
\draw[color2,thick,-] (b) to [out=315,in=45] (h2);
\draw[color2,thick,-] (b) to [out=0,in=0] (h3) node[] at (6,1) {$b_3$};

\draw[color1,thick,-] (a) to [out=270,in=90] (x1);
\draw[color1,thick,-] (a) to [out=225,in=135] (x2);
\draw[color1,thick,-] (a) to  [out=180,in=180] (x3) node[] at (-2,1) {$a_3$};


% Add some text
\node[below=1em of x3] {visible};
\node[below=1em of h3] {hidden};
\end{tikzpicture}

	\end{figure}
}

\mframe{
	\begin{center}
		Methods
	\end{center}
}

\mframe{\frametitle{Variational Monte Carlo (VMC)}
Exploit the variational principle in order to obtain the ground state energy
	\begin{equation}
	\begin{aligned}
	E_0 < E_{\text{VMC}} &= \frac{\int d\bs{r}\Psi_T(\bs{r})^*\hat{\mathcal{H}}\Psi_T(\bs{r})}{\int d\bs{r}\Psi_T(\bs{r})^*\Psi_T(\bs{r})}\\
	&= \int d\bs{r}\underbrace{\frac{\Psi_T(\bs{r})^*\Psi_T(\bs{r})}{\int d\bs{r}\Psi_T(\bs{r})^*\Psi_T(\bs{r})}}_{P(\bs{r})}\cdot\underbrace{\frac{1}{\Psi_T(\bs{r})}\hat{\mathcal{H}}\Psi_T(\bs{r})}_{E_L(\bs{r})}
	\end{aligned}
	\end{equation}
	
}

\mframe{\frametitle{Monte Carlo Integration}
	We attempt to solve the integral by sampling from the probability density function $P(\bs{r})$
	\begin{equation}
	\begin{aligned}
	E_{\text{VMC}} &= \int d\bs{r} E_L(\bs{r})P(\bs{r}) \\
	&\approx\frac{1}{M}\sum_{i=1}^ME_L(\bs{r}_i)
	\end{aligned}
	\end{equation}
}

\mframe{\frametitle{Trial Wave Function}
	\begin{equation}
	P(\bs{r})\propto\Psi_T(\bs{r})^*\Psi_T(\bs{r})
	\end{equation}
	
	Use the Slater-Jastrow function as our trial wave function
	\begin{equation}
	\Psi_T(\bs{r})=|\hat{D}(\bs{r})|J(\bs{r})
	\end{equation}
	where the Slater matrix, $\hat{D}(\bs{r})$, contains all the single-particle functions
	\begin{equation}
	\hat{D}(\bs{r})=
	\begin{pmatrix}
	\phi_1(\boldsymbol{r}_1) & \phi_2(\boldsymbol{r}_1) & \hdots & \phi_N(\boldsymbol{r}_1)\\
	\phi_1(\boldsymbol{r}_2) & \phi_2(\boldsymbol{r}_2) & \hdots & \phi_N(\boldsymbol{r}_2)\\
	\vdots & \vdots & \ddots & \vdots \\
	\phi_1(\boldsymbol{r}_N) & \phi_2(\boldsymbol{r}_N) & \hdots & \phi_N(\boldsymbol{r}_N)
	\end{pmatrix}
	\end{equation}
}

\mframe{\frametitle{Single-particle Functions}
	The Hermite functions,
	\begin{equation}
	\phi_n(\bs{r})\propto H_n(\bs{r})\exp(-\frac{1}{2}\alpha\omega|\bs{r}|^2),
	\end{equation}
	are used as the single-particle functions for quantum dots in standard VMC. The Gaussian can be factorized out from the Slater determinant.
	\begin{equation}
	|\hat{D}(\bs{r};\alpha)|\propto\exp(-\frac{1}{2}\alpha\omega|\bs{r}|^2)
	\begin{vmatrix}
	H_1(\boldsymbol{r}_1) & H_2(\boldsymbol{r}_1) & \hdots & H_N(\boldsymbol{r}_1)\\
	H_1(\boldsymbol{r}_2) & H_2(\boldsymbol{r}_2) & \hdots & H_N(\boldsymbol{r}_2)\\
	\vdots & \vdots & \ddots & \vdots \\
	H_1(\boldsymbol{r}_N) & H_2(\boldsymbol{r}_N) & \hdots & H_N(\boldsymbol{r}_N)
	\end{vmatrix}
	\end{equation}
}

\mframe{\frametitle{Restricted Boltzmann Machine}
	We use the marginal distribution of the visible units as the single-particle functions in the Slater determinant, and see if them can model the correlations 
	\begin{equation}
	\phi_n(\bs{r})\propto H_n(\bs{r})P(\bs{r};\bs{a},\bs{b},\bs{W})
	\end{equation}
	where $P(\bs{r})$ is the marginal distribution of the visible units.
	\begin{equation}
	|\hat{D}(\bs{r};\bs{a},\bs{b},\bs{W})|\propto P(\bs{r};\bs{a},\bs{b},\bs{W})
	\begin{vmatrix}
	H_1(\boldsymbol{r}_1) & H_2(\boldsymbol{r}_1) & \hdots & H_N(\boldsymbol{r}_1)\\
	H_1(\boldsymbol{r}_2) & H_2(\boldsymbol{r}_2) & \hdots & H_N(\boldsymbol{r}_2)\\
	\vdots & \vdots & \ddots & \vdots \\
	H_1(\boldsymbol{r}_N) & H_2(\boldsymbol{r}_N) & \hdots & H_N(\boldsymbol{r}_N)
	\end{vmatrix}
	\end{equation}
}


\mframe{\frametitle{Jastrow Factor}
The Jastrow factor is added to account for the correlations

Simple Jastrow factor
\begin{equation}
J(\bs{r}; \bs{\beta}) = \exp\left(\sum_{i=1}^N\sum_{j>i}^N{\beta_{ij}r_{ij}}\right).
\end{equation}

Padé-Jastrow factor
\begin{equation}
J(\bs{r};\beta) = \exp\bigg(\sum_{i=1}^N\sum_{j>i}^N\frac{a_{ij}r_{ij}}{1+\beta r_{ij}}\bigg).
\end{equation}
}

\mframe{
	\begin{center}
		Results
	\end{center}
}

\iffalse
\mframe{\frametitle{Ground State Energy}
	\begin{tabularx}{\textwidth}{lR{1.5cm}R{1.6cm}R{1.6cm}R{1.5cm}R{1.3cm}} \hline\hline
		\label{tab:cputimefit}
		\makecell{$\omega$ \\ \phantom{=}} & RBM & RBM+SJ & RBM+PJ & HF\footnote{Produced by the HF code developed by \citeauthor{mariadason_quantum_2018}} & Exact\footnote{\citeauthor{taut_two_1993}, \citetitle{taut_two_1993}, \citeyear{taut_two_1993}}} \\ \hline \\
	1/6 & 0.7036(1) & 0.67684(7) & 0.66715(6) & 0.768675 & 2/3 \\ 
	1 & 3.0803(2) & 3.02108(5) & 2.999587(5) & 3.1690 & 3 \\ \hline\hline
\end{tabularx}
}
\fi

\mframe{\frametitle{Computational Cost}
	\begin{figure}
		\centering 
		% This file was created by matplotlib2tikz v0.7.4.
\begin{tikzpicture}[scale=0.9]

\begin{axis}[name=2D, xlabel=$N$, ylabel={CPU-time [s]}, grid=major, legend pos=north west, xtick=data] 
\addplot[color=color0,mark=oplus*, dashed] coordinates { 
	(2,6.05)
	(6,11.25)
	(12,20.53) 
	(20,38.99) 
	(30,73.73) 
	(42,130.49) 
	(56,213.47)
	(72,360.22)
	(90,856.84) }; 
\addlegendentry{RBM};

\addplot[color=color1,mark=oplus*, dash dot] coordinates { 
	(2,7.12) 
	(6,14.07) 
	(12,28.42) 
	(20,63.27) 
	(30,122.93) 
	(42,199.60)
	(56,349.22)}; 
\addlegendentry{RBM+SJ};

\addplot[color=color2,mark=oplus*, dotted] coordinates { 
	(2,7.26)
	(6,13.50)
	(12,27.68)
	(20,57.09) 
	(30,119.17) 
	(42,212.53) 
	(56,382.13) }; 
\addlegendentry{RBM+PJ};

\addplot[color=color3,mark=oplus*] coordinates { 
	(2,5.11)
	(6,10.51)
	(12,20.85) 
	(20,41.20) 
	(30,76.26) 
	(42,137.39) 
	(56,230.63)
	(72,355.81)
	(90,544.03) }; 
\addlegendentry{VMC};
\end{axis}
\end{tikzpicture}
	\end{figure} 
}

\mframe{
	\begin{center}
		Conclusion
	\end{center}
}

\mframe{\frametitle{Retrospect}
	\begin{itemize}
		\item RBM is able to account for most of the correlations
		\item RBM+PJ implies to give a lower ground state energy and model the correlations better than a traditional VMC
		\item RBM+SJ is both more expensive and less accurate than its fellow methods, and we see no reason to choose it
	\end{itemize}
}

\mframe{\frametitle{Future Work}
	\begin{itemize}
		\item Repeat the exercise using spherical coordinates - interactions are easier to model in spherical coordinates
		\item Check the ability of modeling the three-body correlations, considering nuclear systems
		\item Reduce the computational cost
	\end{itemize}
}

\mframe{
	\begin{center}
		Thank you!
	\end{center}
}

\end{document}