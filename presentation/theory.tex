\titleframe{Quantum Theory}

\note{Now we will give a breif introduction to the essential quantum theory.}

\mframe{The Schrödinger Equation}{}{
	\begin{empheq}[box={\mybluebox[5pt]}]{equation}
	\hat{\mathcal{H}}\Psi=E\Psi
	\end{empheq}
	\pause
	\vspace{0.2cm}
	\begin{center}
		{\large $\Downarrow$}
	\end{center}
	\vspace{0.3cm}
	\begin{equation}
	E=\frac{\int d\bs{X}\Psi^*(\bs{X})\hat{\mathcal{H}}\Psi(\bs{X})}{\int d\bs{X}\Psi^*(\bs{X})\Psi(\bs{X})}
	\end{equation}
	\note<1->{
		\begin{itemize}
			\item Describes the mechanics of all QM systems
			\item Stationary systems $\rightarrow$ Time-independent SE
			\item Linear algebra terms
			\item Configuration interaction
			\item One year on solving 
			\item Difficult to solve bco interactions between particles
		\end{itemize}
	}
}

\iffalse
\note{The Schrödinger equation describes the motion of any quantum mechanical system, and is the equation that I have spent one year solving. Since we will limit us to stationary systems only, the time-independent Schrödinger equation will be our focus. In linear algebra terms, it is an eigenvalue equation with the Hamilton operator, $\hat{\mathcal{H}}$, as a matrix and the wave function, $\Psi$, as the eigen function. $E$ is the energy, which is the eigenvalue.

\vspace{0.5cm}
The most natural way of solving this equation, is to simply express the Hamiltonian as a matrix and obtain the wave function and the energy from diagonalizing the matrix. This is known as configuration interaction. However, we can only do this for small systems, as it is very computational intensive. 
}

\note{Instead, we separate the equation with respect to the energy, and obtain a equation consisting of some integrals. This equation is hard to solve due to the electron-electron correlations. }
\fi

\mframe{The Variational Principle}{}{
	
	
	The variational principle serves as a way of finding the ground state energy. For an arbitrary trial wave function $\Psi_T(\bs{X})$, it states that the obtained energy is larger or equal to the ground state,
	\begin{equation}
	E_0\leq E=\frac{\int d\bs{X}\Psi_T^*(\bs{X})\hat{\mathcal{H}}\Psi_T(\bs{X})}{\int d\bs{X}\Psi_T^*(\bs{X})\Psi_T(\bs{X})}.
	\end{equation}
	Thus, by minimizing the obtained energy, $E$, we can estimate the ground state energy. 
}

\note{
	\begin{itemize}
		\item To obtain the ground state energy
		\item States $\rightarrow$ minimizing
	\end{itemize}
}

\mframe{Quantum Dots}{}{
	Circular quantum dots $\rightarrow$ electrons confined in a harmonic oscillator potential:
	\begin{equation} 		\hat{\mathcal{H}}=\sum_{i=1}^N\left[-\frac{1}{2}\nabla_i^2+\frac{1}{2}\omega^2|\bs{r}_i|^2+\sum_{j>i}^N\frac{1}{r_{ij}}\right].
	\end{equation}
	The number of electrons that give full shells are given by
	\begin{equation}
	N=2\binom{n+d}{d},
	\end{equation}
	which are the magic numbers. 
}

\note{
	\begin{itemize}
		\item Hamiltonian of the circular quantum dots consisting of electrons. In natural units.
		\item The magic numbers give the number of electron in each shell. Looked at closed-shell systems only
	\end{itemize}
}

\titleframe{Machine Learning Theory}

\note{Now over to the machine learning theory. We have already mentioned the artificial neural networks, and we will now look at how they actually work. }

\mframe{Feed-forward Neural Network (FNN)}{}{
	\begin{figure}
		\centering
		\begin{overprint}[9cm]
		\onslide<1>\begin{tikzpicture}

% Define outputs
\node[] (center) {};
\node[input, above=0.3em of center] (y1) {};
\node[input, below=0.3em of center] (y2) {};

% Draw lines from output nodes
\node[right of=y1] (righty1) {};
\node[right of=y2] (righty2) {};
\path[draw,->] (y1) -- (righty1);
\path[draw,->] (y2) -- (righty2);

% Hidden nodes L
\node[input,left=5em of center] (aL3) {};
\node[input,above of=aL3] (aL2) {};
\node[input,above of=aL2] (aL1) {};
\node[input,below of=aL3] (aL4) {};
\node[input,below of=aL4] (aL5) {};

% Hidden nodes 1
\node[input,left=15em of center] (a13) {};
\node[input,above of=a13] (a12) {};
\node[input,above of=a12] (a11) {};
\node[input,below of=a13] (a14) {};
\node[input,below of=a14] (a15) {};

% Draw lines from hidden nodes
\path[draw,->] (aL1) -- (y1);
\path[draw,->] (aL2) -- (y1);
\path[draw,->] (aL3) -- (y1);
\path[draw,->] (aL4) -- (y1);
\path[draw,->] (aL5) -- (y1);

\path[draw,->] (aL1) -- (y2);
\path[draw,->] (aL2) -- (y2);
\path[draw,->] (aL3) -- (y2);
\path[draw,->] (aL4) -- (y2);
\path[draw,->] (aL5) -- (y2);

% Define place left of left
\node[input,left=5em of a13, fill=red] (x2) {};
\node[input,above of=x2, fill=red] (x1) {};
\node[input,below of=x2, fill=red] (x3) {};

% Draw lines from input nodes
\path[draw,->] (x1) -- (a11);
\path[draw,->] (x1) -- (a12);
\path[draw,->] (x1) -- (a13);
\path[draw,->] (x1) -- (a14);
\path[draw,->] (x1) -- (a15);

\path[draw,->] (x2) -- (a11);
\path[draw,->] (x2) -- (a12);
\path[draw,->] (x2) -- (a13);
\path[draw,->] (x2) -- (a14);
\path[draw,->] (x2) -- (a15);

\path[draw,->] (x3) -- (a11);
\path[draw,->] (x3) -- (a12);
\path[draw,->] (x3) -- (a13);
\path[draw,->] (x3) -- (a14);
\path[draw,->] (x3) -- (a15);

% Draw lines from first hidden layer
\path[draw,->] (a11) -- (aL1);
\path[draw,->] (a11) -- (aL2);
\path[draw,->] (a11) -- (aL3);
\path[draw,->] (a11) -- (aL4);
\path[draw,->] (a11) -- (aL5);

\path[draw,->] (a12) -- (aL1);
\path[draw,->] (a12) -- (aL2);
\path[draw,->] (a12) -- (aL3);
\path[draw,->] (a12) -- (aL4);
\path[draw,->] (a12) -- (aL5);

\path[draw,->] (a13) -- (aL1);
\path[draw,->] (a13) -- (aL2);
\path[draw,->] (a13) -- (aL3);
\path[draw,->] (a13) -- (aL4);
\path[draw,->] (a13) -- (aL5);

\path[draw,->] (a14) -- (aL1);
\path[draw,->] (a14) -- (aL2);
\path[draw,->] (a14) -- (aL3);
\path[draw,->] (a14) -- (aL4);
\path[draw,->] (a14) -- (aL5);

\path[draw,->] (a15) -- (aL1);
\path[draw,->] (a15) -- (aL2);
\path[draw,->] (a15) -- (aL3);
\path[draw,->] (a15) -- (aL4);
\path[draw,->] (a15) -- (aL5);


% Draw lines towards input nodes
\node[left of=x1] (leftx1) {};
\node[left of=x2] (leftx2) {};
\node[left of=x3] (leftx3) {};
\path[draw,->] (leftx1) -- (x1);
\path[draw,->] (leftx2) -- (x2);
\path[draw,->] (leftx3) -- (x3); 

\end{tikzpicture}
		\onslide<2>\begin{tikzpicture}

% Define outputs
\node[] (center) {};
\node[input, above=0.3em of center] (y1) {};
\node[input, below=0.3em of center] (y2) {};

% Draw lines from output nodes
\node[right of=y1] (righty1) {};
\node[right of=y2] (righty2) {};
\path[draw,->] (y1) -- (righty1);
\path[draw,->] (y2) -- (righty2);

% Hidden nodes L
\node[input,left=5em of center] (aL3) {};
\node[input,above of=aL3] (aL2) {};
\node[input,above of=aL2] (aL1) {};
\node[input,below of=aL3] (aL4) {};
\node[input,below of=aL4] (aL5) {};

% Hidden nodes 1
\node[input,left=15em of center, fill=red] (a13) {};
\node[input,above of=a13, fill=red] (a12) {};
\node[input,above of=a12, fill=red] (a11) {};
\node[input,below of=a13, fill=red] (a14) {};
\node[input,below of=a14, fill=red] (a15) {};

% Draw lines from hidden nodes
\path[draw,->] (aL1) -- (y1);
\path[draw,->] (aL2) -- (y1);
\path[draw,->] (aL3) -- (y1);
\path[draw,->] (aL4) -- (y1);
\path[draw,->] (aL5) -- (y1);

\path[draw,->] (aL1) -- (y2);
\path[draw,->] (aL2) -- (y2);
\path[draw,->] (aL3) -- (y2);
\path[draw,->] (aL4) -- (y2);
\path[draw,->] (aL5) -- (y2);

% Define place left of left
\node[input,left=5em of a13] (x2) {};
\node[input,above of=x2] (x1) {};
\node[input,below of=x2] (x3) {};

% Draw lines from input nodes
\path[draw,->] (x1) -- (a11);
\path[draw,->] (x1) -- (a12);
\path[draw,->] (x1) -- (a13);
\path[draw,->] (x1) -- (a14);
\path[draw,->] (x1) -- (a15);

\path[draw,->] (x2) -- (a11);
\path[draw,->] (x2) -- (a12);
\path[draw,->] (x2) -- (a13);
\path[draw,->] (x2) -- (a14);
\path[draw,->] (x2) -- (a15);

\path[draw,->] (x3) -- (a11);
\path[draw,->] (x3) -- (a12);
\path[draw,->] (x3) -- (a13);
\path[draw,->] (x3) -- (a14);
\path[draw,->] (x3) -- (a15);

% Draw lines from first hidden layer
\path[draw,->] (a11) -- (aL1);
\path[draw,->] (a11) -- (aL2);
\path[draw,->] (a11) -- (aL3);
\path[draw,->] (a11) -- (aL4);
\path[draw,->] (a11) -- (aL5);

\path[draw,->] (a12) -- (aL1);
\path[draw,->] (a12) -- (aL2);
\path[draw,->] (a12) -- (aL3);
\path[draw,->] (a12) -- (aL4);
\path[draw,->] (a12) -- (aL5);

\path[draw,->] (a13) -- (aL1);
\path[draw,->] (a13) -- (aL2);
\path[draw,->] (a13) -- (aL3);
\path[draw,->] (a13) -- (aL4);
\path[draw,->] (a13) -- (aL5);

\path[draw,->] (a14) -- (aL1);
\path[draw,->] (a14) -- (aL2);
\path[draw,->] (a14) -- (aL3);
\path[draw,->] (a14) -- (aL4);
\path[draw,->] (a14) -- (aL5);

\path[draw,->] (a15) -- (aL1);
\path[draw,->] (a15) -- (aL2);
\path[draw,->] (a15) -- (aL3);
\path[draw,->] (a15) -- (aL4);
\path[draw,->] (a15) -- (aL5);


% Draw lines towards input nodes
\node[left of=x1] (leftx1) {};
\node[left of=x2] (leftx2) {};
\node[left of=x3] (leftx3) {};
\path[draw,->] (leftx1) -- (x1);
\path[draw,->] (leftx2) -- (x2);
\path[draw,->] (leftx3) -- (x3); 
\end{tikzpicture}
		\onslide<3>\begin{tikzpicture}

% Define outputs
\node[] (center) {};
\node[input, above=0.3em of center] (y1) {};
\node[input, below=0.3em of center] (y2) {};

% Draw lines from output nodes
\node[right of=y1] (righty1) {};
\node[right of=y2] (righty2) {};
\path[draw,->] (y1) -- (righty1);
\path[draw,->] (y2) -- (righty2);

% Hidden nodes L
\node[input,left=5em of center, fill=red] (aL3) {};
\node[input,above of=aL3, fill=red] (aL2) {};
\node[input,above of=aL2, fill=red] (aL1) {};
\node[input,below of=aL3, fill=red] (aL4) {};
\node[input,below of=aL4, fill=red] (aL5) {};

% Hidden nodes 1
\node[input,left=15em of center] (a13) {};
\node[input,above of=a13] (a12) {};
\node[input,above of=a12] (a11) {};
\node[input,below of=a13] (a14) {};
\node[input,below of=a14] (a15) {};

% Draw lines from hidden nodes
\path[draw,->] (aL1) -- (y1);
\path[draw,->] (aL2) -- (y1);
\path[draw,->] (aL3) -- (y1);
\path[draw,->] (aL4) -- (y1);
\path[draw,->] (aL5) -- (y1);

\path[draw,->] (aL1) -- (y2);
\path[draw,->] (aL2) -- (y2);
\path[draw,->] (aL3) -- (y2);
\path[draw,->] (aL4) -- (y2);
\path[draw,->] (aL5) -- (y2);

% Define place left of left
\node[input,left=5em of a13] (x2) {};
\node[input,above of=x2] (x1) {};
\node[input,below of=x2] (x3) {};

% Draw lines from input nodes
\path[draw,->] (x1) -- (a11);
\path[draw,->] (x1) -- (a12);
\path[draw,->] (x1) -- (a13);
\path[draw,->] (x1) -- (a14);
\path[draw,->] (x1) -- (a15);

\path[draw,->] (x2) -- (a11);
\path[draw,->] (x2) -- (a12);
\path[draw,->] (x2) -- (a13);
\path[draw,->] (x2) -- (a14);
\path[draw,->] (x2) -- (a15);

\path[draw,->] (x3) -- (a11);
\path[draw,->] (x3) -- (a12);
\path[draw,->] (x3) -- (a13);
\path[draw,->] (x3) -- (a14);
\path[draw,->] (x3) -- (a15);

% Draw lines from first hidden layer
\path[draw,->] (a11) -- (aL1);
\path[draw,->] (a11) -- (aL2);
\path[draw,->] (a11) -- (aL3);
\path[draw,->] (a11) -- (aL4);
\path[draw,->] (a11) -- (aL5);

\path[draw,->] (a12) -- (aL1);
\path[draw,->] (a12) -- (aL2);
\path[draw,->] (a12) -- (aL3);
\path[draw,->] (a12) -- (aL4);
\path[draw,->] (a12) -- (aL5);

\path[draw,->] (a13) -- (aL1);
\path[draw,->] (a13) -- (aL2);
\path[draw,->] (a13) -- (aL3);
\path[draw,->] (a13) -- (aL4);
\path[draw,->] (a13) -- (aL5);

\path[draw,->] (a14) -- (aL1);
\path[draw,->] (a14) -- (aL2);
\path[draw,->] (a14) -- (aL3);
\path[draw,->] (a14) -- (aL4);
\path[draw,->] (a14) -- (aL5);

\path[draw,->] (a15) -- (aL1);
\path[draw,->] (a15) -- (aL2);
\path[draw,->] (a15) -- (aL3);
\path[draw,->] (a15) -- (aL4);
\path[draw,->] (a15) -- (aL5);


% Draw lines towards input nodes
\node[left of=x1] (leftx1) {};
\node[left of=x2] (leftx2) {};
\node[left of=x3] (leftx3) {};
\path[draw,->] (leftx1) -- (x1);
\path[draw,->] (leftx2) -- (x2);
\path[draw,->] (leftx3) -- (x3); 
\end{tikzpicture}
		\onslide<4>\begin{tikzpicture}

% Define outputs
\node[] (center) {};
\node[input, above=0.3em of center, fill=red] (y1) {};
\node[input, below=0.3em of center, fill=red] (y2) {};

% Draw lines from output nodes
\node[right of=y1] (righty1) {};
\node[right of=y2] (righty2) {};
\path[draw,->, color=red] (y1) -- (righty1);
\path[draw,->, color=red] (y2) -- (righty2);

% Hidden nodes L
\node[input,left=5em of center] (aL3) {};
\node[input,above of=aL3] (aL2) {};
\node[input,above of=aL2] (aL1) {};
\node[input,below of=aL3] (aL4) {};
\node[input,below of=aL4] (aL5) {};

% Hidden nodes 1
\node[input,left=15em of center] (a13) {};
\node[input,above of=a13] (a12) {};
\node[input,above of=a12] (a11) {};
\node[input,below of=a13] (a14) {};
\node[input,below of=a14] (a15) {};

% Draw lines from hidden nodes
\path[draw,->] (aL1) -- (y1);
\path[draw,->] (aL2) -- (y1);
\path[draw,->] (aL3) -- (y1);
\path[draw,->] (aL4) -- (y1);
\path[draw,->] (aL5) -- (y1);

\path[draw,->] (aL1) -- (y2);
\path[draw,->] (aL2) -- (y2);
\path[draw,->] (aL3) -- (y2);
\path[draw,->] (aL4) -- (y2);
\path[draw,->] (aL5) -- (y2);

% Define place left of left
\node[input,left=5em of a13] (x2) {};
\node[input,above of=x2] (x1) {};
\node[input,below of=x2] (x3) {};

% Draw lines from input nodes
\path[draw,->] (x1) -- (a11);
\path[draw,->] (x1) -- (a12);
\path[draw,->] (x1) -- (a13);
\path[draw,->] (x1) -- (a14);
\path[draw,->] (x1) -- (a15);

\path[draw,->] (x2) -- (a11);
\path[draw,->] (x2) -- (a12);
\path[draw,->] (x2) -- (a13);
\path[draw,->] (x2) -- (a14);
\path[draw,->] (x2) -- (a15);

\path[draw,->] (x3) -- (a11);
\path[draw,->] (x3) -- (a12);
\path[draw,->] (x3) -- (a13);
\path[draw,->] (x3) -- (a14);
\path[draw,->] (x3) -- (a15);

% Draw lines from first hidden layer
\path[draw,->] (a11) -- (aL1);
\path[draw,->] (a11) -- (aL2);
\path[draw,->] (a11) -- (aL3);
\path[draw,->] (a11) -- (aL4);
\path[draw,->] (a11) -- (aL5);

\path[draw,->] (a12) -- (aL1);
\path[draw,->] (a12) -- (aL2);
\path[draw,->] (a12) -- (aL3);
\path[draw,->] (a12) -- (aL4);
\path[draw,->] (a12) -- (aL5);

\path[draw,->] (a13) -- (aL1);
\path[draw,->] (a13) -- (aL2);
\path[draw,->] (a13) -- (aL3);
\path[draw,->] (a13) -- (aL4);
\path[draw,->] (a13) -- (aL5);

\path[draw,->] (a14) -- (aL1);
\path[draw,->] (a14) -- (aL2);
\path[draw,->] (a14) -- (aL3);
\path[draw,->] (a14) -- (aL4);
\path[draw,->] (a14) -- (aL5);

\path[draw,->] (a15) -- (aL1);
\path[draw,->] (a15) -- (aL2);
\path[draw,->] (a15) -- (aL3);
\path[draw,->] (a15) -- (aL4);
\path[draw,->] (a15) -- (aL5);


% Draw lines towards input nodes
\node[left of=x1] (leftx1) {};
\node[left of=x2] (leftx2) {};
\node[left of=x3] (leftx3) {};
\path[draw,->] (leftx1) -- (x1);
\path[draw,->] (leftx2) -- (x2);
\path[draw,->] (leftx3) -- (x3); 

%\node[below=3em of y2] {$\tilde{\bs{y}}$};
\end{tikzpicture}
		\end{overprint}
	\end{figure}
	\hspace{1.45cm}
	\onslide<1-> $\bs{a}_0=\bs{x}$
	\hspace{.95cm}
	\onslide<2-> $\bs{a}_1=f_1(\bs{a}_0)$
	\hspace{1.35cm}
	\onslide<3-> $\bs{a}_2=f_2(\bs{a}_1)$
	\hspace{.35cm}
	\onslide<4-> $\tilde{\bs{y}}=f_3(\bs{a}_2)$
	\note<1->{
		\begin{itemize}
			\item FNNs are among the most popular neural networks
			\item Here a FNN
			\item Many different architectures
			\item Data set $\rightarrow$ propagating
			\item 
		\end{itemize}
	}
}

\mframe{Cost function}{}{
	\begin{itemize}
		\setlength\itemsep{3em}
		\item<1-> The cost function defines the error
		\item<2-> Mean square error (MSE): $$\mathcal{C}=\frac{1}{2}\sum_{i=1}^n(\bs{y}-\tilde{\bs{y}})^2.$$
		\item<3-> Attempt to minimize the cost function
	\end{itemize}
	\note<1->{
		\begin{itemize}
			\item To decide how good the model performs
			\item Continuous model $\rightarrow$ MSE
			\item Want the error to be small $\rightarrow$ minimize the cost function
		\end{itemize}
	}
}

\mframe{Optimization Algorithms}{}{
	\begin{itemize}
		\setlength\itemsep{3em}
		\item<1-> Minimize the cost function
		\item<2-> The gradient descent method: $$\theta^+=\theta-\frac{\partial\mathcal{C}}{\partial\theta}.$$
	\end{itemize}
	\begin{figure}
		\centering
		\begin{overprint}[7cm]
		\onslide<3>\begin{tikzpicture}[declare function={f(\x)=0.5*\x^2;}]

\newcommand\h{2};			% Particle height
\newcommand\s{1};			% Dashed spacing
\newcommand\X{-2};			% Horizontal position

\begin{axis}[domain=-2:2, samples=50,no markers, hide axis,y=1cm,thick]
\addplot [black] {0.5*x^2};
\node[circle,inner sep=3pt, draw=black, fill=myblue] at (axis cs:-1.7,\h) {};

\draw[->, color=color0] (axis cs:-1.5,1.7) -- (axis cs:-1.3,1.4);

\end{axis}
\end{tikzpicture}
		\onslide<4>\begin{tikzpicture}[declare function={f(\x)=0.5*\x^2;}]

\newcommand\h{2};			% Particle height
\newcommand\s{1};			% Dashed spacing
\newcommand\X{-2};			% Horizontal position

\begin{axis}[domain=-2:2, samples=50,no markers, hide axis,y=1cm,thick]
\addplot [black] {0.5*x^2};
\node[circle,inner sep=3pt, draw=black, fill=myblue] at (axis cs:-1.7,\h) {};

\draw[->, color=color0] (axis cs:-1.5,1.7) -- (axis cs:-1.3,1.4);
\draw[->, color=color0] (axis cs:-1.,1.) -- (axis cs:-.7,.7);

\end{axis}
\end{tikzpicture}
		\onslide<5>\begin{tikzpicture}[declare function={f(\x)=0.5*\x^2;}]

\newcommand\h{2};			% Particle height
\newcommand\s{1};			% Dashed spacing
\newcommand\X{-2};			% Horizontal position

\begin{axis}[domain=-2:2, samples=50,no markers, hide axis,y=1cm,thick]
\addplot [black] {0.5*x^2};
\node[circle,inner sep=3pt, draw=black, fill=myblue] at (axis cs:-1.7,\h) {};

\draw[->, color=color0] (axis cs:-1.5,1.7) -- (axis cs:-1.3,1.4);
\draw[->, color=color0] (axis cs:-1.,1.) -- (axis cs:-.7,.7);
\draw[->, color=color0] (axis cs:-.4,.45) -- (axis cs:-.1,.25);

\draw[color=color1, dashed]  (axis cs:0,0.1) -- (axis cs:0,1.7);
\node[] at (axis cs:0,2.) {Bingo!};

\end{axis}
\end{tikzpicture}
		\end{overprint}
	\end{figure}
	\note<1->{
		\begin{itemize}
			\item For this, we use optimization algorithms
			\item Plenty of methods $\rightarrow$ tradeoff between simplicity and performance
			\item GD perhaps the simplest $\rightarrow$ move in the direction that minimizes the cost function
			\item We have used the ADAM optimizer, which is slightly more complex. Contains momentum
		\end{itemize}
	}
}

%\note{For the minimization, we use the optimization algorithms which basically find the minimum of any function. There are plenty of different algorithms, which have to tradeoff between simplicity and performance. Perhaps the most basic algorithm is gradient descent, which seeks the minimum based on the gradient. The function is then minimized with respect to the steepest slope, until we have found a minimum. }

\mframe{Find Appropriate Complexity}{}{
	\begin{figure}
		\centering
		% This file was created by tikzplotlib v0.8.1.
\begin{tikzpicture}

\begin{axis}[
height=7cm, 
width=9cm,
ticks=none,
ylabel near ticks,
xlabel near ticks,
axis x line=bottom,
axis y line=left,
xlabel={Model complexity},
xmin=-0.2, xmax=4.2,
ylabel={Error},
ymin=-16, ymax=4.4
]
\addplot [thick, color0]
table {%
0 3.46573590279973
0.004004004004004 3.26943984054636
0.00800800800800801 3.08055996975182
0.012012012012012 2.89855625448287
0.016016016016016 2.72294557934374
0.02002002002002 2.55329402090492
0.024024024024024 2.38921038789678
0.028028028028028 2.23034078814828
0.032032032032032 2.07636403245592
0.036036036036036 1.92698772528678
0.04004004004004 1.78194492271509
0.044044044044044 1.64099126160566
0.048048048048048 1.5039024824885
0.0520520520520521 1.37047228306363
0.0560560560560561 1.24051045075361
0.0600600600600601 1.11384123187272
0.0640640640640641 0.990301902323346
0.0680680680680681 0.86974151065479
0.0720720720720721 0.752019769128248
0.0760760760760761 0.637006072354703
0.0800800800800801 0.524578626289829
0.0840840840840841 0.414623673020889
0.0880880880880881 0.307034798975167
0.0920920920920921 0.201712316003994
0.0960960960960961 0.0985627063197901
0.1001001001001 -0.00250187645950529
0.104104104104104 -0.101564056829782
0.108108108108108 -0.198701643247571
0.112112112112112 -0.293987995299001
0.116116116116116 -0.387492356531543
0.12012012012012 -0.479280156731312
0.124124124124124 -0.56941328695128
0.128128128128128 -0.657950350185927
0.132132132132132 -0.744946890235164
0.136136136136136 -0.830455600995881
0.14014014014014 -0.914526518155979
0.144144144144144 -0.997207195037107
0.148148148148148 -1.07854286413346
0.152152152152152 -1.15857658572056
0.156156156156156 -1.23734938475645
0.16016016016016 -1.31490037716496
0.164164164164164 -1.39126688647413
0.168168168168168 -1.46648455168055
0.172172172172172 -1.54058742711986
0.176176176176176 -1.61360807504394
0.18018018018018 -1.68557765153477
0.184184184184184 -1.75652598632236
0.188188188188188 -1.82648165701839
0.192192192192192 -1.89547205822808
0.196196196196196 -1.96352346595855
0.2002002002002 -2.03066109770255
0.204204204204204 -2.09690916854147
0.208208208208208 -2.16229094358001
0.212212212212212 -2.22682878699652
0.216216216216216 -2.29054420796793
0.22022022022022 -2.3534579037051
0.224224224224224 -2.41558979981415
0.228228228228228 -2.47695908818075
0.232232232232232 -2.53758426255738
0.236236236236236 -2.59748315201897
0.24024024024024 -2.65667295243809
0.244244244244244 -2.71517025611884
0.248248248248248 -2.77299107971722
0.252252252252252 -2.83015089056543
0.256256256256256 -2.88666463150847
0.26026026026026 -2.94254674435266
0.264264264264264 -2.9978111920183
0.268268268268268 -3.05247147948138
0.272272272272272 -3.10654067358286
0.276276276276276 -3.1600314217784
0.28028028028028 -3.21295596989563
0.284284284284284 -3.26532617896149
0.288288288288288 -3.31715354115734
0.292292292292292 -3.36844919495566
0.296296296296296 -3.41922393948816
0.3003003003003 -3.4694882481916
0.304304304304304 -3.51925228177457
0.308308308308308 -3.5685259005453
0.312312312312312 -3.61731867613791
0.316316316316316 -3.66563990267206
0.32032032032032 -3.71349860737847
0.324324324324324 -3.76090356072069
0.328328328328328 -3.80786328604159
0.332332332332332 -3.85438606876101
0.336336336336336 -3.90047996514947
0.34034034034034 -3.9461528107011
0.344344344344344 -3.99141222812768
0.348348348348348 -4.03626563499401
0.352352352352352 -4.08072025101391
0.356356356356356 -4.12478310502471
0.36036036036036 -4.16846104165712
0.364364364364364 -4.21176072771623
0.368368368368368 -4.25468865828859
0.372372372372372 -4.29725116258944
0.376376376376376 -4.33945440956306
0.38038038038038 -4.38130441324873
0.384384384384384 -4.4228070379241
0.388388388388388 -4.46396800303669
0.392392392392392 -4.50479288793421
0.396396396396396 -4.54528713640318
0.4004004004004 -4.58545606102536
0.404404404404404 -4.6253048473605
0.408408408408408 -4.66483855796373
0.412412412412412 -4.70406213624542
0.416416416416416 -4.74298041018076
0.42042042042042 -4.78159809587608
0.424424424424424 -4.8199198009986
0.428428428428428 -4.85795002807559
0.432432432432432 -4.89569317766909
0.436436436436436 -4.93315355143176
0.44044044044044 -4.97033535504894
0.444444444444444 -5.00724270107231
0.448448448448448 -5.04387961164958
0.452452452452452 -5.08025002115501
0.456456456456456 -5.11635777872485
0.46046046046046 -5.15220665070197
0.464464464464464 -5.18780032299342
0.468468468468468 -5.22314240334467
0.472472472472472 -5.25823642353409
0.476476476476476 -5.29308584149087
0.48048048048048 -5.32769404333971
0.484484484484485 -5.36206434537519
0.488488488488488 -5.39619999596878
0.492492492492492 -5.43010417741113
0.496496496496497 -5.4637800076924
0.500500500500501 -5.49723054222297
0.504504504504504 -5.53045877549701
0.508508508508508 -5.56346764270112
0.512512512512513 -5.59626002127021
0.516516516516517 -5.62883873239272
0.520520520520521 -5.6612065424671
0.524524524524524 -5.69336616451144
0.528528528528528 -5.72532025952805
0.532532532532533 -5.75707143782478
0.536536536536537 -5.78862226029455
0.540540540540541 -5.81997523965481
0.544544544544544 -5.85113284164842
0.548548548548549 -5.88209748620729
0.552552552552553 -5.9128715485802
0.556556556556557 -5.94345736042618
0.560560560560561 -5.97385721087461
0.564564564564565 -6.0040733475533
0.568568568568569 -6.03410797758569
0.572572572572573 -6.06396326855826
0.576576576576577 -6.09364134945928
0.580580580580581 -6.1231443115898
0.584584584584585 -6.15247420944799
0.588588588588589 -6.18163306158758
0.592592592592593 -6.21062285145156
0.596596596596597 -6.23944552818176
0.600600600600601 -6.2681030074052
0.604604604604605 -6.29659717199816
0.608608608608609 -6.32492987282845
0.612612612612613 -6.35310292947687
0.616616616616617 -6.38111813093843
0.620620620620621 -6.40897723630402
0.624624624624625 -6.43668197542321
0.628628628628629 -6.4642340495488
0.632632632632633 -6.49163513196364
0.636636636636637 -6.5188868685905
0.640640640640641 -6.54599087858526
0.644644644644645 -6.57294875491421
0.648648648648649 -6.59976206491584
0.652652652652653 -6.62643235084764
0.656656656656657 -6.65296113041839
0.660660660660661 -6.67934989730643
0.664664664664665 -6.7056001216643
0.668668668668669 -6.73171325061022
0.672672672672673 -6.75769070870678
0.676676676676677 -6.78353389842725
0.680680680680681 -6.80924420060999
0.684684684684685 -6.83482297490112
0.688688688688689 -6.86027156018595
0.692692692692693 -6.88559127500957
0.696696696696697 -6.91078341798675
0.700700700700701 -6.93584926820161
0.704704704704705 -6.96079008559731
0.708708708708709 -6.98560711135605
0.712712712712713 -7.01030156826974
0.716716716716717 -7.03487466110151
0.720720720720721 -7.05932757693839
0.724724724724725 -7.08366148553546
0.728728728728729 -7.10787753965154
0.732732732732733 -7.13197687537695
0.736736736736737 -7.15596061245329
0.740740740740741 -7.17982985458564
0.744744744744745 -7.20358568974734
0.748748748748749 -7.22722919047752
0.752752752752753 -7.25076141417171
0.756756756756757 -7.27418340336556
0.760760760760761 -7.29749618601199
0.764764764764765 -7.32070077575193
0.768768768768769 -7.34379817217873
0.772772772772773 -7.36678936109658
0.776776776776777 -7.38967531477295
0.780780780780781 -7.41245699218531
0.784784784784785 -7.43513533926225
0.788788788788789 -7.45771128911914
0.792792792792793 -7.48018576228854
0.796796796796797 -7.5025596669454
0.800800800800801 -7.52483389912725
0.804804804804805 -7.54700934294956
0.808808808808809 -7.56908687081629
0.812812812812813 -7.59106734362583
0.816816816816817 -7.61295161097244
0.820820820820821 -7.63474051134329
0.824824824824825 -7.65643487231123
0.828828828828829 -7.6780355107234
0.832832832832833 -7.69954323288578
0.836836836836837 -7.72095883474379
0.840840840840841 -7.74228310205901
0.844844844844845 -7.76351681058223
0.848848848848849 -7.78466072622271
0.852852852852853 -7.80571560521403
0.856856856856857 -7.82668219427632
0.860860860860861 -7.84756123077519
0.864864864864865 -7.86835344287735
0.868868868868869 -7.88905954970295
0.872872872872873 -7.90968026147486
0.876876876876877 -7.93021627966481
0.880880880880881 -7.9506682971366
0.884884884884885 -7.97103699828633
0.888888888888889 -7.99132305917988
0.892892892892893 -8.01152714768748
0.896896896896897 -8.0316499236157
0.900900900900901 -8.05169203883675
0.904904904904905 -8.07165413741514
0.908908908908909 -8.0915368557319
0.912912912912913 -8.11134082260631
0.916916916916917 -8.13106665941523
0.920920920920921 -8.1507149802101
0.924924924924925 -8.17028639183159
0.928928928928929 -8.18978149402212
0.932932932932933 -8.20920087953608
0.936936936936937 -8.22854513424802
0.940940940940941 -8.24781483725867
0.944944944944945 -8.267010560999
0.948948948948949 -8.28613287133221
0.952952952952953 -8.30518232765388
0.956956956956957 -8.32415948299008
0.960960960960961 -8.34306488409378
0.964964964964965 -8.36189907153934
0.968968968968969 -8.38066257981523
0.972972972972973 -8.3993559374151
0.976976976976977 -8.41797966692705
0.980980980980981 -8.4365342851213
0.984984984984985 -8.45502030303626
0.988988988988989 -8.47343822606292
0.992992992992993 -8.49178855402784
0.996996996996997 -8.51007178127447
1.001001001001 -8.52828839674309
1.00500500500501 -8.54643888404926
1.00900900900901 -8.56452372156087
1.01301301301301 -8.58254338247375
1.01701701701702 -8.60049833488601
1.02102102102102 -8.61838904187097
1.02502502502503 -8.63621596154884
1.02902902902903 -8.65397954715709
1.03303303303303 -8.67168024711961
1.03703703703704 -8.68931850511461
1.04104104104104 -8.7068947601414
1.04504504504505 -8.72440944658587
1.04904904904905 -8.74186299428499
1.05305305305305 -8.75925582859004
1.05705705705706 -8.77658837042884
1.06106106106106 -8.79386103636689
1.06506506506507 -8.81107423866739
1.06906906906907 -8.82822838535033
1.07307307307307 -8.84532388025052
1.07707707707708 -8.86236112307459
1.08108108108108 -8.8793405094571
1.08508508508509 -8.89626243101569
1.08908908908909 -8.9131272754052
1.09309309309309 -8.92993542637105
1.0970970970971 -8.94668726380154
1.1011011011011 -8.96338316377945
1.10510510510511 -8.98002349863264
1.10910910910911 -8.9966086369839
1.11311311311311 -9.01313894379994
1.11711711711712 -9.0296147804396
1.12112112112112 -9.04603650470119
1.12512512512513 -9.0624044708692
1.12912912912913 -9.07871902976007
1.13313313313313 -9.09498052876738
1.13713713713714 -9.11118931190618
1.14114114114114 -9.12734571985671
1.14514514514515 -9.14345009000733
1.14914914914915 -9.15950275649678
1.15315315315315 -9.17550405025582
1.15715715715716 -9.19145429904814
1.16116116116116 -9.20735382751066
1.16516516516517 -9.22320295719314
1.16916916916917 -9.23900200659725
1.17317317317317 -9.25475129121493
1.17717717717718 -9.27045112356622
1.18118118118118 -9.28610181323647
1.18518518518519 -9.30170366691292
1.18918918918919 -9.31725698842086
1.19319319319319 -9.33276207875901
1.1971971971972 -9.34821923613458
1.2012012012012 -9.3636287559976
1.20520520520521 -9.37899093107487
1.20920920920921 -9.39430605140327
1.21321321321321 -9.40957440436266
1.21721721721722 -9.42479627470821
1.22122122122122 -9.43997194460228
1.22522522522523 -9.45510169364579
1.22922922922923 -9.47018579890914
1.23323323323323 -9.48522453496266
1.23723723723724 -9.50021817390657
1.24124124124124 -9.51516698540055
1.24524524524525 -9.53007123669283
1.24924924924925 -9.54493119264886
1.25325325325325 -9.55974711577953
1.25725725725726 -9.57451926626905
1.26126126126126 -9.5892479020023
1.26526526526527 -9.60393327859188
1.26926926926927 -9.61857564940473
1.27327327327327 -9.63317526558833
1.27727727727728 -9.64773237609659
1.28128128128128 -9.66224722771528
1.28528528528529 -9.67672006508716
1.28928928928929 -9.6911511307367
1.29329329329329 -9.70554066509451
1.2972972972973 -9.7198889065213
1.3013013013013 -9.73419609133163
1.30530530530531 -9.74846245381726
1.30930930930931 -9.76268822627014
1.31331331331331 -9.77687363900513
1.31731731731732 -9.79101892038236
1.32132132132132 -9.80512429682929
1.32532532532533 -9.81918999286248
1.32932932932933 -9.83321623110898
1.33333333333333 -9.84720323232754
1.33733733733734 -9.86115121542939
1.34134134134134 -9.87506039749886
1.34534534534535 -9.8889309938136
1.34934934934935 -9.90276321786461
1.35335335335335 -9.91655728137594
1.35735735735736 -9.93031339432413
1.36136136136136 -9.9440317649574
1.36536536536537 -9.95771259981457
1.36936936936937 -9.97135610374367
1.37337337337337 -9.98496247992044
1.37737737737738 -9.99853192986638
1.38138138138138 -10.0120646534667
1.38538538538539 -10.0255608489881
1.38938938938939 -10.039020713096
1.39339339339339 -10.0524444408717
1.3973973973974 -10.0658322258298
1.4014014014014 -10.0791842599343
1.40540540540541 -10.0925007336156
1.40940940940941 -10.1057818357865
1.41341341341341 -10.1190277538585
1.41741741741742 -10.1322386737576
1.42142142142142 -10.1454147799398
1.42542542542543 -10.1585562554068
1.42942942942943 -10.1716632817211
1.43343343343343 -10.184736039021
1.43743743743744 -10.1977747060356
1.44144144144144 -10.2107794600996
1.44544544544545 -10.2237504771671
1.44944944944945 -10.2366879318269
1.45345345345345 -10.2495919973156
1.45745745745746 -10.2624628455323
1.46146146146146 -10.2753006470518
1.46546546546547 -10.2881055711386
1.46946946946947 -10.3008777857598
1.47347347347347 -10.3136174575987
1.47747747747748 -10.3263247520679
1.48148148148148 -10.3389998333217
1.48548548548549 -10.3516428642696
1.48948948948949 -10.364254006588
1.49349349349349 -10.3768334207332
1.4974974974975 -10.3893812659534
1.5015015015015 -10.4018977003011
1.50550550550551 -10.4143828806444
1.50950950950951 -10.4268369626796
1.51351351351351 -10.4392601009422
1.51751751751752 -10.4516524488187
1.52152152152152 -10.4640141585582
1.52552552552553 -10.4763453812829
1.52952952952953 -10.4886462670001
1.53353353353353 -10.5009169646122
1.53753753753754 -10.5131576219284
1.54154154154154 -10.5253683856747
1.54554554554555 -10.5375494015051
1.54954954954955 -10.5497008140112
1.55355355355355 -10.5618227667332
1.55755755755756 -10.5739154021699
1.56156156156156 -10.5859788617886
1.56556556556557 -10.598013286035
1.56956956956957 -10.6100188143434
1.57357357357357 -10.6219955851457
1.57757757757758 -10.6339437358819
1.58158158158158 -10.6458634030086
1.58558558558559 -10.6577547220091
1.58958958958959 -10.6696178274021
1.59359359359359 -10.6814528527515
1.5975975975976 -10.6932599306745
1.6016016016016 -10.7050391928514
1.60560560560561 -10.7167907700337
1.60960960960961 -10.7285147920535
1.61361361361361 -10.7402113878314
1.61761761761762 -10.7518806853856
1.62162162162162 -10.76352281184
1.62562562562563 -10.7751378934324
1.62962962962963 -10.7867260555231
1.63363363363363 -10.7982874226027
1.63763763763764 -10.8098221183003
1.64164164164164 -10.8213302653911
1.64564564564565 -10.8328119858048
1.64964964964965 -10.8442674006329
1.65365365365365 -10.8556966301366
1.65765765765766 -10.8670997937542
1.66166166166166 -10.8784770101086
1.66566566566567 -10.8898283970149
1.66966966966967 -10.9011540714876
1.67367367367367 -10.9124541497479
1.67767767767768 -10.9237287472305
1.68168168168168 -10.9349779785913
1.68568568568569 -10.9462019577138
1.68968968968969 -10.9574007977166
1.69369369369369 -10.9685746109596
1.6976976976977 -10.9797235090512
1.7017017017017 -10.9908476028549
1.70570570570571 -11.0019470024959
1.70970970970971 -11.0130218173676
1.71371371371371 -11.024072156138
1.71771771771772 -11.0350981267565
1.72172172172172 -11.0460998364595
1.72572572572573 -11.0570773917775
1.72972972972973 -11.0680308985405
1.73373373373373 -11.0789604618848
1.73773773773774 -11.0898661862585
1.74174174174174 -11.100748175428
1.74574574574575 -11.1116065324833
1.74974974974975 -11.1224413598446
1.75375375375375 -11.1332527592675
1.75775775775776 -11.144040831849
1.76176176176176 -11.1548056780331
1.76576576576577 -11.1655473976165
1.76976976976977 -11.176266089754
1.77377377377377 -11.1869618529642
1.77777777777778 -11.1976347851345
1.78178178178178 -11.2082849835273
1.78578578578579 -11.2189125447844
1.78978978978979 -11.2295175649327
1.79379379379379 -11.2401001393896
1.7977977977978 -11.2506603629677
1.8018018018018 -11.2611983298802
1.80580580580581 -11.2717141337459
1.80980980980981 -11.2822078675941
1.81381381381381 -11.2926796238695
1.81781781781782 -11.3031294944375
1.82182182182182 -11.3135575705883
1.82582582582583 -11.3239639430424
1.82982982982983 -11.3343487019549
1.83383383383383 -11.3447119369203
1.83783783783784 -11.3550537369773
1.84184184184184 -11.365374190613
1.84584584584585 -11.3756733857681
1.84984984984985 -11.3859514098405
1.85385385385385 -11.3962083496906
1.85785785785786 -11.4064442916451
1.86186186186186 -11.4166593215018
1.86586586586587 -11.4268535245336
1.86986986986987 -11.437026985493
1.87387387387387 -11.4471797886159
1.87787787787788 -11.4573120176266
1.88188188188188 -11.467423755741
1.88588588588589 -11.4775150856714
1.88988988988989 -11.4875860896301
1.89389389389389 -11.4976368493339
1.8978978978979 -11.5076674460075
1.9019019019019 -11.5176779603879
1.90590590590591 -11.5276684727282
1.90990990990991 -11.5376390628013
1.91391391391391 -11.5475898099037
1.91791791791792 -11.5575207928596
1.92192192192192 -11.5674320900245
1.92592592592593 -11.5773237792885
1.92992992992993 -11.5871959380808
1.93393393393393 -11.5970486433726
1.93793793793794 -11.6068819716809
1.94194194194194 -11.6166959990723
1.94594594594595 -11.6264908011664
1.94994994994995 -11.6362664531391
1.95395395395395 -11.6460230297262
1.95795795795796 -11.6557606052272
1.96196196196196 -11.6654792535079
1.96596596596597 -11.6751790480044
1.96996996996997 -11.6848600617263
1.97397397397397 -11.6945223672599
1.97797797797798 -11.7041660367715
1.98198198198198 -11.7137911420105
1.98598598598599 -11.723397754313
1.98998998998999 -11.7329859446044
1.99399399399399 -11.7425557834031
1.997997997998 -11.7521073408233
2.002002002002 -11.7616406865781
2.00600600600601 -11.7711558899826
2.01001001001001 -11.780653019957
2.01401401401401 -11.7901321450294
2.01801801801802 -11.799593333339
2.02202202202202 -11.8090366526388
2.02602602602603 -11.8184621702987
2.03003003003003 -11.8278699533085
2.03403403403403 -11.8372600682802
2.03803803803804 -11.8466325814515
2.04204204204204 -11.8559875586882
2.04604604604605 -11.8653250654872
2.05005005005005 -11.8746451669788
2.05405405405405 -11.8839479279302
2.05805805805806 -11.8932334127475
2.06206206206206 -11.9025016854788
2.06606606606607 -11.9117528098164
2.07007007007007 -11.9209868491001
2.07407407407407 -11.9302038663191
2.07807807807808 -11.9394039241151
2.08208208208208 -11.9485870847846
2.08608608608609 -11.9577534102814
2.09009009009009 -11.9669029622192
2.09409409409409 -11.9760358018743
2.0980980980981 -11.9851519901875
2.1021021021021 -11.9942515877671
2.10610610610611 -12.003334654891
2.11011011011011 -12.0124012515092
2.11411411411411 -12.0214514372463
2.11811811811812 -12.0304852714035
2.12212212212212 -12.0395028129613
2.12612612612613 -12.0485041205817
2.13013013013013 -12.0574892526101
2.13413413413413 -12.0664582670784
2.13813813813814 -12.0754112217065
2.14214214214214 -12.0843481739048
2.14614614614615 -12.0932691807765
2.15015015015015 -12.1021742991196
2.15415415415415 -12.1110635854292
2.15815815815816 -12.1199370958996
2.16216216216216 -12.1287948864265
2.16616616616617 -12.137637012609
2.17017017017017 -12.1464635297518
2.17417417417417 -12.1552744928672
2.17817817817818 -12.1640699566771
2.18218218218218 -12.1728499756152
2.18618618618619 -12.1816146038291
2.19019019019019 -12.1903638951819
2.19419419419419 -12.1990979032545
2.1981981981982 -12.2078166813477
2.2022022022022 -12.2165202824836
2.20620620620621 -12.2252087594081
2.21021021021021 -12.2338821645927
2.21421421421421 -12.2425405502362
2.21821821821822 -12.2511839682666
2.22222222222222 -12.2598124703432
2.22622622622623 -12.2684261078583
2.23023023023023 -12.277024931939
2.23423423423423 -12.2856089934492
2.23823823823824 -12.2941783429909
2.24224224224224 -12.302733030907
2.24624624624625 -12.3112731072818
2.25025025025025 -12.3197986219438
2.25425425425425 -12.3283096244669
2.25825825825826 -12.3368061641724
2.26226226226226 -12.3452882901305
2.26626626626627 -12.3537560511619
2.27027027027027 -12.3622094958401
2.27427427427427 -12.3706486724922
2.27827827827828 -12.3790736292014
2.28228228228228 -12.3874844138081
2.28628628628629 -12.3958810739115
2.29029029029029 -12.4042636568716
2.29429429429429 -12.4126322098105
2.2982982982983 -12.4209867796143
2.3023023023023 -12.4293274129342
2.30630630630631 -12.4376541561884
2.31031031031031 -12.4459670555637
2.31431431431431 -12.4542661570167
2.31831831831832 -12.4625515062758
2.32232232232232 -12.4708231488422
2.32632632632633 -12.4790811299918
2.33033033033033 -12.4873254947766
2.33433433433433 -12.4955562880258
2.33833833833834 -12.503773554348
2.34234234234234 -12.5119773381319
2.34634634634635 -12.5201676835483
2.35035035035035 -12.528344634551
2.35435435435435 -12.5365082348787
2.35835835835836 -12.5446585280563
2.36236236236236 -12.552795557396
2.36636636636637 -12.560919365999
2.37037037037037 -12.5690299967567
2.37437437437437 -12.5771274923522
2.37837837837838 -12.5852118952615
2.38238238238238 -12.5932832477548
2.38638638638639 -12.601341591898
2.39039039039039 -12.6093869695541
2.39439439439439 -12.617419422384
2.3983983983984 -12.6254389918486
2.4024024024024 -12.6334457192091
2.40640640640641 -12.6414396455292
2.41041041041041 -12.6494208116757
2.41441441441441 -12.6573892583202
2.41841841841842 -12.6653450259401
2.42242242242242 -12.6732881548198
2.42642642642643 -12.6812186850521
2.43043043043043 -12.6891366565394
2.43443443443443 -12.6970421089946
2.43843843843844 -12.7049350819427
2.44244244244244 -12.7128156147218
2.44644644644645 -12.7206837464844
2.45045045045045 -12.7285395161981
2.45445445445445 -12.7363829626475
2.45845845845846 -12.7442141244348
2.46246246246246 -12.752033039981
2.46646646646647 -12.7598397475273
2.47047047047047 -12.7676342851361
2.47447447447447 -12.7754166906918
2.47847847847848 -12.7831870019025
2.48248248248248 -12.7909452563006
2.48648648648649 -12.7986914912439
2.49049049049049 -12.8064257439172
2.49449449449449 -12.8141480513328
2.4984984984985 -12.8218584503317
2.5025025025025 -12.829556977585
2.50650650650651 -12.8372436695943
2.51051051051051 -12.8449185626934
2.51451451451451 -12.8525816930489
2.51851851851852 -12.8602330966615
2.52252252252252 -12.8678728093667
2.52652652652653 -12.8755008668362
2.53053053053053 -12.8831173045784
2.53453453453453 -12.89072215794
2.53853853853854 -12.8983154621064
2.54254254254254 -12.9058972521031
2.54654654654655 -12.9134675627964
2.55055055055055 -12.9210264288945
2.55455455455455 -12.9285738849485
2.55855855855856 -12.9361099653533
2.56256256256256 -12.9436347043483
2.56656656656657 -12.9511481360187
2.57057057057057 -12.9586502942963
2.57457457457457 -12.9661412129602
2.57857857857858 -12.9736209256382
2.58258258258258 -12.9810894658072
2.58658658658659 -12.9885468667944
2.59059059059059 -12.995993161778
2.59459459459459 -13.0034283837883
2.5985985985986 -13.0108525657084
2.6026026026026 -13.0182657402752
2.60660660660661 -13.0256679400801
2.61061061061061 -13.03305919757
2.61461461461461 -13.0404395450483
2.61861861861862 -13.0478090146753
2.62262262262262 -13.0551676384693
2.62662662662663 -13.0625154483078
2.63063063063063 -13.0698524759276
2.63463463463463 -13.0771787529261
2.63863863863864 -13.0844943107621
2.64264264264264 -13.0917991807564
2.64664664664665 -13.0990933940928
2.65065065065065 -13.1063769818188
2.65465465465465 -13.1136499748463
2.65865865865866 -13.1209124039528
2.66266266266266 -13.1281642997815
2.66666666666667 -13.1354056928427
2.67067067067067 -13.1426366135144
2.67467467467467 -13.1498570920427
2.67867867867868 -13.1570671585432
2.68268268268268 -13.1642668430011
2.68668668668669 -13.1714561752725
2.69069069069069 -13.1786351850848
2.69469469469469 -13.1858039020373
2.6986986986987 -13.1929623556027
2.7027027027027 -13.2001105751268
2.70670670670671 -13.2072485898299
2.71071071071071 -13.2143764288075
2.71471471471471 -13.2214941210305
2.71871871871872 -13.2286016953465
2.72272272272272 -13.2356991804802
2.72672672672673 -13.2427866050342
2.73073073073073 -13.2498639974894
2.73473473473473 -13.2569313862063
2.73873873873874 -13.263988799425
2.74274274274274 -13.2710362652663
2.74674674674675 -13.2780738117323
2.75075075075075 -13.285101466707
2.75475475475475 -13.292119257957
2.75875875875876 -13.2991272131321
2.76276276276276 -13.3061253597659
2.76676676676677 -13.3131137252769
2.77077077077077 -13.3200923369684
2.77477477477477 -13.3270612220298
2.77877877877878 -13.3340204075369
2.78278278278278 -13.3409699204524
2.78678678678679 -13.3479097876271
2.79079079079079 -13.3548400357999
2.79479479479479 -13.3617606915988
2.7987987987988 -13.3686717815411
2.8028028028028 -13.3755733320347
2.80680680680681 -13.3824653693781
2.81081081081081 -13.3893479197611
2.81481481481481 -13.3962210092656
2.81881881881882 -13.4030846638662
2.82282282282282 -13.4099389094305
2.82682682682683 -13.4167837717199
2.83083083083083 -13.4236192763903
2.83483483483483 -13.4304454489922
2.83883883883884 -13.437262314972
2.84284284284284 -13.4440698996718
2.84684684684685 -13.4508682283305
2.85085085085085 -13.4576573260843
2.85485485485485 -13.4644372179669
2.85885885885886 -13.4712079289104
2.86286286286286 -13.4779694837459
2.86686686686687 -13.4847219072037
2.87087087087087 -13.4914652239142
2.87487487487487 -13.498199458408
2.87887887887888 -13.5049246351171
2.88288288288288 -13.5116407783748
2.88688688688689 -13.5183479124166
2.89089089089089 -13.5250460613804
2.89489489489489 -13.5317352493074
2.8988988988989 -13.5384155001424
2.9029029029029 -13.5450868377345
2.90690690690691 -13.5517492858371
2.91091091091091 -13.558402868109
2.91491491491491 -13.5650476081149
2.91891891891892 -13.5716835293252
2.92292292292292 -13.5783106551172
2.92692692692693 -13.5849290087756
2.93093093093093 -13.5915386134924
2.93493493493493 -13.5981394923678
2.93893893893894 -13.6047316684109
2.94294294294294 -13.6113151645396
2.94694694694695 -13.6178900035817
2.95095095095095 -13.6244562082746
2.95495495495495 -13.6310138012668
2.95895895895896 -13.6375628051174
2.96296296296296 -13.6441032422971
2.96696696696697 -13.6506351351885
2.97097097097097 -13.6571585060868
2.97497497497497 -13.6636733771997
2.97897897897898 -13.6701797706485
2.98298298298298 -13.6766777084681
2.98698698698699 -13.6831672126076
2.99099099099099 -13.6896483049309
2.99499499499499 -13.6961210072167
2.998998998999 -13.7025853411596
3.003003003003 -13.7090413283698
3.00700700700701 -13.7154889903742
3.01101101101101 -13.7219283486164
3.01501501501502 -13.7283594244572
3.01901901901902 -13.7347822391751
3.02302302302302 -13.7411968139669
3.02702702702703 -13.7476031699475
3.03103103103103 -13.7540013281513
3.03503503503504 -13.7603913095316
3.03903903903904 -13.7667731349616
3.04304304304304 -13.7731468252348
3.04704704704705 -13.7795124010651
3.05105105105105 -13.7858698830876
3.05505505505506 -13.7922192918586
3.05905905905906 -13.7985606478562
3.06306306306306 -13.8048939714808
3.06706706706707 -13.8112192830552
3.07107107107107 -13.8175366028255
3.07507507507508 -13.8238459509607
3.07907907907908 -13.8301473475539
3.08308308308308 -13.8364408126222
3.08708708708709 -13.8427263661072
3.09109109109109 -13.8490040278754
3.0950950950951 -13.8552738177185
3.0990990990991 -13.861535755354
3.1031031031031 -13.8677898604254
3.10710710710711 -13.8740361525023
3.11111111111111 -13.8802746510813
3.11511511511512 -13.8865053755862
3.11911911911912 -13.892728345368
3.12312312312312 -13.8989435797057
3.12712712712713 -13.9051510978065
3.13113113113113 -13.911350918806
3.13513513513514 -13.917543061769
3.13913913913914 -13.9237275456893
3.14314314314314 -13.9299043894904
3.14714714714715 -13.9360736120257
3.15115115115115 -13.942235232079
3.15515515515516 -13.9483892683647
3.15915915915916 -13.9545357395281
3.16316316316316 -13.960674664146
3.16716716716717 -13.9668060607266
3.17117117117117 -13.9729299477103
3.17517517517518 -13.9790463434698
3.17917917917918 -13.9851552663104
3.18318318318318 -13.9912567344705
3.18718718718719 -13.9973507661216
3.19119119119119 -14.003437379369
3.1951951951952 -14.0095165922521
3.1991991991992 -14.0155884227441
3.2032032032032 -14.0216528887533
3.20720720720721 -14.0277100081228
3.21121121121121 -14.0337597986307
3.21521521521522 -14.0398022779908
3.21921921921922 -14.0458374638528
3.22322322322322 -14.0518653738026
3.22722722722723 -14.0578860253624
3.23123123123123 -14.0638994359914
3.23523523523524 -14.0699056230855
3.23923923923924 -14.0759046039785
3.24324324324324 -14.0818963959416
3.24724724724725 -14.0878810161839
3.25125125125125 -14.0938584818529
3.25525525525526 -14.0998288100347
3.25925925925926 -14.1057920177543
3.26326326326326 -14.1117481219757
3.26726726726727 -14.1176971396025
3.27127127127127 -14.1236390874779
3.27527527527528 -14.1295739823852
3.27927927927928 -14.135501841048
3.28328328328328 -14.1414226801306
3.28728728728729 -14.147336516238
3.29129129129129 -14.1532433659164
3.2952952952953 -14.1591432456534
3.2992992992993 -14.1650361718785
3.3033033033033 -14.1709221609628
3.30730730730731 -14.17680122922
3.31131131131131 -14.1826733929062
3.31531531531532 -14.1885386682203
3.31931931931932 -14.1943970713042
3.32332332332332 -14.2002486182431
3.32732732732733 -14.2060933250661
3.33133133133133 -14.2119312077457
3.33533533533534 -14.2177622821988
3.33933933933934 -14.2235865642866
3.34334334334334 -14.2294040698151
3.34734734734735 -14.2352148145348
3.35135135135135 -14.2410188141418
3.35535535535536 -14.2468160842774
3.35935935935936 -14.2526066405284
3.36336336336336 -14.2583904984277
3.36736736736737 -14.2641676734545
3.37137137137137 -14.269938181034
3.37537537537538 -14.2757020365384
3.37937937937938 -14.2814592552867
3.38338338338338 -14.287209852545
3.38738738738739 -14.2929538435268
3.39139139139139 -14.2986912433933
3.3953953953954 -14.3044220672534
3.3993993993994 -14.3101463301644
3.4034034034034 -14.3158640471315
3.40740740740741 -14.321575233109
3.41141141141141 -14.3272799029996
3.41541541541542 -14.3329780716553
3.41941941941942 -14.3386697538773
3.42342342342342 -14.3443549644162
3.42742742742743 -14.3500337179725
3.43143143143143 -14.3557060291965
3.43543543543544 -14.361371912689
3.43943943943944 -14.367031383001
3.44344344344344 -14.372684454634
3.44744744744745 -14.3783311420407
3.45145145145145 -14.3839714596246
3.45545545545546 -14.3896054217407
3.45945945945946 -14.3952330426956
3.46346346346346 -14.4008543367473
3.46746746746747 -14.4064693181061
3.47147147147147 -14.4120780009343
3.47547547547548 -14.4176803993467
3.47947947947948 -14.4232765274106
3.48348348348348 -14.4288663991463
3.48748748748749 -14.434450028527
3.49149149149149 -14.440027429479
3.4954954954955 -14.4455986158823
3.4994994994995 -14.4511636015704
3.5035035035035 -14.4567224003308
3.50750750750751 -14.4622750259049
3.51151151151151 -14.4678214919885
3.51551551551552 -14.4733618122319
3.51951951951952 -14.4788960002398
3.52352352352352 -14.4844240695721
3.52752752752753 -14.4899460337436
3.53153153153153 -14.4954619062244
3.53553553553554 -14.5009717004402
3.53953953953954 -14.5064754297722
3.54354354354354 -14.5119731075574
3.54754754754755 -14.5174647470891
3.55155155155155 -14.5229503616167
3.55555555555556 -14.528429964346
3.55955955955956 -14.5339035684396
3.56356356356356 -14.5393711870166
3.56756756756757 -14.5448328331535
3.57157157157157 -14.5502885198837
3.57557557557558 -14.5557382601982
3.57957957957958 -14.5611820670453
3.58358358358358 -14.5666199533313
3.58758758758759 -14.5720519319204
3.59159159159159 -14.5774780156348
3.5955955955956 -14.582898217255
3.5995995995996 -14.5883125495202
3.6036036036036 -14.593721025128
3.60760760760761 -14.599123656735
3.61161161161161 -14.6045204569567
3.61561561561562 -14.6099114383679
3.61961961961962 -14.6152966135028
3.62362362362362 -14.6206759948549
3.62762762762763 -14.6260495948778
3.63163163163163 -14.6314174259846
3.63563563563564 -14.6367795005488
3.63963963963964 -14.6421358309038
3.64364364364364 -14.6474864293437
3.64764764764765 -14.6528313081231
3.65165165165165 -14.6581704794572
3.65565565565566 -14.6635039555223
3.65965965965966 -14.6688317484556
3.66366366366366 -14.6741538703557
3.66766766766767 -14.6794703332825
3.67167167167167 -14.6847811492575
3.67567567567568 -14.690086330264
3.67967967967968 -14.695385888247
3.68368368368368 -14.7006798351139
3.68768768768769 -14.705968182734
3.69169169169169 -14.7112509429391
3.6956956956957 -14.7165281275236
3.6996996996997 -14.7217997482444
3.7037037037037 -14.7270658168214
3.70770770770771 -14.7323263449376
3.71171171171171 -14.7375813442389
3.71571571571572 -14.7428308263348
3.71971971971972 -14.7480748027981
3.72372372372372 -14.7533132851652
3.72772772772773 -14.7585462849365
3.73173173173173 -14.7637738135759
3.73573573573574 -14.7689958825119
3.73973973973974 -14.7742125031369
3.74374374374374 -14.7794236868078
3.74774774774775 -14.7846294448459
3.75175175175175 -14.7898297885374
3.75575575575576 -14.795024729133
3.75975975975976 -14.8002142778487
3.76376376376376 -14.8053984458654
3.76776776776777 -14.8105772443294
3.77177177177177 -14.8157506843522
3.77577577577578 -14.8209187770111
3.77977977977978 -14.8260815333489
3.78378378378378 -14.8312389643743
3.78778778778779 -14.8363910810619
3.79179179179179 -14.8415378943525
3.7957957957958 -14.8466794151531
3.7997997997998 -14.8518156543372
3.8038038038038 -14.8569466227445
3.80780780780781 -14.8620723311818
3.81181181181181 -14.8671927904225
3.81581581581582 -14.8723080112067
3.81981981981982 -14.877418004242
3.82382382382382 -14.8825227802029
3.82782782782783 -14.8876223497314
3.83183183183183 -14.8927167234368
3.83583583583584 -14.8978059118962
3.83983983983984 -14.9028899256544
3.84384384384384 -14.9079687752239
3.84784784784785 -14.9130424710853
3.85185185185185 -14.9181110236874
3.85585585585586 -14.9231744434472
3.85985985985986 -14.92823274075
3.86386386386386 -14.9332859259498
3.86786786786787 -14.9383340093691
3.87187187187187 -14.9433770012992
3.87587587587588 -14.9484149120002
3.87987987987988 -14.9534477517014
3.88388388388388 -14.9584755306012
3.88788788788789 -14.963498258867
3.89189189189189 -14.968515946636
3.8958958958959 -14.9735286040147
3.8998998998999 -14.9785362410792
3.9039039039039 -14.9835388678753
3.90790790790791 -14.988536494419
3.91191191191191 -14.9935291306959
3.91591591591592 -14.998516786662
3.91991991991992 -15.0034994722434
3.92392392392392 -15.0084771973366
3.92792792792793 -15.0134499718085
3.93193193193193 -15.0184178054967
3.93593593593594 -15.0233807082094
3.93993993993994 -15.0283386897258
3.94394394394394 -15.0332917597957
3.94794794794795 -15.0382399281404
3.95195195195195 -15.043183204452
3.95595595595596 -15.048121598394
3.95995995995996 -15.0530551196014
3.96396396396396 -15.0579837776806
3.96796796796797 -15.0629075822096
3.97197197197197 -15.0678265427381
3.97597597597598 -15.0727406687879
3.97997997997998 -15.0776499698524
3.98398398398398 -15.0825544553973
3.98798798798799 -15.0874541348604
3.99199199199199 -15.0923490176518
3.995995995996 -15.0972391131539
4 -15.1021244307218
};
\addplot [thick, color1]
table {%
0 3.5
0.004004004004004 3.46399602806009
0.00800800800800801 3.42805618431244
0.012012012012012 3.39218046875705
0.016016016016016 3.35636888139391
0.02002002002002 3.32062142222302
0.024024024024024 3.2849380912444
0.028028028028028 3.24931888845803
0.032032032032032 3.21376381386391
0.036036036036036 3.17827286746206
0.04004004004004 3.14284604925246
0.044044044044044 3.10748335923511
0.048048048048048 3.07218479741002
0.0520520520520521 3.03695036377719
0.0560560560560561 3.00178005833661
0.0600600600600601 2.9666738810883
0.0640640640640641 2.93163183203223
0.0680680680680681 2.89665391116843
0.0720720720720721 2.86174011849688
0.0760760760760761 2.82689045401758
0.0800800800800801 2.79210491773054
0.0840840840840841 2.75738350963576
0.0880880880880881 2.72272622973324
0.0920920920920921 2.68813307802297
0.0960960960960961 2.65360405450496
0.1001001001001 2.6191391591792
0.104104104104104 2.5847383920457
0.108108108108108 2.55040175310446
0.112112112112112 2.51612924235547
0.116116116116116 2.48192085979874
0.12012012012012 2.44777660543426
0.124124124124124 2.41369647926204
0.128128128128128 2.37968048128208
0.132132132132132 2.34572861149438
0.136136136136136 2.31184086989893
0.14014014014014 2.27801725649574
0.144144144144144 2.2442577712848
0.148148148148148 2.21056241426612
0.152152152152152 2.17693118543969
0.156156156156156 2.14336408480553
0.16016016016016 2.10986111236362
0.164164164164164 2.07642226811396
0.168168168168168 2.04304755205656
0.172172172172172 2.00973696419142
0.176176176176176 1.97649050451853
0.18018018018018 1.9433081730379
0.184184184184184 1.91018996974953
0.188188188188188 1.87713589465341
0.192192192192192 1.84414594774955
0.196196196196196 1.81122012903795
0.2002002002002 1.7783584385186
0.204204204204204 1.74556087619151
0.208208208208208 1.71282744205667
0.212212212212212 1.68015813611409
0.216216216216216 1.64755295836377
0.22022022022022 1.6150119088057
0.224224224224224 1.58253498743989
0.228228228228228 1.55012219426634
0.232232232232232 1.51777352928504
0.236236236236236 1.485488992496
0.24024024024024 1.45326858389922
0.244244244244244 1.42111230349469
0.248248248248248 1.38902015128241
0.252252252252252 1.3569921272624
0.256256256256256 1.32502823143464
0.26026026026026 1.29312846379914
0.264264264264264 1.26129282435589
0.268268268268268 1.2295213131049
0.272272272272272 1.19781393004616
0.276276276276276 1.16617067517968
0.28028028028028 1.13459154850546
0.284284284284284 1.1030765500235
0.288288288288288 1.07162567973379
0.292292292292292 1.04023893763633
0.296296296296296 1.00891632373114
0.3003003003003 0.977657838018199
0.304304304304304 0.946463480497514
0.308308308308308 0.915333251169088
0.312312312312312 0.884267150032915
0.316316316316316 0.853265177089002
0.32032032032032 0.822327332337343
0.324324324324324 0.791453615777939
0.328328328328328 0.760644027410795
0.332332332332332 0.729898567235904
0.336336336336336 0.699217235253272
0.34034034034034 0.668600031462895
0.344344344344344 0.638046955864773
0.348348348348348 0.60755800845891
0.352352352352352 0.577133189245301
0.356356356356356 0.54677249822395
0.36036036036036 0.516475935394856
0.364364364364364 0.486243500758015
0.368368368368368 0.456075194313432
0.372372372372372 0.425971016061106
0.376376376376376 0.395930966001036
0.38038038038038 0.365955044133223
0.384384384384384 0.336043250457664
0.388388388388388 0.306195584974363
0.392392392392392 0.276412047683318
0.396396396396396 0.24669263858453
0.4004004004004 0.217037357678
0.404404404404404 0.187446204963722
0.408408408408408 0.157919180441702
0.412412412412412 0.128456284111939
0.416416416416416 0.0990575159744331
0.42042042042042 0.0697228760291821
0.424424424424424 0.040452364276188
0.428428428428428 0.0112459807154499
0.432432432432432 -0.0178962746530313
0.436436436436436 -0.0469744018292566
0.44044044044044 -0.0759884008132259
0.444444444444444 -0.104938271604938
0.448448448448448 -0.133824014204395
0.452452452452452 -0.162645628611594
0.456456456456456 -0.191403114826538
0.46046046046046 -0.220096472849225
0.464464464464464 -0.248725702679657
0.468468468468468 -0.277290804317832
0.472472472472472 -0.30579177776375
0.476476476476476 -0.334228623017411
0.48048048048048 -0.362601340078817
0.484484484484485 -0.390909928947967
0.488488488488488 -0.41915438962486
0.492492492492492 -0.447334722109497
0.496496496496497 -0.475450926401877
0.500500500500501 -0.503503002502002
0.504504504504504 -0.531490950409869
0.508508508508508 -0.559414770125481
0.512512512512513 -0.587274461648836
0.516516516516517 -0.615070024979934
0.520520520520521 -0.642801460118777
0.524524524524524 -0.670468767065364
0.528528528528528 -0.698071945819694
0.532532532532533 -0.725610996381767
0.536536536536537 -0.753085918751584
0.540540540540541 -0.780496712929145
0.544544544544544 -0.80784337891445
0.548548548548549 -0.835125916707498
0.552552552552553 -0.86234432630829
0.556556556556557 -0.889498607716825
0.560560560560561 -0.916588760933105
0.564564564564565 -0.94361478595713
0.568568568568569 -0.970576682788896
0.572572572572573 -0.997474451428405
0.576576576576577 -1.02430809187566
0.580580580580581 -1.05107760413066
0.584584584584585 -1.0777829881934
0.588588588588589 -1.10442424406388
0.592592592592593 -1.13100137174211
0.596596596596597 -1.15751437122808
0.600600600600601 -1.1839632425218
0.604604604604605 -1.21034798562326
0.608608608608609 -1.23666860053246
0.612612612612613 -1.26292508724941
0.616616616616617 -1.2891174457741
0.620620620620621 -1.31524567610654
0.624624624624625 -1.34130977824672
0.628628628628629 -1.36730975219464
0.632632632632633 -1.3932455979503
0.636636636636637 -1.41911731551371
0.640640640640641 -1.44492490488487
0.644644644644645 -1.47066836606376
0.648648648648649 -1.4963476990504
0.652652652652653 -1.52196290384479
0.656656656656657 -1.54751398044691
0.660660660660661 -1.57300092885678
0.664664664664665 -1.5984237490744
0.668668668668669 -1.62378244109976
0.672672672672673 -1.64907700493286
0.676676676676677 -1.67430744057371
0.680680680680681 -1.6994737480223
0.684684684684685 -1.72457592727863
0.688688688688689 -1.74961397834271
0.692692692692693 -1.77458790121453
0.696696696696697 -1.79949769589409
0.700700700700701 -1.8243433623814
0.704704704704705 -1.84912490067645
0.708708708708709 -1.87384231077925
0.712712712712713 -1.89849559268979
0.716716716716717 -1.92308474640807
0.720720720720721 -1.9476097719341
0.724724724724725 -1.97207066926787
0.728728728728729 -1.99646743840938
0.732732732732733 -2.02080007935864
0.736736736736737 -2.04506859211564
0.740740740740741 -2.06927297668038
0.744744744744745 -2.09341323305287
0.748748748748749 -2.11748936123311
0.752752752752753 -2.14150136122108
0.756756756756757 -2.1654492330168
0.760760760760761 -2.18933297662026
0.764764764764765 -2.21315259203147
0.768768768768769 -2.23690807925042
0.772772772772773 -2.26059943827712
0.776776776776777 -2.28422666911155
0.780780780780781 -2.30778977175374
0.784784784784785 -2.33128874620366
0.788788788788789 -2.35472359246133
0.792792792792793 -2.37809431052674
0.796796796796797 -2.4014009003999
0.800800800800801 -2.4246433620808
0.804804804804805 -2.44782169556944
0.808808808808809 -2.47093590086583
0.812812812812813 -2.49398597796996
0.816816816816817 -2.51697192688184
0.820820820820821 -2.53989374760146
0.824824824824825 -2.56275144012882
0.828828828828829 -2.58554500446392
0.832832832832833 -2.60827444060677
0.836836836836837 -2.63093974855737
0.840840840840841 -2.6535409283157
0.844844844844845 -2.67607797988178
0.848848848848849 -2.69855090325561
0.852852852852853 -2.72095969843718
0.856856856856857 -2.74330436542649
0.860860860860861 -2.76558490422354
0.864864864864865 -2.78780131482834
0.868868868868869 -2.80995359724088
0.872872872872873 -2.83204175146117
0.876876876876877 -2.8540657774892
0.880880880880881 -2.87602567532498
0.884884884884885 -2.89792144496849
0.888888888888889 -2.91975308641975
0.892892892892893 -2.94152059967876
0.896896896896897 -2.96322398474551
0.900900900900901 -2.98486324162
0.904904904904905 -3.00643837030223
0.908908908908909 -3.02794937079221
0.912912912912913 -3.04939624308994
0.916916916916917 -3.0707789871954
0.920920920920921 -3.09209760310861
0.924924924924925 -3.11335209082957
0.928928928928929 -3.13454245035827
0.932932932932933 -3.15566868169471
0.936936936936937 -3.17673078483889
0.940940940940941 -3.19772875979082
0.944944944944945 -3.21866260655049
0.948948948948949 -3.23953232511791
0.952952952952953 -3.26033791549307
0.956956956956957 -3.28107937767597
0.960960960960961 -3.30175671166662
0.964964964964965 -3.32236991746501
0.968968968968969 -3.34291899507115
0.972972972972973 -3.36340394448502
0.976976976976977 -3.38382476570665
0.980980980980981 -3.40418145873601
0.984984984984985 -3.42447402357312
0.988988988988989 -3.44470246021798
0.992992992992993 -3.46486676867057
0.996996996996997 -3.48496694893091
1.001001001001 -3.505003000999
1.00500500500501 -3.52497492487482
1.00900900900901 -3.5448827205584
1.01301301301301 -3.56472638804971
1.01701701701702 -3.58450592734877
1.02102102102102 -3.60422133845557
1.02502502502503 -3.62387262137012
1.02902902902903 -3.64345977609241
1.03303303303303 -3.66298280262244
1.03703703703704 -3.68244170096022
1.04104104104104 -3.70183647110574
1.04504504504505 -3.72116711305901
1.04904904904905 -3.74043362682001
1.05305305305305 -3.75963601238877
1.05705705705706 -3.77877426976526
1.06106106106106 -3.7978483989495
1.06506506506507 -3.81685839994148
1.06906906906907 -3.83580427274121
1.07307307307307 -3.85468601734868
1.07707707707708 -3.87350363376389
1.08108108108108 -3.89225712198685
1.08508508508509 -3.91094648201755
1.08908908908909 -3.929571713856
1.09309309309309 -3.94813281750219
1.0970970970971 -3.96662979295612
1.1011011011011 -3.9850626402178
1.10510510510511 -4.00343135928722
1.10910910910911 -4.02173595016438
1.11311311311311 -4.03997641284929
1.11711711711712 -4.05815274734194
1.12112112112112 -4.07626495364233
1.12512512512513 -4.09431303175047
1.12912912912913 -4.11229698166635
1.13313313313313 -4.13021680338998
1.13713713713714 -4.14807249692135
1.14114114114114 -4.16586406226046
1.14514514514515 -4.18359149940732
1.14914914914915 -4.20125480836192
1.15315315315315 -4.21885398912426
1.15715715715716 -4.23638904169435
1.16116116116116 -4.25385996607218
1.16516516516517 -4.27126676225775
1.16916916916917 -4.28860943025107
1.17317317317317 -4.30588797005213
1.17717717717718 -4.32310238166094
1.18118118118118 -4.34025266507749
1.18518518518519 -4.35733882030178
1.18918918918919 -4.37436084733382
1.19319319319319 -4.3913187461736
1.1971971971972 -4.40821251682113
1.2012012012012 -4.42504215927639
1.20520520520521 -4.4418076735394
1.20920920920921 -4.45850905961016
1.21321321321321 -4.47514631748866
1.21721721721722 -4.4917194471749
1.22122122122122 -4.50822844866889
1.22522522522523 -4.52467332197062
1.22922922922923 -4.54105406708009
1.23323323323323 -4.55737068399731
1.23723723723724 -4.57362317272227
1.24124124124124 -4.58981153325498
1.24524524524525 -4.60593576559542
1.24924924924925 -4.62199586974362
1.25325325325325 -4.63799184569955
1.25725725725726 -4.65392369346323
1.26126126126126 -4.66979141303466
1.26526526526527 -4.68559500441382
1.26926926926927 -4.70133446760073
1.27327327327327 -4.71700980259539
1.27727727727728 -4.73262100939779
1.28128128128128 -4.74816808800793
1.28528528528529 -4.76365103842581
1.28928928928929 -4.77906986065144
1.29329329329329 -4.79442455468481
1.2972972972973 -4.80971512052593
1.3013013013013 -4.82494155817479
1.30530530530531 -4.8401038676314
1.30930930930931 -4.85520204889574
1.31331331331331 -4.87023610196783
1.31731731731732 -4.88520602684767
1.32132132132132 -4.90011182353525
1.32532532532533 -4.91495349203057
1.32932932932933 -4.92973103233363
1.33333333333333 -4.94444444444444
1.33733733733734 -4.959093728363
1.34134134134134 -4.97367888408929
1.34534534534535 -4.98819991162333
1.34934934934935 -5.00265681096512
1.35335335335335 -5.01704958211465
1.35735735735736 -5.03137822507192
1.36136136136136 -5.04564273983693
1.36536536536537 -5.05984312640969
1.36936936936937 -5.0739793847902
1.37337337337337 -5.08805151497844
1.37737737737738 -5.10205951697443
1.38138138138138 -5.11600339077817
1.38538538538539 -5.12988313638964
1.38938938938939 -5.14369875380886
1.39339339339339 -5.15745024303583
1.3973973973974 -5.17113760407054
1.4014014014014 -5.18476083691299
1.40540540540541 -5.19831994156319
1.40940940940941 -5.21181491802112
1.41341341341341 -5.22524576628681
1.41741741741742 -5.23861248636023
1.42142142142142 -5.2519150782414
1.42542542542543 -5.26515354193032
1.42942942942943 -5.27832787742698
1.43343343343343 -5.29143808473138
1.43743743743744 -5.30448416384352
1.44144144144144 -5.31746611476341
1.44544544544545 -5.33038393749105
1.44944944944945 -5.34323763202642
1.45345345345345 -5.35602719836954
1.45745745745746 -5.3687526365204
1.46146146146146 -5.38141394647901
1.46546546546547 -5.39401112824536
1.46946946946947 -5.40654418181946
1.47347347347347 -5.4190131072013
1.47747747747748 -5.43141790439088
1.48148148148148 -5.4437585733882
1.48548548548549 -5.45603511419327
1.48948948948949 -5.46824752680608
1.49349349349349 -5.48039581122664
1.4974974974975 -5.49247996745494
1.5015015015015 -5.50449999549099
1.50550550550551 -5.51645589533477
1.50950950950951 -5.52834766698631
1.51351351351351 -5.54017531044558
1.51751751751752 -5.5519388257126
1.52152152152152 -5.56363821278736
1.52552552552553 -5.57527347166987
1.52952952952953 -5.58684460236012
1.53353353353353 -5.59835160485811
1.53753753753754 -5.60979447916385
1.54154154154154 -5.62117322527733
1.54554554554555 -5.63248784319855
1.54954954954955 -5.64373833292752
1.55355355355355 -5.65492469446423
1.55755755755756 -5.66604692780869
1.56156156156156 -5.67710503296089
1.56556556556557 -5.68809900992083
1.56956956956957 -5.69902885868852
1.57357357357357 -5.70989457926395
1.57757757757758 -5.72069617164712
1.58158158158158 -5.73143363583804
1.58558558558559 -5.7421069718367
1.58958958958959 -5.75271617964311
1.59359359359359 -5.76326125925726
1.5975975975976 -5.77374221067915
1.6016016016016 -5.78415903390878
1.60560560560561 -5.79451172894616
1.60960960960961 -5.80480029579129
1.61361361361361 -5.81502473444415
1.61761761761762 -5.82518504490476
1.62162162162162 -5.83528122717312
1.62562562562563 -5.84531328124922
1.62962962962963 -5.85528120713306
1.63363363363363 -5.86518500482464
1.63763763763764 -5.87502467432397
1.64164164164164 -5.88480021563105
1.64564564564565 -5.89451162874586
1.64964964964965 -5.90415891366842
1.65365365365365 -5.91374207039873
1.65765765765766 -5.92326109893677
1.66166166166166 -5.93271599928257
1.66566566566567 -5.9421067714361
1.66966966966967 -5.95143341539738
1.67367367367367 -5.9606959311664
1.67767767767768 -5.96989431874317
1.68168168168168 -5.97902857812768
1.68568568568569 -5.98809870931993
1.68968968968969 -5.99710471231993
1.69369369369369 -6.00604658712767
1.6976976976977 -6.01492433374315
1.7017017017017 -6.02373795216638
1.70570570570571 -6.03248744239735
1.70970970970971 -6.04117280443607
1.71371371371371 -6.04979403828253
1.71771771771772 -6.05835114393673
1.72172172172172 -6.06684412139868
1.72572572572573 -6.07527297066837
1.72972972972973 -6.0836376917458
1.73373373373373 -6.09193828463098
1.73773773773774 -6.1001747493239
1.74174174174174 -6.10834708582456
1.74574574574575 -6.11645529413297
1.74974974974975 -6.12449937424912
1.75375375375375 -6.13247932617302
1.75775775775776 -6.14039514990466
1.76176176176176 -6.14824684544404
1.76576576576577 -6.15603441279117
1.76976976976977 -6.16375785194604
1.77377377377377 -6.17141716290865
1.77777777777778 -6.17901234567901
1.78178178178178 -6.18654340025711
1.78578578578579 -6.19401032664296
1.78978978978979 -6.20141312483655
1.79379379379379 -6.20875179483788
1.7977977977978 -6.21602633664696
1.8018018018018 -6.22323675026378
1.80580580580581 -6.23038303568834
1.80980980980981 -6.23746519292065
1.81381381381381 -6.2444832219607
1.81781781781782 -6.25143712280849
1.82182182182182 -6.25832689546403
1.82582582582583 -6.26515253992731
1.82982982982983 -6.27191405619834
1.83383383383383 -6.27861144427711
1.83783783783784 -6.28524470416362
1.84184184184184 -6.29181383585788
1.84584584584585 -6.29831883935988
1.84984984984985 -6.30475971466962
1.85385385385385 -6.31113646178711
1.85785785785786 -6.31744908071234
1.86186186186186 -6.32369757144532
1.86586586586587 -6.32988193398604
1.86986986986987 -6.3360021683345
1.87387387387387 -6.34205827449071
1.87787787787788 -6.34805025245466
1.88188188188188 -6.35397810222635
1.88588588588589 -6.35984182380579
1.88988988988989 -6.36564141719297
1.89389389389389 -6.37137688238789
1.8978978978979 -6.37704821939056
1.9019019019019 -6.38265542820097
1.90590590590591 -6.38819850881913
1.90990990990991 -6.39367746124503
1.91391391391391 -6.39909228547867
1.91791791791792 -6.40444298152006
1.92192192192192 -6.40972954936919
1.92592592592593 -6.41495198902606
1.92992992992993 -6.42011030049068
1.93393393393393 -6.42520448376304
1.93793793793794 -6.43023453884315
1.94194194194194 -6.435200465731
1.94594594594595 -6.44010226442659
1.94994994994995 -6.44493993492993
1.95395395395395 -6.449713477241
1.95795795795796 -6.45442289135983
1.96196196196196 -6.4590681772864
1.96596596596597 -6.46364933502071
1.96996996996997 -6.46816636456276
1.97397397397397 -6.47261926591256
1.97797797797798 -6.4770080390701
1.98198198198198 -6.48133268403539
1.98598598598599 -6.48559320080842
1.98998998998999 -6.48978958938919
1.99399399399399 -6.49392184977771
1.997997997998 -6.49798998197397
2.002002002002 -6.50199398597797
2.00600600600601 -6.50593386178972
2.01001001001001 -6.50980960940921
2.01401401401401 -6.51362122883644
2.01801801801802 -6.51736872007142
2.02202202202202 -6.52105208311415
2.02602602602603 -6.52467131796461
2.03003003003003 -6.52822642462282
2.03403403403403 -6.53171740308877
2.03803803803804 -6.53514425336247
2.04204204204204 -6.53850697544391
2.04604604604605 -6.5418055693331
2.05005005005005 -6.54504003503003
2.05405405405405 -6.5482103725347
2.05805805805806 -6.55131658184711
2.06206206206206 -6.55435866296727
2.06606606606607 -6.55733661589517
2.07007007007007 -6.56025044063082
2.07407407407407 -6.56310013717421
2.07807807807808 -6.56588570552534
2.08208208208208 -6.56860714568422
2.08608608608609 -6.57126445765084
2.09009009009009 -6.57385764142521
2.09409409409409 -6.57638669700732
2.0980980980981 -6.57885162439717
2.1021021021021 -6.58125242359477
2.10610610610611 -6.58358909460011
2.11011011011011 -6.58586163741319
2.11411411411411 -6.58807005203402
2.11811811811812 -6.59021433846259
2.12212212212212 -6.5922944966989
2.12612612612613 -6.59431052674296
2.13013013013013 -6.59626242859476
2.13413413413413 -6.59815020225431
2.13813813813814 -6.5999738477216
2.14214214214214 -6.60173336499663
2.14614614614615 -6.6034287540794
2.15015015015015 -6.60506001496992
2.15415415415415 -6.60662714766819
2.15815815815816 -6.6081301521742
2.16216216216216 -6.60956902848795
2.16616616616617 -6.61094377660944
2.17017017017017 -6.61225439653868
2.17417417417417 -6.61350088827566
2.17817817817818 -6.61468325182039
2.18218218218218 -6.61580148717286
2.18618618618619 -6.61685559433307
2.19019019019019 -6.61784557330103
2.19419419419419 -6.61877142407673
2.1981981981982 -6.61963314666017
2.2022022022022 -6.62043074105136
2.20620620620621 -6.62116420725029
2.21021021021021 -6.62183354525697
2.21421421421421 -6.62243875507139
2.21821821821822 -6.62297983669355
2.22222222222222 -6.62345679012346
2.22622622622623 -6.62386961536111
2.23023023023023 -6.6242183124065
2.23423423423423 -6.62450288125964
2.23823823823824 -6.62472332192052
2.24224224224224 -6.62487963438914
2.24624624624625 -6.62497181866551
2.25025025025025 -6.62499987474962
2.25425425425425 -6.62496380264148
2.25825825825826 -6.62486360234108
2.26226226226226 -6.62469927384842
2.26626626626627 -6.62447081716351
2.27027027027027 -6.62417823228634
2.27427427427427 -6.62382151921691
2.27827827827828 -6.62340067795523
2.28228228228228 -6.62291570850129
2.28628628628629 -6.6223666108551
2.29029029029029 -6.62175338501665
2.29429429429429 -6.62107603098594
2.2982982982983 -6.62033454876298
2.3023023023023 -6.61952893834776
2.30630630630631 -6.61865919974028
2.31031031031031 -6.61772533294055
2.31431431431431 -6.61672733794856
2.31831831831832 -6.61566521476431
2.32232232232232 -6.61453896338781
2.32632632632633 -6.61334858381905
2.33033033033033 -6.61209407605804
2.33433433433433 -6.61077544010477
2.33833833833834 -6.60939267595924
2.34234234234234 -6.60794578362146
2.34634634634635 -6.60643476309142
2.35035035035035 -6.60485961436912
2.35435435435435 -6.60322033745457
2.35835835835836 -6.60151693234776
2.36236236236236 -6.5997493990487
2.36636636636637 -6.59791773755738
2.37037037037037 -6.5960219478738
2.37437437437437 -6.59406202999797
2.37837837837838 -6.59203798392988
2.38238238238238 -6.58994980966953
2.38638638638639 -6.58779750721693
2.39039039039039 -6.58558107657207
2.39439439439439 -6.58330051773495
2.3983983983984 -6.58095583070558
2.4024024024024 -6.57854701548395
2.40640640640641 -6.57607407207007
2.41041041041041 -6.57353700046393
2.41441441441441 -6.57093580066553
2.41841841841842 -6.56827047267488
2.42242242242242 -6.56554101649197
2.42642642642643 -6.5627474321168
2.43043043043043 -6.55988971954938
2.43443443443443 -6.5569678787897
2.43843843843844 -6.55398190983777
2.44244244244244 -6.55093181269357
2.44644644644645 -6.54781758735713
2.45045045045045 -6.54463923382842
2.45445445445445 -6.54139675210746
2.45845845845846 -6.53809014219425
2.46246246246246 -6.53471940408877
2.46646646646647 -6.53128453779104
2.47047047047047 -6.52778554330106
2.47447447447447 -6.52422242061882
2.47847847847848 -6.52059516974432
2.48248248248248 -6.51690379067756
2.48648648648649 -6.51314828341855
2.49049049049049 -6.50932864796729
2.49449449449449 -6.50544488432376
2.4984984984985 -6.50149699248798
2.5025025025025 -6.49748497245995
2.50650650650651 -6.49340882423965
2.51051051051051 -6.48926854782711
2.51451451451451 -6.4850641432223
2.51851851851852 -6.48079561042524
2.52252252252252 -6.47646294943592
2.52652652652653 -6.47206616025435
2.53053053053053 -6.46760524288052
2.53453453453453 -6.46308019731443
2.53853853853854 -6.45849102355609
2.54254254254254 -6.45383772160549
2.54654654654655 -6.44912029146263
2.55055055055055 -6.44433873312752
2.55455455455455 -6.43949304660015
2.55855855855856 -6.43458323188053
2.56256256256256 -6.42960928896865
2.56656656656657 -6.42457121786451
2.57057057057057 -6.41946901856812
2.57457457457457 -6.41430269107947
2.57857857857858 -6.40907223539856
2.58258258258258 -6.4037776515254
2.58658658658659 -6.39841893945998
2.59059059059059 -6.39299609920231
2.59459459459459 -6.38750913075237
2.5985985985986 -6.38195803411019
2.6026026026026 -6.37634280927574
2.60660660660661 -6.37066345624904
2.61061061061061 -6.36491997503009
2.61461461461461 -6.35911236561887
2.61861861861862 -6.3532406280154
2.62262262262262 -6.34730476221968
2.62662662662663 -6.3413047682317
2.63063063063063 -6.33524064605146
2.63463463463463 -6.32911239567896
2.63863863863864 -6.32292001711421
2.64264264264264 -6.3166635103572
2.64664664664665 -6.31034287540794
2.65065065065065 -6.30395811226642
2.65465465465465 -6.29750922093264
2.65865865865866 -6.29099620140661
2.66266266266266 -6.28441905368832
2.66666666666667 -6.27777777777778
2.67067067067067 -6.27107237367498
2.67467467467467 -6.26430284137992
2.67867867867868 -6.2574691808926
2.68268268268268 -6.25057139221303
2.68668668668669 -6.24360947534121
2.69069069069069 -6.23658343027712
2.69469469469469 -6.22949325702078
2.6986986986987 -6.22233895557219
2.7027027027027 -6.21512052593134
2.70670670670671 -6.20783796809823
2.71071071071071 -6.20049128207286
2.71471471471471 -6.19308046785524
2.71871871871872 -6.18560552544537
2.72272272272272 -6.17806645484323
2.72672672672673 -6.17046325604884
2.73073073073073 -6.1627959290622
2.73473473473473 -6.15506447388329
2.73873873873874 -6.14726889051213
2.74274274274274 -6.13940917894872
2.74674674674675 -6.13148533919305
2.75075075075075 -6.12349737124512
2.75475475475475 -6.11544527510493
2.75875875875876 -6.10732905077249
2.76276276276276 -6.0991486982478
2.76676676676677 -6.09090421753084
2.77077077077077 -6.08259560862163
2.77477477477477 -6.07422287152017
2.77877877877878 -6.06578600622645
2.78278278278278 -6.05728501274047
2.78678678678679 -6.04871989106223
2.79079079079079 -6.04009064119174
2.79479479479479 -6.03139726312899
2.7987987987988 -6.02263975687399
2.8028028028028 -6.01381812242673
2.80680680680681 -6.00493235978722
2.81081081081081 -5.99598246895544
2.81481481481481 -5.98696844993141
2.81881881881882 -5.97789030271513
2.82282282282282 -5.96874802730659
2.82682682682683 -5.95954162370579
2.83083083083083 -5.95027109191273
2.83483483483483 -5.94093643192742
2.83883883883884 -5.93153764374986
2.84284284284284 -5.92207472738003
2.84684684684685 -5.91254768281795
2.85085085085085 -5.90295651006362
2.85485485485485 -5.89330120911703
2.85885885885886 -5.88358177997818
2.86286286286286 -5.87379822264707
2.86686686686687 -5.86395053712371
2.87087087087087 -5.85403872340809
2.87487487487487 -5.84406278150022
2.87887887887888 -5.83402271140009
2.88288288288288 -5.8239185131077
2.88688688688689 -5.81375018662306
2.89089089089089 -5.80351773194616
2.89489489489489 -5.793221149077
2.8988988988989 -5.78286043801559
2.9029029029029 -5.77243559876193
2.90690690690691 -5.761946631316
2.91091091091091 -5.75139353567782
2.91491491491491 -5.74077631184738
2.91891891891892 -5.73009495982469
2.92292292292292 -5.71934947960974
2.92692692692693 -5.70853987120253
2.93093093093093 -5.69766613460307
2.93493493493493 -5.68672826981135
2.93893893893894 -5.67572627682738
2.94294294294294 -5.66466015565115
2.94694694694695 -5.65352990628266
2.95095095095095 -5.64233552872192
2.95495495495495 -5.63107702296892
2.95895895895896 -5.61975438902366
2.96296296296296 -5.60836762688615
2.96696696696697 -5.59691673655638
2.97097097097097 -5.58540171803435
2.97497497497497 -5.57382257132007
2.97897897897898 -5.56217929641353
2.98298298298298 -5.55047189331474
2.98698698698699 -5.53870036202368
2.99099099099099 -5.52686470254038
2.99499499499499 -5.51496491486481
2.998998998999 -5.503000998997
3.003003003003 -5.49097295493692
3.00700700700701 -5.47888078268459
3.01101101101101 -5.46672448224
3.01501501501502 -5.45450405360315
3.01901901901902 -5.44221949677405
3.02302302302302 -5.42987081175269
3.02702702702703 -5.41745799853908
3.03103103103103 -5.40498105713321
3.03503503503504 -5.39243998753508
3.03903903903904 -5.3798347897447
3.04304304304304 -5.36716546376206
3.04704704704705 -5.35443200958716
3.05105105105105 -5.34163442722001
3.05505505505506 -5.3287727166606
3.05905905905906 -5.31584687790894
3.06306306306306 -5.30285691096502
3.06706706706707 -5.28980281582884
3.07107107107107 -5.27668459250041
3.07507507507508 -5.26350224097972
3.07907907907908 -5.25025576126677
3.08308308308308 -5.23694515336157
3.08708708708709 -5.22357041726411
3.09109109109109 -5.2101315529744
3.0950950950951 -5.19662856049242
3.0990990990991 -5.1830614398182
3.1031031031031 -5.16943019095171
3.10710710710711 -5.15573481389297
3.11111111111111 -5.14197530864197
3.11511511511512 -5.12815167519872
3.11911911911912 -5.11426391356321
3.12312312312312 -5.10031202373545
3.12712712712713 -5.08629600571542
3.13113113113113 -5.07221585950315
3.13513513513514 -5.05807158509861
3.13913913913914 -5.04386318250182
3.14314314314314 -5.02959065171278
3.14714714714715 -5.01525399273147
3.15115115115115 -5.00085320555791
3.15515515515516 -4.98638829019209
3.15915915915916 -4.97185924663402
3.16316316316316 -4.95726607488369
3.16716716716717 -4.94260877494111
3.17117117117117 -4.92788734680627
3.17517517517518 -4.91310179047917
3.17917917917918 -4.89825210595981
3.18318318318318 -4.8833382932482
3.18718718718719 -4.86836035234434
3.19119119119119 -4.85331828324821
3.1951951951952 -4.83821208595983
3.1991991991992 -4.8230417604792
3.2032032032032 -4.80780730680631
3.20720720720721 -4.79250872494116
3.21121121121121 -4.77714601488375
3.21521521521522 -4.76171917663409
3.21921921921922 -4.74622821019218
3.22322322322322 -4.730673115558
3.22722722722723 -4.71505389273157
3.23123123123123 -4.69937054171288
3.23523523523524 -4.68362306250194
3.23923923923924 -4.66781145509874
3.24324324324324 -4.65193571950329
3.24724724724725 -4.63599585571558
3.25125125125125 -4.61999186373561
3.25525525525526 -4.60392374356338
3.25925925925926 -4.5877914951989
3.26326326326326 -4.57159511864216
3.26726726726727 -4.55533461389317
3.27127127127127 -4.53900998095192
3.27527527527528 -4.52262121981842
3.27927927927928 -4.50616833049266
3.28328328328328 -4.48965131297464
3.28728728728729 -4.47307016726436
3.29129129129129 -4.45642489336183
3.2952952952953 -4.43971549126704
3.2992992992993 -4.42294196098
3.3033033033033 -4.4061043025007
3.30730730730731 -4.38920251582914
3.31131131131131 -4.37223660096533
3.31531531531532 -4.35520655790926
3.31931931931932 -4.33811238666094
3.32332332332332 -4.32095408722035
3.32732732732733 -4.30373165958752
3.33133133133133 -4.28644510376242
3.33533533533534 -4.26909441974507
3.33933933933934 -4.25167960753546
3.34334334334334 -4.2342006671336
3.34734734734735 -4.21665759853948
3.35135135135135 -4.19905040175311
3.35535535535536 -4.18137907677447
3.35935935935936 -4.16364362360358
3.36336336336336 -4.14584404224044
3.36736736736737 -4.12798033268504
3.37137137137137 -4.11005249493738
3.37537537537538 -4.09206052899747
3.37937937937938 -4.0740044348653
3.38338338338338 -4.05588421254087
3.38738738738739 -4.03769986202419
3.39139139139139 -4.01945138331525
3.3953953953954 -4.00113877641405
3.3993993993994 -3.9827620413206
3.4034034034034 -3.96432117803489
3.40740740740741 -3.94581618655693
3.41141141141141 -3.92724706688671
3.41541541541542 -3.90861381902423
3.41941941941942 -3.8899164429695
3.42342342342342 -3.87115493872251
3.42742742742743 -3.85232930628326
3.43143143143143 -3.83343954565176
3.43543543543544 -3.814485656828
3.43943943943944 -3.79546763981199
3.44344344344344 -3.77638549460371
3.44744744744745 -3.75723922120318
3.45145145145145 -3.7380288196104
3.45545545545546 -3.71875428982536
3.45945945945946 -3.69941563184806
3.46346346346346 -3.68001284567851
3.46746746746747 -3.6605459313167
3.47147147147147 -3.64101488876264
3.47547547547548 -3.62141971801632
3.47947947947948 -3.60176041907774
3.48348348348348 -3.5820369919469
3.48748748748749 -3.56224943662381
3.49149149149149 -3.54239775310846
3.4954954954955 -3.52248194140086
3.4994994994995 -3.502502001501
3.5035035035035 -3.48245793340888
3.50750750750751 -3.46234973712451
3.51151151151151 -3.44217741264788
3.51551551551552 -3.421940959979
3.51951951951952 -3.40164037911786
3.52352352352352 -3.38127567006446
3.52752752752753 -3.3608468328188
3.53153153153153 -3.3403538673809
3.53553553553554 -3.31979677375073
3.53953953953954 -3.2991755519283
3.54354354354354 -3.27849020191362
3.54754754754755 -3.25774072370669
3.55155155155155 -3.2369271173075
3.55555555555556 -3.21604938271605
3.55955955955956 -3.19510751993234
3.56356356356356 -3.17410152895638
3.56756756756757 -3.15303140978817
3.57157157157157 -3.13189716242769
3.57557557557558 -3.11069878687496
3.57957957957958 -3.08943628312998
3.58358358358358 -3.06810965119273
3.58758758758759 -3.04671889106324
3.59159159159159 -3.02526400274148
3.5955955955956 -3.00374498622747
3.5995995995996 -2.9821618415212
3.6036036036036 -2.96051456862268
3.60760760760761 -2.9388031675319
3.61161161161161 -2.91702763824886
3.61561561561562 -2.89518798077357
3.61961961961962 -2.87328419510602
3.62362362362362 -2.85131628124621
3.62762762762763 -2.82928423919415
3.63163163163163 -2.80718806894983
3.63563563563564 -2.78502777051326
3.63963963963964 -2.76280334388442
3.64364364364364 -2.74051478906334
3.64764764764765 -2.71816210604999
3.65165165165165 -2.6957452948444
3.65565565565566 -2.67326435544654
3.65965965965966 -2.65071928785642
3.66366366366366 -2.62811009207406
3.66766766766767 -2.60543676809943
3.67167167167167 -2.58269931593255
3.67567567567568 -2.55989773557341
3.67967967967968 -2.53703202702202
3.68368368368368 -2.51410219027837
3.68768768768769 -2.49110822534246
3.69169169169169 -2.4680501322143
3.6956956956957 -2.44492791089388
3.6996996996997 -2.4217415613812
3.7037037037037 -2.39849108367627
3.70770770770771 -2.37517647777908
3.71171171171171 -2.35179774368964
3.71571571571572 -2.32835488140794
3.71971971971972 -2.30484789093398
3.72372372372372 -2.28127677226776
3.72772772772773 -2.25764152540929
3.73173173173173 -2.23394215035857
3.73573573573574 -2.21017864711558
3.73973973973974 -2.18635101568034
3.74374374374374 -2.16245925605285
3.74774774774775 -2.1385033682331
3.75175175175175 -2.11448335222109
3.75575575575576 -2.09039920801682
3.75975975975976 -2.0662509356203
3.76376376376376 -2.04203853503153
3.76776776776777 -2.0177620062505
3.77177177177177 -1.99342134927721
3.77577577577578 -1.96901656411166
3.77977977977978 -1.94454765075386
3.78378378378378 -1.9200146092038
3.78778778778779 -1.89541743946148
3.79179179179179 -1.87075614152691
3.7957957957958 -1.84603071540008
3.7997997997998 -1.821241161081
3.8038038038038 -1.79638747856966
3.80780780780781 -1.77146966786607
3.81181181181181 -1.74648772897021
3.81581581581582 -1.7214416618821
3.81981981981982 -1.69633146660174
3.82382382382382 -1.67115714312912
3.82782782782783 -1.64591869146424
3.83183183183183 -1.6206161116071
3.83583583583584 -1.59524940355771
3.83983983983984 -1.56981856731606
3.84384384384384 -1.54432360288216
3.84784784784785 -1.518764510256
3.85185185185185 -1.49314128943759
3.85585585585586 -1.46745394042691
3.85985985985986 -1.44170246322398
3.86386386386386 -1.4158868578288
3.86786786786787 -1.39000712424136
3.87187187187187 -1.36406326246166
3.87587587587588 -1.33805527248971
3.87987987987988 -1.3119831543255
3.88388388388388 -1.28584690796903
3.88788788788789 -1.25964653342031
3.89189189189189 -1.23338203067933
3.8958958958959 -1.20705339974609
3.8998998998999 -1.1806606406206
3.9039039039039 -1.15420375330285
3.90790790790791 -1.12768273779285
3.91191191191191 -1.10109759409059
3.91591591591592 -1.07444832219607
3.91991991991992 -1.0477349221093
3.92392392392392 -1.02095739383027
3.92792792792793 -0.994115737358982
3.93193193193193 -0.96720995269544
3.93593593593594 -0.940240039839639
3.93993993993994 -0.913205998791584
3.94394394394394 -0.886107829551273
3.94794794794795 -0.858945532118707
3.95195195195195 -0.831719106493883
3.95595595595596 -0.8044285526768
3.95995995995996 -0.777073870667464
3.96396396396396 -0.749655060465872
3.96796796796797 -0.722172122072023
3.97197197197197 -0.694625055485915
3.97597597597598 -0.667013860707554
3.97997997997998 -0.639338537736936
3.98398398398398 -0.611599086574062
3.98798798798799 -0.583795507218932
3.99199199199199 -0.555927799671542
3.995995995996 -0.527995963931899
4 -0.5
};
\node at (axis cs:2,-5)[
  %scale=0.5,
  anchor=base west,
  text=black,
  rotate=0.0
]{test};
\node at (axis cs:3,-13)[
  %scale=0.5,
  anchor=base west,
  text=black,
  rotate=0.0
]{train};
\end{axis}

\end{tikzpicture}
	\end{figure}
}

\note{
	\begin{itemize}
		\item Different architectures with different complexities
		\item Spkit data set in training and test set
		\item Want to minimize test error $\rightarrow$ trial and error
	\end{itemize}
}

\mframe{Restricted Boltzmann Machines}{}{
	\begin{figure}
		\centering
		\begin{tikzpicture}

% Define visible units
\node[input] (x2) {};
\node[input] at (1,+3/2) (x1) {};
\node[input] at (1,-3/2) (x3) {};

% Define hidden units
\node[input] at (3,+3/2) (h1) {};
\node[input] at (4,0) (h2) {};
\node[input] at (3,-3/2) (h3) {};

% Define biases
\node[input] at (0,7/2) (a) {};
\node[input] at (4,7/2) (b) {};

% Define paths
\path[draw,thick,-] (x1) -- (h1);
\path[draw,thick,-] (x1) -- (h2);
\path[draw,thick,-] (x1) -- (h3);

\path[draw,thick,-] (x2) -- (h1);
\path[draw,thick,-] (x2) -- (h2);
\path[draw,thick,-] (x2) -- (h3);

\path[draw,thick,-] (x3) -- (h1);
\path[draw,thick,-] (x3) -- (h2);
\path[draw,thick,-] (x3) -- (h3);

\draw[color0,thick,-] (b) to [out=270,in=90] (h1);
\draw[color0,thick,-] (b) to [out=315,in=45] (h2);
\draw[color0,thick,-] (b) to [out=0,in=0] (h3);

\draw[color1,thick,-] (a) to [out=270,in=90] (x1);
\draw[color1,thick,-] (a) to [out=225,in=135] (x2);
\draw[color1,thick,-] (a) to  [out=180,in=180] (x3);


% Add some text
\node[below=1em of x3] {visible};
\node[below=1em of h3] {hidden};
\end{tikzpicture}

	\end{figure}
	\pause
	\begin{equation}
	E(\bs{x},\bs{h})=-\sum_{i=1}^V\frac{(x_i-a_i)^2}{2\sigma_i^2}-\sum_{j=1}^Hh_jb_j-\sum_{i=1}^V\sum_{j=1}^H\frac{x_iw_{ij}h_j}{\sigma_i^2}
	\end{equation}
}

\note{
	\begin{itemize}
		\item What we have used in our work
		\item Energy based model
		\item Differs from FNN $\rightarrow$ obey unsupervised $\rightarrow$ no labeled data
		\item Finds the most likely configuration by minimizing the system energy
	\end{itemize}
}

\mframe{Probability Distribution}{}{
	The joint probability distribution is given by the Boltzmann distribution:
	
	\begin{equation}
	P(\bs{x},\bs{h})=\frac{1}{Z}\exp(-E(\bs{x},\bs{h})/kT).
	\end{equation}
	The marginal distribution of the visible units is given by
	\begin{equation}
	P(\bs{x})=\sum_{\{\bs{h}\}}P(\bs{x},\bs{h}).
	\end{equation}
}

\note{
	\begin{itemize}
		\item Named Boltzmann machine because of the joint probability distribution
		\item Find the marginal distribution of the visible units by integrating over all the hidden units
	\end{itemize}
}