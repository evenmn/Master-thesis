\chapter{Quantum Many-Body Physics} \label{sec:quantum}
\epigraph{If you are not completely confused by quantum mechanics, you do not understand it.}{John Wheeler}
\begin{figure}[H]
	\centering
	\captionsetup[subfigure]{labelformat=empty}
	\includegraphics[scale=3.0]{Images/art_quantum.jpg}
	\caption{The first photograph of a Hydrogen atom, which shows the true wave function. Published in Phys. rev. lett. 110, 213001 (2017), \textit{Hydrogen atoms under magnification}.}
\end{figure}
Around 1900, some physicists thought that there were nothing new to be discovered in physics and all that remained was more precise measurements, as Lord Kelvin famously pointed out. He could not be more wrong. In the following years, things were observed that could only be described by a quantized theory, led by Albert Einstein's explanation of the photoelectric effect in 1905. 

Immense efforts were placed on completing the theory, and contributions from an array of scientists over a period of 20 years were necessary to get it finished. In 1929, Paul Dirac stated something similar to what Lord Kelvin said 30 years earlier, but hopefully with greater accuracy. 

\newpage
\section{Introductory Quantum Physics} \label{subsec:elementary}
In this section we will present the fundamentals of the quantum theory, that will make up the framework of this project. The theory is based on David Griffith's incredible textbook, \textit{Introduction to Quantum Mechanics}, where the reader is relegated for in-depth information.

Before we get started, we make a few assumptions in order to simplify our problem. They most important ones are specified below with an explanation why they are valid.

\begin{itemize}
	\item \textbf{Point-like particles:} First, all particles involved will be assumed to be point-like, such that they lack spatial extension. This includes the nucleus in atomic systems, but it still makes sense since the distance from the nucleus to the electrons is known to be much larger then the nucleus extent.
	
	\item \textbf{Non-relativistic spacetime:}  Second, we operate in the non-relativistic spacetime, which is valid as long as we do not approach the speed of light and the forces are not strong. Applying classical physics, we can find that the speed of the electron in a hydrogen atom is about 1\% of the speed of light, and even though the electrons get higher speed in heavier atoms, we do not need to worry about it. The forces acting are the weak Coulomb forces.
	\item For specific systems we might make new assumptions and approximate. For instance, for atomic systems we will assume that the nucleus is at rest. Those approximations will be discussed consecutively. 
\end{itemize}

\subsection{The Schrödinger Equation} \label{subsubsec:schrodinger}
The Schrödinger equation is a natural starting point, which gives the energy eigenstates of a system defined by a Hamiltonian $\hat{\mathcal{H}}$ and its eigenfunctions, $\Psi$, which are the wave functions. The time-independent Schrödinger equation reads
\begin{equation}
\label{eq:Energy}
 \hat{\mathcal{H}}\psin=\epsilon_n\psin
\end{equation}
where the Hamiltonian is the total energy operator. By analogy with the classical mechanics, this is given by
\begin{equation}
\hat{\mathcal{H}}=\hat{\mathcal{T}}+\hat{\mathcal{V}}
\end{equation}
with $\hat{\mathcal{T}}$ and $\hat{\mathcal{V}}$ as the kinetic and potential energy operators respectively. 

The kinetic energy yields $T=p^2/2m$, such that the kinetic energy operator can be represented as 
\begin{equation}
\hat{\mathcal{T}}=\frac{\hat{\mathcal{P}}^2}{2m}
\end{equation}
according to Ehrenfest's theorem. Further, the momentum operator is $\hat{\mathcal{P}}=-i\hbar\nabla$.

The potential energy can be split into an external part and an interaction part, where the latter is given by the Coulomb interaction. For two identical particles of charge $q$, the repulsive interaction gives the energy
\begin{equation}
V_I=k\frac{q^2}{r_{12}}
\end{equation}
where $r_{12}$ is the distance between the particles. The total Hamiltonian of a system of $N$ identical particles takes the form
\begin{equation}
\hat{\mathcal{H}}=-\sum_i^N\frac{\hbar^2}{2m}\nabla_i^2+\sum_i^Nu_i + \sum_i^N\sum_{j>i}^Mk\frac{q^2}{r_{ij}}
\label{eq:ElectronicHamiltonian}
\end{equation}
which is the farthest we can go without specifying the external potential $u_i$. $r_{ij}$ is the relative distance between particle $i$ and $j$, defined by $r_{ij}\equiv|\bs{r}_i-\bs{r}_j|$.

Setting up equation \eqref{eq:Energy} with respect to the energies, we obtain an integral,
\begin{equation}
\epsilon_n=\frac{\int d\bs{r}\psinc\hatH\psin}{\int d\bs{r}\psinc\psin},
\label{eq:energyintegral}
\end{equation}
which not necessarily is trivial to solve. If we take the wave function squared we get the probability of finding a particle at a certain position,
\begin{equation}
P(\bs{r})=\psinc\psin=|\psin|^2
\end{equation}
so the nominator is simply the integral over all probabilities. If the wave function is normalized correctly, this should always give 1. Assuming that is the case, the expectation value can be expressed more elegantly by using Dirac notation,
\begin{equation}
E[\Psi]=\mel{\Psi}{\hatH}{\Psi},
\end{equation}
where the first part, $\bra{\Psi}$ is called a bra and the last part, $\ket{\Psi}$ is called a ket. At first this might look artificial and less informative, but it simplifies the notation significantly. More information about the notation is found in Appendix A. 

In many cases we do not know the exact wave function, and need to rely on a trial wave function guess. Henceforth, we will use $\Psi$ as the exact total wave function, $\psi$ as the exact single particle function (SPF), $\Psi_T$ as the total trial wave function and $\phi$ as the trial SPF. 
\cite{GriffQuan}

\subsection{The Variational Principle}
In the equations above, the presented wave functions are assumed to be the exact eigen functions of the Hamiltonian. But often we do not know the exact wave functions, and we need to guess what the wave functions might be. In those cases we make use of the variational principle, which states that only the exact ground state wave function is able to give the ground state energy. All other wave functions that fulfill the required properties (see section \ref{subsec:wavefunction}) give higher energies, and mathematically we can express the statement
\begin{equation}
\epsilon_0\leq\mel{\Psi_T}{\hat{\mathcal{H}}}{\Psi_T}.
\label{eq:variationalprinciple}
\end{equation}

Variational Monte-Carlo is a method based on (and named after) the variational principle, where we vary the trial wave function in order to obtain the lowest energy. It will be detailed in chaper \eqref{chp:methods}.

\subsection{Postulates of Quantum Mechanics}
The quantum theory is built on a few fundamental postulates, which will always be true. Before we go further, we will take a quick look at some of them.

\begin{enumerate}
\item Associated with any particle moving in a conservative field of force is a wave function which determines everything that can be known about the system.

\item With every physical observable q there is associated an operator Q, which when operating upon the wavefunction associated with a definite value of that observable will yield that value times the wavefunction.

\item Any operator Q associated with a physically measurable property q will be Hermitian.

\item The set of eigenfunctions of operator Q will form a complete set of linearly independent functions.

\item For a system described by a given wavefunction, the expectation value of any property q can be found by performing the expectation value integral with respect to that wavefunction.

\item The time evolution of the wavefunction is given by the time dependent Schrodinger equation. 
\end{enumerate}

\section{The Trial Wave Function} \label{subsec:wavefunction}
By the first postulate of quantum mechanics, the wave function contains all the information specifying the state of the system. This means that all observable in classical mechanics can also be measured from the wave function, resulting finding the wave function as our main goal. 

The true wave function is in most cases unknown, and we need an initial wave function based on educated guesses, called a trial wave function. This function has to meet some requirements in order to be used in the variational principle, including the CUSP condition, normalizable and ... In addition, the wave function should be symmetric or anti-symmetric under exchange of two coordinates.

\subsection{Bosons and fermions} \label{subsubsec:symmetry}
An introductary sentence is needed to increase the FLOW. 
Assume we have a permutation operator $\hat{P}$ which exchanges two coordinates in the wave function,

\begin{equation}
\hat{P}(i\rightarrow j)\Psi_n(\bs{x}_1,\hdots,\bs{x}_i,\hdots,\bs{x}_j,\hdots,\bs{x}_M)=p\Psi_n(\bs{x}_1,\hdots,\bs{x}_j,\hdots,\bs{x}_i,\hdots,\bs{x}_M),
\end{equation}
where $p$ is just a factor which comes from the transformation. If we again apply the $\hat{P}$ operator, we should switch the same coordinates back, and we expect to end up with the initial wave function. For that reason, $p=\pm1$. \footnote{This was true until 1976, when J.M. Leinaas and J. Myrheim discovered the anyon, https://www.uio.no/studier/emner/matnat/fys/FYS4130/v14/documents/kompendium.pdf.}

The particles that have an antisymmetric (AS) wavefunction under exchange of two coordinates are called fermions, named after Enrico Fermi, and have half integer spin. On the other hand, the particles that have a symmetric (S) wavefunction under exchange of two coordinates are called bosons, named after Satyendra Nath Bose, and have integer spin. 

It turns out that because of their antisymmetric wavefunction, two identical fermions cannot be found at the same position at the same time, known as the Pauli principle. This causes some difficulties when dealing with multiple fermions, because we always need to ensure that the total wavefunction becomes zero if two identical particles happen to be at the same position. To do this, we introduce a Slater determinant as described in the next chapter. In this particular project, we are going to focus on electrons and therefore fermions. Anyway, much of the theory applies for bosons as well.

Read https://manybodyphysics.github.io/FYS4480/doc/pub/secondquant/html/secondquant-bs.html

\subsection{Slater determinant} \label{subsubsec:slater}
For a system of more particles we can define a total wavefunction, which is a composition of all the single particle wavefunctions (SPF) and contains all the information about the system. For fermions we need to compile the SPFs such that the Pauli principle is fulfilled at all times. One way to do this is by setting up the SPFs in a determinant, known as a Slater determinant.

Consider a system of two identical fermions with SPFs $\phi_1$ and $\phi_2$ at positions $\boldsymbol{r}_1$ and $\boldsymbol{r}_2$ respectively. The way we define the wavefunction of the system is then
\begin{equation}
\Psi_T(\bs{r}_1,\bs{r}_2)=
\begin{vmatrix}
\phi_1(\boldsymbol{r}_1) & \phi_2(\boldsymbol{r}_1)\\
\phi_1(\boldsymbol{r}_2) & \phi_2(\boldsymbol{r}_2)
\end{vmatrix}
=\phi_1(\boldsymbol{r}_1)\phi_2(\boldsymbol{r}_2)-\phi_2(\boldsymbol{r}_1)\phi_1(\boldsymbol{r}_2),
\end{equation}
which is set to zero if the particles are at the same position. This is called a Slater determinant, and yields the same no matter how big the system is.

Notice that we denote the wave function with the $'T'$, which indicates that it is a trial wave function. We do this because the spin part is avoided with $\psi$ as the radial parts only, thus this wavefunction is not the true wavefunction. We will look closer at how we can factorize out the spin part later. The spin part is assumed to not affect the energies.

A general Slater determinant for a system of $N$ particles takes the form

\begin{equation}
\Psi(\boldsymbol{r}_1,\boldsymbol{r}_2,\hdots,\boldsymbol{r}_N)=
\begin{vmatrix}
\psi_1(\boldsymbol{r}_1) & \psi_2(\boldsymbol{r}_1) & \hdots & \psi_N(\boldsymbol{r}_1)\\
\psi_1(\boldsymbol{r}_2) & \psi_2(\boldsymbol{r}_2) & \hdots & \psi_N(\boldsymbol{r}_2)\\
\vdots & \vdots & \ddots & \vdots \\
\psi_1(\boldsymbol{r}_N) & \psi_2(\boldsymbol{r}_N) & \hdots & \psi_N(\boldsymbol{r}_N)
\end{vmatrix}
\end{equation}
where the $\psi$'s are the true single particle wave functions, which are the tensor products 
\begin{equation}
\psi=\phi\otimes\xi
\end{equation}
with $\xi$ as the spin part. 

There is a similar "Slater permanent" which applies for bosons. It is similar to a determinant, but all negative signs are replaced by positive signs. 

\subsection{Electron structure calculations} \label{subsubsec:electronsystem}
For our purpose we will study fermions with spin $\sigma=\pm 1/2$ only, which can be seen as electrons. In this particular case, the SPFs can be arranged in spin-up and spin-down parts, such that the Slater determinant can be simplied to 
\begin{equation}
\Psi(\boldsymbol{r}_1,\boldsymbol{r}_2,\hdots,\boldsymbol{r}_N)=
\begin{vmatrix}
\phi_1(\boldsymbol{r}_1)\xi_{\uparrow} & \phi_1(\boldsymbol{r}_1)\xi_{\downarrow} & \hdots & \phi_{N/2}(\boldsymbol{r}_1)\xi_{\downarrow}\\
\phi_1(\boldsymbol{r}_2)\xi_{\uparrow} & \phi_1(\boldsymbol{r}_2)\xi_{\downarrow} & \hdots & \phi_{N/2}(\boldsymbol{r}_2)\xi_{\downarrow}\\
\vdots & \vdots & \ddots & \vdots \\
\phi_1(\boldsymbol{r}_N)\xi_{\uparrow} & \phi_1(\boldsymbol{r}_N)\xi_{\downarrow} & \hdots & \phi_{N/2}(\boldsymbol{r}_N)\xi_{\downarrow}
\end{vmatrix}.
\end{equation}
This is called the wavefunction ansatz, because assumptions are raised, like two particles with opposite spins are found to be at the same position all the time, i.e, an equal number of fermions have spin up as spin down, are applied. Further the...

SHOULD END UP WITH SPLITTED DETERMINANTS HERE

For a detailed walkthrough, see appendix I in REF(The Stochastic Gradient Approximation: an application to Li nanoclusters, Daniel Nissenbaum). 

\subsection{Cusp condition} \label{subsubsec:cusp}
The Coulomb potential gives a cusp when two charged particles approach each other. This means that the wave function should also have a cusp, known as the Coulomb cusp condition, Kato theorem or just the cusp condition. 

\subsection{Basis set} \label{subsubsec:basisset}
To go further, we need to define a basis set, $\phi_n(\bs{r})$ which should be chosen carefully based on the system. 

\subsection{Jastrow factor} \label{subsubsec:jastrow}
The Jastrow factor is introduced in order to capture interaction properties, ...

We will use the Padé-Jastrow factor, which can be expressed as
\begin{equation}
J(\bs{r}; \beta, \gamma) = \exp\bigg(\sum_{i=1}^N\sum_{j=1}^i\frac{\beta(i,j)r_{ij}}{1+\gamma r_{ij}}\bigg)
\label{eq:PadeJastrow}
\end{equation}
which gets small when $r_{ij}$ gets small. If we recall that the probability distribution is given by the wave function squared, we see that the Padé-Jastrow factor gives a lower probability of finding particles close to each other, which is what we want.  





