\chapter{Introduction}
The properties and behavior of quantum many-body systems are determined by the laws of quantum physics which have been known since the 1930s. The time-dependent Schrödinger equation describes the binding energy of atoms and molecules, as well as the interaction between particles in a gas. In addition, it has been used to determine the energy of artificial structures like quantum dots, nanowires and ultracold condensates. As the quantum theory is the most precisely tested theory in the history of science, computer experiments are capable of obtaining the energy as precise as laboratory experiments, and can in that sense replace laboratory experiments. 



Even though we know the laws of quantum mechanics, many challenges are encountered when calculating real-world problems. First, interesting systems often involve large number of particles, which causes expensive calculations. Second, the correct wave functions are seldomly known for a complex system, which is vital for measuring the observable accurately. They are both covered by the many-body problem. 

\section{The many-body problem}
In quantum mechanics, a many-body system contains three or more particles interacting with each other. These interactions create so-called quantum correlations, which makes the wave function of the system a complicated object holding a large amount of information. As a consequence, exact or analytical calculations become impractical or even impossible, which is known as the many-body problem. Indeed, Paul Dirac recognized those problems already in 1929:

\begin{shadequote}{
		The general theory of quantum mechanics is now almost complete... ...The underlying physical laws necessary for the mathematical theory of a large part of physics and the whole of chemistry are thus completely known, and the difficulty is only that the exact application of these laws leads to equations much too complicated to be soluble. \par Paul M. Dirac, \emph{Quantum Mechanics of Many-electron Systems}, \cite{dirac_paul_adrien_maurice_quantum_1929}.}
\end{shadequote}

There are numerous approaches to solve this problem where approximative methods often are used to reduce the sometimes extreme computational cost. Popular methods include the Hartree-Fock method, which replaces the interaction by a mean field, and methods like Full Configuration Interaction (FCI) and Coupled Cluster which seek to solve the problem by an expansion of wave functions. Lastly, quantum Monte Carlo (QMC) methods are totally different approaches attempting to solve the problem directly using a stochastic evaluations of the integrals occurring from the Schrödinger equation. 

What all these methods have in common, is that they require significant amounts of physical intuition to work. In general, prior knowledge of the wave function, covering all the interactions and the cusps, is required. This knowledge is often unavailable, especially for complex systems, which also make accurate estimates of the observable unavailable. In this thesis we will try to bypass this problem inventing more flexible and robust methods which allow a relatively bad wave function guess. A natural base for this are the machine learning algorithms, as they can "learn" themselves and thus hopefully find good estimates through a training process. The obvious link between machine learning and quantum mechanics are the QMC methods, since they both are based on minimizing a \textit{cost function} in order to obtain optimal configurations. The connection between machine learning and QMC methods, in particular variational Monte Carlo (VMC), will be discussed thoroughly throughout this thesis.

\section{Machine learning} \label{sec:machinelearning}
Machine learning has recently achieved immense popularity in fields such as computer vision, economics, autonomy - and science, due to its ability of learning without being explicitly programmed. As a branch of artificial intelligence, machine learning is based on studies of the human brain and attempts to recreate the way neurons in the brain process information. 

Especially the artificial neural networks have experienced a significant progress over the past decade, which can be attributed to an array of innovations. Most notably, the convolutional neural network (CNN) \textbf{AlexNet}  managed to increase the top-5 test error rate of image recognition with a remarkable 11.1\% compared to the second best back in 2012 \cite{krizhevsky_imagenet_2012}! Today, the CNNs have been further improved, and they are even able to beat humans in recognizing images \cite{alom_history_2018}. Also speech recognition algorithms have lately been revolutionized, thanks to recurrent neural networks (RNNs), and especially long short-term memory (LSTM) networks. Their ability to recognize sequential (time-dependent) data made the technology good enough for an entry to millions of peoples everyday-life through services such as \textbf{Google Translate} \cite{wu_googles_2016}, Apple's \textbf{Siri} \cite{smith_ios_2016} and \textbf{Amazon Alexa} \cite{noauthor_bringing_nodate}. It is also interesting to see how machine learning has made computers eminent tacticians using reinforcement learning. The \textbf{Google DeepMind} developed program \textbf{AlphaGo} demonstrated this by beating the 9-dan professional L. Sedol in the board game Go \cite{noauthor_alphago_nodate}, before an improved version, \textbf{AlphaZero}, beat the at that time highest rated chess computer, \textbf{StockFish}, playing chess \cite{klein_mikeklein_googles_nodate}. Both these scenarios were unbelievable just a couple of decades ago.

Even though all these branches are both exciting and promising, they will not be discussed further in this work, since they will simply not work for our purposes. The reason is that they initially require a data set with known outputs in order to be trained, they obey so-called \textit{supervised} learning. For our quantum mechanical systems, we do not have those targets and need to rely on \textit{unsupervised} learning with the focus on restricted Boltzmann machines (RBMs). Lately, some effort has been put into this field, known as quantum machine learning. G.Carleo and M.Troyer demonstrated the link between RBMs and QMC and named the states \textit{neural-network quantum states} (NQS). They used the technique to study the Ising model and the Heisenberg model \cite{carleo_solving_2017}. V.Flugsrud went further and investigated circular quantum dots with the same method \cite{flugsrud_vilde_moe_solving_nodate}. We will extend the work she did to larger quantum dots and double quantum dots.

\section{Quantum dots}
Quantum dots are often called artificial atoms because of their common features to real atoms, and their popularity is increasing due to their applications in semiconductor technology. For instance, quantum dots are expected to be the next big thing in display technology due to their ability of emitting photons of specific wavelength in addition to using 30\% less energy than today's LED displays \cite{manders_8.3:_2015}. Samsung already claim that they use this technology in their displays \cite{noauthor_2019_nodate}.

Another reason why we are interested in simulating quantum dots, is that there exist experiments which can be used as benchmarks. Due to very strong confinement in the $z$-direction, the experimental dots, made by patterning GaAs/AlGaAs heterostructures, become essentially two-dimensional \cite{marzin_photoluminescence_1994,brunner_sharp-line_1994}. For that reason, our main focus in this work is two-dimensional dots, but also dots of three dimensions will be investigated. Quantum dots also allow the study of Wigner crystals, as their strength can be decreased such that the interaction energy dominates the kinetic energy. 

\section{Computer experiments}
The advent of computer technology has offered us a new window of opportunity for studies of quantum (and many other) problems. It serves as a third way of doing science which is based on simulations, in contrast to laboratory experiments and analytical approaches. In a broad sense, by simulations we mean computational models of reality based on physical laws, such as the fundamental Schrödinger equation. Such models have value when they enable to make predictions or to provide new information which is otherwise impossible or too costly to obtain. In this respect, QMC methods represent an illustration and an example of what is the potential of such methodologies.

The use and popularity of QMC methods has increased as personal computers and computer clusters have been more powerful. With today's strong computers, we see those methods as a natural choice when ground state properties of quantum mechanical systems are investigated. Even the simplest method, VMC, does typically yield excellent results, and the more complicated diffusion Monte Carlo (DMC) is in principle capable of employing exact results. They both appears to have among the highest performance-to-cost ratios out of all the quantum many-body methods. 

Albeit the fact that the QMC methods relatively recently have been applied in large scale, some of the ideas go back to the time before the invention of the electronic computer. Already in the 1940's, Enrico Fermi revealed the similarities between the imaginary time Schrödinger equation and stochastic processes in statistical mechanics. The first attempt of using this link on actual calculations, was performed by a group of scientists at Los Alamos National laboratory in the early 1950's when they tried to compute the ground state energy of the hydrogen molecule using a simple version of VMC \cite{bajdich_electronic_2010}. Around the same time, Metropolis et. al. introduced the original Metropolis algorithm which estimates the integrals by moving particles randomly in an ergodic scheme and rejecting inappropriate moves \cite{metropolis_monte_1949}. This method was further improved in the late 1950's when Kalos laid down the statistical and mathematical framework for the Green's function in QMC methods, and Hastings et. al. developed an efficient algorithm based on the theory, where the particles are moved after an educated guess \cite{hastings_monte_1970}. The use of the QMC methods on many-fermion systems was first done by Ceperly and Alder in the 1980's, and started a new era of stochastic methods applied to electronic structure problems \cite{ceperley_quantum_1986}. 

\section{Ethics in science}
In science as an entirety, there are some general guidelines that we all should follow in order to maintain an ethical behavior. Firstly, one should always have respect for others work and the authors should be credited whenever one uses others work, no matter the scope. This includes illustrations, text, code, methods, algorithms and so on, which are covered by the Copyright Act (in Norway, åndsverkloven). Secondly, all the research that one does should always be detailed in a such way that the experiments and results can be reproduced by others. This means that all the factors which possibly have a significant impact on the experiments should be described, and computer experiments is no exception. Lastly, there are unfortunately many examples on misuse of knowledge throughout the history, which obviously has to be avoided.

In our specific work, most of the ethical aspects are related to the use of machine learning, which can cause fatal consequences if it is not used correctly. The fact that machine learning allows the computers to learn things themselves have made profiled people like Stephen Hawking and Elon Musk warn us that they can be greatly misused if they are set on learning the wrong things \cite{cellan-jones_hawking:_2014, vance_elon_2015}. Every person who develop machine learning algorithms should take this warning seriously, remember that in the end there are one of us who end up creating the multi-headed monster \textit{Hydra}.

\section{Our goals and contributions} \label{sec:goals}
The goals of this thesis are mainly related to the attempt of develop and investigate a quantum many-body method that required less physical intuition. In order to do this, the implementation of code is essential, which obviously is a significant part of the work and thereby the goals, including a VMC code for benchmark purposes. Further, the next goal is naturally to obtain results, and we will mostly look at the quantum dots. In the end, we will provide a thorough comparison of the VMC and RBM results, inclusive wave function studies. We can summarize the goals in a list
\begin{itemize}
	\item Develop a VMC code to study large fermionic systems.
	\item Implementing RBMs as flexible trial wave function guesses.
	\item Study ground state properties of quantum dots like energy, variance and electron densities.
	\item Make a critical evaluation of the RBMs compared with VMC studies.
\end{itemize}

We believe that machine learning based methods is the next big thing in many-body quantum mechanics, and our contribution to the field is therefore to investigate one of the many such approaches. There are also some very interesting similarities between the typical sampling process behind Boltzmann machines and the QMC methods, and by revealing the actual mechanisms, we might be able to develop more robust methods. 

\section{Our developed code}
There exist plenty commercial software programs for solving the quantum many-body problem, and they are often tremendously fast. In general it is wise to use already existing code and not try reinvent the wheel, but in our case we will investigate a new approach, which forces us to write the code from scratch. 

Our VMC solver is written in object-orientated C++ inspired by Morten Ledum's example implementation \cite{ledum_simple_2016}, but our basis set is assumed to be given in Cartesian coordinates. The goal is not to compete with the performance of commercial software, but we will still put in significant effort to make the code fast. As far it is possible, all operations are vectorized using the open source template library Eigen, which again is based on the tremendously fast packages BLAS and LAPACK. We used the profiling tool Valgrind to analyze which functions that the program spends most of the time on. For implementation details, see chapter \ref{chp:WFE} and \ref{chp:rbmimplementation}. The VMC code can in its entirety be found on \url{https://github.com/evenmn} under MIT license. 

\section{Structure of the thesis}
Fundamental theory, including many-body quantum physics and basic machine learning, is given in chapter \ref{chp:quantum}-\ref{chp:systems}. The theory behind the RBMs and VMC is given in chapter \ref{chp:restricted}-\ref{chp:methods} as our advanced theory, and their optimization schemes and implementation is presented in chapter \ref{chp:WFE}-\ref{chp:rbmimplementation}. The remaining chapters include presentation and discussion of the results and conclusion.
