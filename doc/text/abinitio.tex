\chapter{Ab Initio Quantum Methods}
The term \textit{ab initio} means from first principles, implying that only physical constants are put into the methods. The Monte-Carlo methods are not considered \textit{ab initio} as nonphysical hyper parameters are required. Some example methods are Hartree-Fock methods and post-Hartree-Fock methods like Configuration Interaction and Coupled Cluster, which will be discussed in the following. 

\section{Hartree-Fock} \label{sec:hf}
Describe this detailed

\section{Configuration Interaction} \label{subsec:ci}
The completeness relation

\section{Coupled Cluster} \label{subsec:cc}
The coupled cluster method is the \textit{de facto} standard wave function-based method for electronic structure calculations. \cite{paldus} The method approximates the wave function with an exponential expansion, 
\begin{equation}
\ket{\Psi_{\text{CC}}}=e^{\hat{T}}\ket{\Phi_0}
\end{equation}
where $\hatT$ is the cluster operator, entirely given by $\hatT=\hatT_1+\hatT_2 +\hatT_3+\hdots$ with
\begin{equation}
\hatT_n = \left( \frac{1}{n!}\right)^2 \sum_{abc...} \sum_{ijk...} t_{ijk...}^{abc...}a_a^\dagger a_b^\dagger a_c^\dagger \cdots a_k a_j a_i.
\end{equation}
We again want to solve the Schrödinger equation,
\begin{equation}
\hat{H}\ket{\Psi}=\hat{H}e^{\hatT}\ket{\Phi_0}=\epsilon e^{\hatT}\ket{\Phi_0},
\end{equation}
which can be simplified by multiplying with $e^{-\hatT}$ from the left. This introduces us to the \textbf{similarity transformed Hamiltonian} 
\begin{equation}
\bar{H}=e^{-\hatT}\hat{H}e^{\hatT}.
\end{equation}
If we on one hand now multiply with the reference bra on the left hand side, we easily observe that
\begin{equation}
\mel{\Phi_0}{\bar{H}}{\Phi_0}=\epsilon
\end{equation}
which is the coupled cluster energy equation. On the other hand, we can multiply with an excited bra on left hand side, and find that
\begin{equation}
\mel{\Phi_{ijk\hdots}^{abc\hdots}}{\bar{H}}{\Phi_0}=0
\end{equation}
which are the coupled cluster amplitude equations. The similarity transformed Hamiltonian can be rewritten using the Baker-Campbell-Hausdorff expansion
\begin{align}
\label{eq:BCH}
\bar{H} = \hat{H} &+ [\hat{H},\hat{T}]\notag \\
&+ \frac{1}{2}[[\hat{H},\hat{T}],\hat{T}]\notag \\
&+ \frac{1}{6}[[[\hat{H},\hat{T}],\hat{T}],\hat{T}] \\
&+ \frac{1}{24}[[[[\hat{H},\hat{T}],\hat{T}],\hat{T}],\hat{T}] \notag \\
&+ \cdots \notag
\end{align}
and we are in principle set to solve the amplitude equations with respect to the amplitudes $t_{ijk\hdots}^{abc\hdots}$ and then find the energy. The expansion is able to reproduce the true wave function exactly using a satisfying number of terms and an infinite basis. This is, of course, not possible, but even by limiting us to the first few coupled cluster operators, the results are often good compared to other methods. \cite{crawford}