\chapter{General Gaussian-binary RBM wave function} \label{sec:appendixd}
\section{Derive the wave function}\label{sec:derive}
We have seen that the probability of having a set of positions $\bs{x}$ with a set of hidden nodes $\bs{h}$ is given by
\begin{equation}
F(\bs{x},\bs{h})=\frac{1}{Z}\exp(-\beta E(\bs{x},\bs{h}))
\end{equation}
where we set $\beta=1/kT=1$, $Z$ is the partition function and $E(\bs{x},\bs{h})$ is the system energy
\begin{equation}
E(\bs{x},\bs{h})=\sum_{i=1}^{F}\frac{(x_i-a_i)^2}{2\sigma_i^2}-\sum_{j=1}^Hb_jh_j-\sum_{i,j=1}^{F,H}\frac{x_iw_{ij}h_j}{\sigma_i^2}
\end{equation}
such that
\begin{equation}
F_{\text{RBM}}(\boldsymbol{x},\boldsymbol{h})=\exp(\sum_{i=1}^F \frac{(x_i - a_i)^2}{2\sigma^2}) \exp(\sum_{j=1}^H\Big(b_jh_j+\sum_{i=1}^F\frac{x_iw_{ij}}{\sigma^2}\Big)).
\end{equation}
We omit the partition function because it will not affect the results (it is just a normalization constant). The probability of a set of positions only is therefore the sum over all sets of $\boldsymbol{h}$, $\{\boldsymbol{h}\}$:
\begin{align}
F_{\text{RBM}}(\boldsymbol{x})&=\sum_{\{\boldsymbol{h}\}}\exp\Big(\sum_{i=1}^F \frac{(x_i - a_i)^2}{2\sigma^2}\Big) \prod_{j=1}^H\exp\Big(b_jh_j+\sum_{i=1}^F\frac{x_iw_{ij}h_j}{\sigma^2}\Big)\notag\\
&=\sum_{h_1}\sum_{h_2}\hdots\sum_{h_N}\exp\Big(\sum_{i=1}^F \frac{(x_i - a_i)^2}{2\sigma^2}\Big)\exp\Big(b_1h_1+\sum_{i=1}^F\frac{x_iw_{i1}h_1}{\sigma^2}\Big)\cross\notag\\
&\phantom{=}\exp\Big(b_2h_2+\sum_{i=1}^F\frac{x_iw_{i2}h_2}{\sigma^2}\Big)\hdots \exp\Big(b_Hh_H+\sum_{i=1}^F\frac{x_iw_{iH}h_H}{\sigma^2}\Big)\notag\\
&=\exp\Big(\sum_{i=1}^F \frac{(x_i - a_i)^2}{2\sigma^2}\Big)\prod_{j=1}^H\sum_{h_j=0}^1\exp\Big(b_jh_j+\sum_{i=1}^F \frac{x_i w_{ij}h_j}{\sigma^2}\Big)\notag\\
&=\exp\Big(\sum_{i=1}^F \frac{(x_i - a_i)^2}{2\sigma^2}\Big) \prod_{j=1}^H \bigg[1+ \exp\Big(b_j + \frac{\boldsymbol{x}^T\boldsymbol{w}_{*j}}{\sigma^2}\Big)\bigg]
\end{align}

\section{Find derivatives} \label{sec:derivatives}
A general Gaussian-binary restricted Boltzmann machine has the form
\begin{equation}
\psi(\bs{x};\bs{a},\bs{\theta})=\exp(-\sum_{i=1}^{ND}\frac{(x_i-a_i)^2}{2\sigma^2})\prod_{j=1}^H\Big[1+\exp(f_j(\bs{x};\bs{\theta}))\Big]
\end{equation}
where $f_j(\bs{x};\bs{\theta})$ is an arbitrary function of the coordinates $\bs{x}$ and the weights $\bs{\theta}$. The Gaussian part is straight-forward to differentiate, so we will keep our attention on the product,
\begin{equation}
\psi_{\text{p}}(\bs{x};\bs{\theta})=\prod_{j=1}^H\Big[1+\exp(f_j(\bs{x};\bs{\theta}))\Big].
\end{equation}
Henceforth, we will omit the variable $\bs{x}$ and $\bs{\theta}$. By introducing the functions
\begin{equation}
p_j\equiv \frac{1}{1+\exp(+f_j)}\quad\wedge\quad n_j\equiv \frac{1}{1+\exp(-f_j)},
\end{equation}
where the last one is the sigmoid function, we find the gradient and Laplacian of $\ln\psi_{\text{prod}}$ to be
\begin{equation}
\nabla_k\ln\psi_{\text{p}}=\sum_{j=1}^Hn_j\nabla_k(f_j)
\end{equation}
and
\begin{equation}
\nabla_k^2\ln\psi_{\text{p}}=\sum_{j=1}^Hn_j\big[\nabla_k^2(f_j)+p_j\big(\nabla_k(f_j)\big)^2\big]
\end{equation}
respectively. The parameter $\theta_i$ can be updated according to the following rule
\begin{equation}
\frac{\partial}{\partial \theta_i}\ln \psi_{\text{p}}=\sum_{j=1}^Hn_j\frac{\partial}{\partial\theta_i}(f_j).
\end{equation}
and the ratio between wave functions can be found by
\begin{equation}
\frac{\psi_{\text{p}}^{\text{new}}}{\psi_{\text{p}}^{\text{old}}}=\prod_{j=1}^H\frac{p_j^{\text{old}}}{p_j^{\text{new}}}.
\end{equation}
As a conclusion, what we actually need to calculate to find respective expressions for each wave function is $\nabla_k(f_j)$, $\nabla_k^2(f_j)$ and $\partial_{\alpha_i}(f_j)$.This applies for a general Gaussian-binary restricted Boltzmann machine, also for a deep Boltzmann machine if all the units are Gaussian-binary.