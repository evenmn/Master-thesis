\chapter{Implementation} \label{sec:implementation}
During the implementations we had two main areas of focus
\begin{itemize}
	\item Efficient code
	\item High legibility, tidy code
\end{itemize}

\section{Foundation} \label{subsec:foundation}
\section{Structure} \label{subsec:structure}
Add structure chart
\begin{figure} [H]
	\centering
	\begin{tikzpicture}[
	>={Latex[width=2mm,length=2mm]},
	base/.style = {rectangle, rounded corners, draw=black,
		minimum width=2cm, minimum height=0.5cm, text
		centered, font=\sffamily},
	basecode/.style = {rectangle, rounded corners, draw=black,
		minimum width=2cm, minimum height=0.5cm, text
		centered, font=\sffamily, align=left},
	activityStarts/.style = {base, fill=blue!30, drop shadow},
	startstop/.style = {base, fill=red!25, drop shadow},
	startstopcode/.style = {basecode, fill=Red!25, drop shadow},
	activityRuns/.style = {base, fill=green!25, drop shadow},
	process/.style = {base, fill=white!15, font=\sffamily, drop shadow},
	processcode/.style = {basecode, fill=white!15, font=\sffamily, drop shadow},
	scale=0.8, 
	node distance=1.5cm, 
	every node/.style={fill=white, font=\sffamily},
	align=center]
	\node (pre) [activityStarts] {
		Pre-build two-body matrix $D_{pqrs}$ \\
		Pre-build one-body matrix $h_{pq}$ \\
		Pre-build overlap matrix $S_{pq}$ \\
		Initialize $\blds{C}$ 
	};
	\node (Fock) [activityRuns, below of=pre, yshift=-1.2cm] {
		Set Fock-matrix $\blds{F}$
	};
	\node (Fockmix) [activityRuns, below of=Fock, yshift=-0.7cm] {
		Perform mixing of $\blds{F}$ \\
		(Optional)
	};
	\node (Fockmixeq) [process, right of=Fockmix, xshift=2.4cm] {
		\ref{eq:extrapolF}
	};
	\draw[->] (Fock) -- (Fockmix);
	\draw[-] (Fockmix) -- (Fockmixeq);
	\node (RHF) [process, above right of=Fock, xshift=3.0cm, yshift=-0.2cm] {
		Use \ref{eq:FockRestrictedDef} if RHF
	};
	\node (UHF) [process, below right of=Fock, xshift=3.0cm, yshift=0.1cm] {
		Use \ref{eq:UHFFockdef} if UHF
	};
	\draw[-] (Fock) to [out=15, in=180] (RHF);
	\draw[-] (Fock) to [out=-15, in=180] (UHF);
	\draw[->] (pre) -- (Fock);
	\node (eigcomp) [activityRuns, below of=Fock, yshift=-2.5cm] {
		Solve generalized \\ 
		eigenvalue problem. \\
		$\blds{F}\blds{C} = \blds{\varepsilon}\blds{S}\blds{C}$
	};
	\draw[->] (Fockmix) -- (eigcomp);
	\node (density) [activityRuns, below of=eigcomp] {
		Set density matrix $\blds{\rho}$.
	};
	\draw[->] (eigcomp) -- (density);
	\node (densityeq) [process, right of=density, xshift=2.5cm] {
		\ref{eq:densitymatrixdef}
	};
	\draw[-] (density) -- (densityeq);
	\node (mix) [activityRuns, below of=density] {
		Perform mixing of $\blds{\rho}$. \\
		(Optional)
	};
	\draw[->] (density) -- (mix);
	\node (mixeq) [process, right of=mix, xshift=2.5cm] {
		\ref{eq:mixing}
	};
	\draw[-] (mix) -- (mixeq);
	\node (conv) [startstop, below of=mix] {
		Check convergence.
	};
	\node (conveq) [process, right of=conv, xshift=2.5cm] {
		\ref{eq:HFconvthresh}
	};
	\draw[->] (mix) -- (conv);
	\draw[-] (conv) -- (conveq);
	\node (no) [above left of=conv, xshift=-4.0cm] {
		$\blds{\varepsilon}^{\text{HF}}_{\text{new}} -
		\blds{\varepsilon}^{\text{HF}}_{\text{old}} \geq \epsilon$
	};
	\draw[->] (conv) to [out=180,in=-90] (no);
	\node (keepOld) [process, above of=no, yshift=1.5cm, xshift=-0.5cm] {
		Save energy eigenvalues and \\
		density matrix(mixed). \\
		$\blds{\varepsilon}^{\text{HF}}_{\text{old}} =
		\blds{\varepsilon}^{\text{HF}}_{\text{new}}$ \\
		$\blds{\rho}_{\text{old}} = \blds{\rho}_{\text{new}}$
	};
	\draw[->] (no) to [out=90,in=-90] (keepOld);
	\draw[->] (keepOld) to [out=90,in=180] (Fock);
	\node (yes) [below of=conv] {
		$\blds{\varepsilon}^{\text{HF}}_{\text{new}} -
		\blds{\varepsilon}^{\text{HF}}_{\text{old}} < \epsilon$
	};
	\draw[->] (conv) -- (yes);
	\node (end) [activityStarts, below of=yes] {
		Output energy
	};
	\node (endeq) [process, right of=end, xshift=3.0cm] {
		\ref{eq:energyJKHF, eq:ch3Efunc}
	};
	\draw[->] (yes) -- (end);
	\draw[-] (end) -- (endeq);
	\end{tikzpicture}
	\caption{Hartree-Fock Algorithm.}
\end{figure}
\section{Optimization algorithms} \label{subsec:optimization}
\subsection{Stochastic Gradient Descent} \label{subsubsec:sgd}
\subsection{ADAM} \label{subsubsec:adam}