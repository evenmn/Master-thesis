\chapter{Quantum Monte-Carlo Methods} \label{chp:methods}
Some great methods are developed over the past decade. We will focus on the Hartree-Fock and variational Monte Carlo methods and give a detailed explanation of those methods. Additionally, the well-known methods configuration interaction and coupled cluster will be described briefly for some kind of completeness. 

Monte Carlo methods in quantum mechanics are a bunch of methods that are built on diffusion processes, and includes Variational Monte Carlo (VMC), Diffusion Monte Carlo (DMC) and others. The common denominator is that we move particles in order to find the optimal configuration, usually where the energy is minimized. The particles can be moved isotropic, i.e., uniformly in all directions, or they can be affected by a drift force which makes the process anisotropic. 

\subsection{Isotropic processes}
In isotropic processes, we have random walks where the particles move randomly in space which falls under the category Markov chains. If we assume constant timestep, the step index can be considered the time, thus we have a time-dependent probability density $P(x,t)$. This probability density needs to satisfy the isotropic diffusion equation,
\begin{equation}
\frac{\partial P(\bs{x},t)}{\partial t}=D\frac{\partial^2}{\partial \bs{x}^2}P(\bs{x},t).
\end{equation}

\subsection{Anisotropic processes}
For anisotropic processes, we have a drift and the moves are no longer considered random but falls still under the category Markov chains. Because of the drift, we need to rewrite the diffusion equation and we end up with the Fokker-Planck equation,
\begin{equation}
\frac{\partial P(\bs{x},t)}{\partial t}=D\frac{\partial}{\partial \bs{x}}\left(\frac{\partial}{\partial \bs{x}}-F\right)P(\bs{x},t)
\end{equation}
which needs to be satisfied. The new positions in coordinate space are given as solution of the Langevin equation 
\begin{equation}
\frac{\partial \bs{x}(t)}{\partial t}=D\bs{F}(\bs{x}(t))+\eta
\end{equation}

\subsection{Variational Monte Carlo} \label{subsec:vmc}
The Variational Monte Carlo (hereafter, VMC) method is today widely used when it comes to the study of ground state properties of quantum mechanical systems. It is a Monte Carlo method which makes use of Metropolis sampling, and has been used in studies of fermionic systems since the 1970's. \cite{deb2014} If we go back to the variational principle in equation \eqref{eq:variationalprinciple}, we see that by choosing a wave function which satisfies the criteria, we will get an energy larger or equal to the ground state energy. \bigskip

There are two main problems we need to solve
\begin{enumerate}
	\item We seldomly know the correct wave function
	\item The integral we need to find the energy is hard or impossible to solve
\end{enumerate}
Let us take the last problem first. The solution is to approximate the integral with a sum,
\begin{align}
E &\leq \frac{\int\Psi_T(\bs{r})^*\hat{H}\Psi_T(\bs{r}) d\bs{r}}{\int\Psi_T(\bs{r})^*\Psi_T(\bs{r}) d\bs{r}} \notag\\
& = \int P(\bs{r}) E_L(\bs{r}) d\bs{r} \notag\\
& \approx \frac{1}{N}\sum_{i=1}^NE_L(\bs{r}_i) \label{eq:energysum}
\end{align}
where the local energy is defined as
\begin{equation}
E_L(\bs{r})\equiv\frac{1}{\Psi_T(\bs{r})}\hat{H}\Psi_T(\bs{r})
\label{eq:local energy}
\end{equation}
and the $\bs{r}_i$ is withdrawn from the probability distribution $P(\bs{r})$, which is given by
\begin{equation}
P(\bs{r})=\frac{|\Psi_T(\bs{r})|^2}{\int|\Psi_T(\bs{r})|^2d\bs{r}}.
\end{equation}
The energy found from equation \eqref{eq:energysum} is an expectation value, and we therefore know that the true energy lies within the standard error. When increasing the number of Monte-Carlo cycles, the standard error decreases and we get a more accurate energy. In the limit $N$ goes to infinity, the variance should approach zero,
\begin{equation}
\langle E\rangle=\lim_{N\to\infty} \frac{1}{N}\sum_{i=1}^NE_L(\bs{r}_i).
\end{equation}
For more statistical details, see \cite{deb2014}. 

Given a wave function we are now able to estimate the corresponding energy, but how do we find a good wave function? In VMC, we define a wave function with variational parameters, which are adjusted in order to minimize the energy for every iteration. For every iteration, we run $N$ Monte-Carlo cycles where we withdraw a new position $\bs{r}_i$. Whether or not the proposed move should be accepted is determined by the Metropolis algorithm.

\subsubsection{The Metropolis Algorithm}
The genius of the metropolis algorithm, is that the acceptance of a move is not based on the probabilities themselves, but the ratio between the new and the old probability\\
. In that way, we avoid calculating the sum over all probabilities, which often is expensive or even impossible to calculate. 

In its simplest form, the move is proposed randomly, and it is accepted if the ratio is larger than a random number between 0 and 1. However, with this approach a lot of moves will be rejected, which wastes CPU time. A better method is \textbf{importance sampling}, which makes a educated guess of the best way to move based on diffusion processes, and move the particle in that direction. 

A time-dependent probability density needs to satisfy the Fokker-Planck equation 