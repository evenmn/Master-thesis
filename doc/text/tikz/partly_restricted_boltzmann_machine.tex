\begin{figure} [H]
	\centering
	\begin{tikzpicture}
	
	% Define visible units
	\node[input] (x2) {$x_2$};
	\node[input] at (1,+3/2) (x1) {$x_1$};
	\node[input] at (1,-3/2) (x3) {$x_3$};
	
	% Define hidden units
	\node[input] at (3,+3/2) (h1) {$h_1$};
	\node[input] at (4,0) (h2) {$h_2$};
	\node[input] at (3,-3/2) (h3) {$h_3$};
	
	% Define biases
	\node[input] at (0,7/2) (a) {$1$};
	\node[input] at (4,7/2) (b) {$1$};
	
	% Define paths
	\path[draw,-] (x1) -- (h1) node[midway,above] {$w_{11}$};
	\path[draw,-] (x1) -- (h2);
	\path[draw,-] (x1) -- (h3);
	
	\path[draw,-] (x2) -- (h1);
	\path[draw,-] (x2) -- (h2);
	\path[draw,-] (x2) -- (h3);
	
	\path[draw,-] (x3) -- (h1);
	\path[draw,-] (x3) -- (h2);
	\path[draw,-] (x3) -- (h3);
	
	\path[draw,red,-] (x1) -- (x2) node[midway, above left] {$c_{12}$};
	\path[draw,red,-] (x2) -- (x3);
	\path[draw,red,-] (x3) -- (x1);
	
	\draw[green,-] (b) to [out=270,in=90] (h1);
	\draw[green,-] (b) to [out=315,in=45] (h2);
	\draw[green,-] (b) to [out=0,in=0] (h3)  node[] at (6,1) {$b_3$};
	
	\draw[blue,-] (a) to [out=270,in=90] (x1);
	\draw[blue,-] (a) to [out=225,in=135] (x2);
	\draw[blue,-] (a) to [out=180,in=180] (x3) node[] at (-2,1) {$a_3$};
	
	
	% Add some text
	\node[below=1em of x3,font=\scriptsize] {visible};
	\node[below=1em of h3,font=\scriptsize] {hidden};
	%\node[below=5.8em of center,font=\scriptsize] {output};
	\end{tikzpicture}
	\caption{Partly restricted Boltzmann machine. Black lines are inter-layer connections, where for instance the line between $x_1$ and $h_1$ is related to the weight $w_{11}$. The blue lines are related to the input bias weights, and, for instance, the line going from the bias node to $x_3$ is related to $a_3$. Similarly, the green lines are related to the hidden nodes bias weights, and, for instance, the line going from the bias node to $h_3$ is related to $b_3$. Finally, the red lines are the intra-layer connections related to the intra-layer weights. The weight between node $x_1$ and $x_2$ is called $c_{12}$. }
	\label{fig:partly_restricted_boltzmann_machine}
\end{figure}