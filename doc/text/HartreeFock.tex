\chapter{The Hartree-Fock Method} \label{chp:hartreefock}
\epigraph{Hartree-Fock is like a good kitchen tool; it can be used to prepare several different dishes.}{Morten Hjorth-Jensen}
\begin{figure}[H]
	\centering
	\includegraphics[scale=0.4]{Images/example.png}
	\caption{Caption}
\end{figure}
Hartree-Fock is an algorithm for finding an approximative expression for the ground state energy of a Hamiltonian. Ever since the childhood of many-body quantum mechanics, it has been an important quantum many-body method due to its accuracy contra expensiveness. 

The method sets up a mean-field potential, often called the Hartree-Fock potential, which replaces the two-body repulsive interaction found in electronic structure calculations. Thereafter, a new basis is expanded in a given basis, and the optimal coefficients with respect to the mean-field constraint are generated. Traditionally, the method has been called the self-consistent field method, but today Hartree-Fock is mostly used. 

Since we are replacing the interaction term, we need to define a new operator to replace the Hamiltonian, known as the Hartree-Fock operator,
\begin{equation}
\hat{h}^{\text{HF}}=\hat{t} + \hat{u}_{\text{ext}} + \hat{u}^{\text{HF}}.
\end{equation}
The $\hat{u}^{\text{HF}}$ is a single-particle potential, and will later be determined by the algorithm. The operator can be used directly in the Schrödinger equation 
\begin{equation}
\hat{h}^{\text{HF}}\ket{p}=\epsilon_{\alpha}\ket{p}
\end{equation}
where the obtained energy is called the Hartree-Fock energy. The Hartree-Fock basis is found from the following expansion,
\begin{equation}
\ket{p}=\sum_{\lambda}C_{p\lambda}\ket{\lambda},
\end{equation}
where the initial basis is denoted with Greek letters and the new basis is denoted with Roman letters. $C_{p\lambda}$ are the coefficients that form an optimal basis. 

\section{Restricted Hartree-Fock}
When we calculated the reference energy above, our basis contained only one Slater determinant, more specifically the ground state. In Hartree-Fock, we still have a single Slater determinant basis, but we now construct new SPFs with the constraint of minimizing the energy. 

In general, one can change from one single-particle basis to another by a unitary transform,
\begin{equation}
\ket{p}=\sum_{\alpha}c_{p\alpha}\ket{\alpha},
\end{equation}
where we use greek letters for the old basis and roman letters for the new one. If we then insert into \eqref{eq:c_H_c}, we get a find energy formula with coefficients, $C_{p\lambda}$, that we can vary
\begin{equation}
E=\sum_{p}^N\sum_{\alpha\beta}C_{p\alpha}^*C_{p\beta}\mel{\alpha}{\hat{h}_0}{\beta}+\frac{1}{2}
\sum_{pq}^N\sum_{\alpha\beta\gamma\delta}C_{p\alpha}^*C_{q\beta}^*C_{p\gamma}C_{q\delta}\mel{\alpha\beta}{\hat{v}}{\gamma\delta}_{\text{AS}}.
\end{equation}
Further, we assume that also our new basis is orthonormal, i.e,
\begin{align}
	\braket{p}{q}&=\sum_{\alpha}c_{p\alpha}^*c_{q\alpha}\braket{\alpha}{\alpha}=\sum_{\alpha}c_{p\alpha}^*c_{q\alpha}=\delta_{pq}\\
	&\Rightarrow\sum_{\alpha}c_{p\alpha}^*c_{q\alpha}-\delta_{pq}=0\label{eq:constraint}
\end{align}

We now have a function, $E$, that we want to minimize with respect to a constraint given in equation \eqref{eq:constraint}. This is a typical situation where Lagrange Multipliers is convenient to use, which in this case can be written as
\begin{equation}
{\cal L}(\{C_{p\alpha}\})=E(\{C_{p\alpha}\})-\sum_a\varepsilon_a\Big(\sum_{\alpha}c_{p\alpha}^*c_{q\alpha}-\delta_{pq}\Big).
\end{equation}
The variation in reference energy is then find to be
\begin{equation}
\delta E=\sum_{k\alpha}\frac{\partial E}{\partial C_{k\alpha}^*}\delta C_{k\alpha}^*+\sum_{k\alpha}\frac{\partial E}{\partial C_{k\alpha}}\delta C_{k\alpha}-\sum_{k\alpha}\varepsilon_k(C_{k\alpha}\delta C_{k\alpha}^*+C_{k\alpha}^*\delta C_{k\alpha})
\label{eq:E_hf}
\end{equation}
which is zero when $E$ is minimized. Each coefficient $C_{k\alpha}$ and $C_{k\alpha}^*$ is independent, so they can be varied independently. Thus
\begin{equation}
\bigg(\frac{\partial E}{\partial C_{k\alpha}^*}-\varepsilon_kC_{k\alpha}\bigg)\delta C_{k\alpha}^*=0,
\end{equation}
which is satisfied if and only if
\begin{equation}
\frac{\partial E}{\partial C_{k\alpha}^*}-\varepsilon_kC_{k\alpha}=0\qquad\forall\,\,\, k,\alpha
\end{equation}
The first term can be derived from \eqref{eq:E_hf}, and reads
\begin{equation}
\frac{\partial E}{\partial C_{k\alpha}^*}=\sum_{\beta}C_{k\beta}\mel{\alpha}{\hat{h}_0}{\beta}+\sum_p^N\sum_{\beta\gamma\delta}C_{p\beta}^*C_{k\gamma}C_{p\delta}\mel{\alpha\beta}{\hat{v}}{\gamma\delta}_{\text{AS}}.
\end{equation}
This results in the equation
\begin{equation}
\sum_{\gamma}\hat{h}_{\alpha\gamma}^{\text{HF}}C_{k\gamma}=\varepsilon_kC_{k\gamma}
\label{eq:HF_sum}
\end{equation}
where we have defined
\begin{equation}
\hat{h}_{\alpha\gamma}^{\text{HF}}\equiv\mel{\alpha}{\hat{h}_0}{\gamma}+\sum_p^N\sum_{\beta\delta}C_{p\beta}^*C_{p\delta}\mel{\alpha\beta}{\hat{v}}{\gamma\delta}_{\text{AS}}.
\end{equation}
We recognize that \eqref{eq:HF_sum} can be written as a matrix-vector product
\begin{equation}
\hat{h}^{\text{HF}}C_k=\varepsilon_k^{\text{HF}}C_k
\end{equation}
where $C_k$ are columns in our coefficient matrix and $\varepsilon_k^{\text{HF}}$ are just the eigenvalues of $\hat{h}^{\text{HF}}$, they have no physical significance. We will use this equation to find the optimal SPFs (optimal $C_k$'s) and then find the energy from equation \eqref{eq:E_hf}.
\begin{equation}
\hat{h}^{\text{HF}}(C_k^{i+1})C_k^i=\varepsilon_k^{\text{HF}}C_k^i
\label{eq:HF_iter}
\end{equation}
Usually one initialize this with $\hat{C}=\hat{{\cal I}}$, the identity matrix.

\section{Unrestricted Hartree-Fock}
