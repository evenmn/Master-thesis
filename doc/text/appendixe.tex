\chapter{Wave Function Elements} \label{chp:appendixe}

\section{Kinetic Energy Calculations}
The local energy, defined in equation \eqref{eq:local energy}, is
\begin{align}
E_L &=\frac{1}{\Psi_T}\hatH\Psi_T\\
&=\sum_{k=1}^M\Big[-\frac{1}{2\Psi_T}\nabla_k^2\Psi_T + U_k + V_k\Big].
\end{align}
The first term, which is the kinetic energy term, is the only wave function-dependent one. It will in this appendix be evaluated for various wave function elements. From the definition of differentiation of a logarithm, we have that
\begin{equation}
\frac{1}{\Psi_T}\nabla_k\Psi_T=\nabla_k\ln\Psi_T,
\end{equation}
which provides the following useful relation 
\begin{equation}
\frac{1}{\Psi_T}\nabla_k^2\Psi_T=\nabla_k^2\ln\Psi_T + (\nabla_k\ln\Psi_T)^2.
\end{equation}
Consider a trial wave function, $\Psi_T$, consisting of a product of $p$ wave function elements, $\{\phi_1, \phi_2\hdots\phi_p\}$,
\begin{equation}
\Psi_T = \prod_{i=1}^p\phi_i.
\end{equation}
The kinetic energy related to this trial wave function is then computed by
\begin{equation}
\frac{1}{\Psi_T}\nabla_k^2\Psi_T=\sum_{i=1}^p\nabla_k^2\ln\phi_i + \Big(\sum_{i=1}^p\nabla_k\ln\phi_i\Big)^2,
\end{equation}
which can be computed given all local derivatives $\nabla_k^2\ln\phi_i$ and $\nabla_k\ln\phi_i$. For each wave function element given below, those local derivatives will be evaluated. In addition, we need to know the derivative of the local energy with respect to the variational parameters in order to update the parameters correctly. 

\section{Parameter Update}
In gradient based optimization methods, as we use, one needs to know the gradient of the local energy with respect to all variational parameters $\alpha_i$,
\begin{equation}
\partial_{\alpha_i} E_L\equiv\frac{\partial E_L(\alpha_i)}{\partial \alpha_i}.
\end{equation}
If we assume that each parameter, $\alpha$, only exists in a wave function element,
\begin{equation}
\Psi_T(\alpha)=\phi_1(\alpha)\prod_{i=2}^{p}\phi_i
\end{equation}
the derivative of the entire local energy is reduced to the derivative of the kinetic energy term given the wave function element,
\begin{align}
\partial_{\alpha} E_L &=-\frac{1}{2}\partial_{\alpha}\bigg(\nabla_k^2\ln\phi_1(\alpha)+\sum_{i=2}^p\nabla_k^2\ln\phi_i + \Big(\nabla_k\ln\phi_1(\alpha)+\sum_{i=2}^p\nabla_k\ln\phi_i\Big)^2\bigg)\\
&=-\frac{1}{2}\partial_{\alpha}\nabla_k^2\ln\phi_1(\alpha)-\Big(\sum_{i=1}^p\nabla_k\ln\phi_i\Big)\cdot\partial_{\alpha}\nabla_k\ln\phi_1(\alpha).
\end{align}
The sum is already evaluated in the kinetic energy calculations, which means that to calculate the gradients, it is sufficient to calculate $\partial_{\alpha}\nabla_k\ln\phi_1(\alpha)$ and $\partial_{\alpha}\nabla_k^2\ln\phi_1(\alpha)$ for all wave function elements with respect to their variational parameters.

\section{Derivatives}
\subsection{Simple Gaussian}
A natural starting point is the Gaussian function, since it appears in all harmonic oscillator calculations. For $N$ particles in $D$ dimensions, the function is given by
\begin{equation}
\Psi(\alpha, \bs{r})=\exp\Big[-\frac{1}{2}\alpha\sum_{i=1}^Nr_i^2\Big]
\end{equation}
where the derivative with respect to coordinate $r_k$ is
\begin{equation}
\nabla_k\ln\Psi(\alpha)=-\alpha r_k
\end{equation}
and the second derivative is
\begin{equation}
\nabla_k^2\ln\Psi(\alpha)=-\alpha ND.
\end{equation}
The gradients for those derivatives are
\begin{equation}
\partial_{\alpha} \nabla_k\ln\Psi(\alpha)=-r_k
\end{equation}
and
\begin{equation}
\partial_{\alpha} \nabla_k^2\ln\Psi(\alpha)=-ND
\end{equation}
respectively.

\subsection{Padé-Jastrow Factor}
The Padé-Jastrow factor is introduced in order to take care of the correlations. It is specified in equation \eqref{eq:PadeJastrow}, 
\begin{equation}
J(\bs{r}; \beta, \gamma) = \exp\bigg(\frac{1}{2}\sum_{i=1}^N\sum_{j=1}^N\frac{\beta_{ij}r_{ij}}{1+\gamma r_{ij}}\bigg),
\end{equation}
where $N$ is the number of particles. One challenge is that we operate in cartesian coordinates, while the expressed Jastrow factor obviously is easier to handle in spherical coordinates. Since we need to differentiate this with respect to all free dimensions, we need to be careful not confuse the 
which gives the first and second derivatives
\begin{equation}
\nabla_k\ln J=\sum_{j\neq k=1}^N\frac{\beta_{kj}}{(1+\gamma r_{kj})^2}\frac{x_k-x_j}{r_{kj}}
\end{equation}
and
\begin{equation}
\nabla_k^2\ln J=\sum_{j\neq k=1}^N\frac{\beta_{kj}}{(1+\gamma r_{kj})^2}\bigg[1-\Big(1+2\frac{\gamma r_{kj}}{1+\gamma r_{kj}}\Big)\frac{(x_k-x_j)^2}{r_{kj}^2}\bigg]\frac{1}{r_{kj}}
\end{equation}
respectively. 

The derivative of those again with respect to $\gamma$ are
\begin{equation}
\partial_{\gamma}\nabla_k\ln J = -2 \sum_{j\neq k=1}^N\frac{\beta_{kj}}{(1+\gamma r_{kj})^3}(x_k-x_j)
\end{equation}
and
\begin{equation}
\partial_{\gamma}\nabla_k^2\ln J = -2 \sum_{j\neq k=1}^N\frac{\beta_{kj}}{(1+\gamma r_{kj})^3}\bigg[1-4\frac{\gamma r_{kj}}{1+\gamma r_{kj}}\frac{(x_k-x_j)^2}{r_{kj}^2}\bigg]
\end{equation}

By defining 
\begin{equation}
f_{ij}=\frac{1}{1+\gamma r_{ij}}\quad g_{ij}=\frac{x_i-x_j}{r_{ij}}\quad h_{ij}=\frac{r_{ij}}{1+\gamma r_{ij}}
\end{equation}
the equations can be written as
\begin{empheq}[box={\mybluebox[5pt]}]{align}
\frac{J_{\text{new}}^2}{J_{\text{old}}^2}&=\exp\Big(2\sum_{j=1}^N\beta_{kj}(h_{kj}^{\text{new}}-h_{kj}^{\text{old}})\Big)\\
\nabla_k\ln J &=\sum_{j\neq k=1}^N\beta_{kj}\cdot f_{kj}^2\cdot g_{kj}\\
\nabla_k^2\ln J &= \sum_{j\neq k=1}^N\frac{\beta_{kj}}{r_{kj}}f_{kj}^2\Big[1-(1+2\gamma h_{kj})g_{kj}^2\Big]\\
\partial_{\gamma}\nabla_k\ln J &=\sum_{j\neq k=1}^N\beta_{kj}\cdot f_{kj}^3(x_k-x_j)\\
\partial_{\gamma}\nabla_k^2\ln J &= \sum_{j\neq k=1}^N \beta_{kj}\cdot f_{kj}^3\Big[1=4\gamma h_{kj}\cdot g_{kj}^2\Big].
\end{empheq}

\subsection{Slater Determinant}
We need to introduce the Slater Determinant to study more fermions. 

\subsection{NQS-Gaussian}
Now over to the real deal; the machine learning inspired wave function elements. 

\subsection{NQS-Jastrow Factor}
\begin{equation}
J=\prod_{j=1}^N\bigg[1+\exp\Big(b_j+\sum_{i=1}^M\frac{W_{ij}x_i}{\sigma^2}\Big)\bigg]
\end{equation}

\begin{equation}
\nabla \ln J=\sum_{j=1}^N\frac{W_{kj}}{\sigma^2}\frac{\exp(b_j+\sum_{i=1}^M\frac{W_{ij}x_i}{\sigma^2})}{1+\exp(b_j+\sum_{i=1}^M\frac{W_{ij}x_i}{\sigma^2})}
\end{equation}



\subsection{Partly Restricted Element}

\subsection{Hydrogen-Like Orbitals}
\begin{equation}
\Psi(\alpha, \bs{r})=\exp\Big[-\frac{1}{2}\alpha\sum_{i=1}^Nr_i\Big]
\end{equation}
where the derivative with respect to coordinate $r_k$ is
\begin{equation}
\nabla_k\ln\Psi(\alpha)=-\alpha
\end{equation}
and the second derivative is
\begin{equation}
\nabla_k^2\ln\Psi(\alpha)=0
\end{equation}
The gradients for those derivatives are
\begin{equation}
\partial_{\alpha} \nabla_k\ln\Psi(\alpha)=-1
\end{equation}
and
\begin{equation}
\partial_{\alpha} \nabla_k^2\ln\Psi(\alpha)=0
\end{equation}
respectively.  