\chapter{Natural Units} \label{app:units}
In everyday life, we usually stick to the standard SI units when measuring or expressing distances, energies, weights, time et centra. A standard unit system is important because it is common for people around the world so that people can communicate with each other conveniently across borders with a reduced risk of misconceptions. Another point is that people gradually develop an intuition about a set of units when they are frequency used, such that they immediately observe if a number makes sense or not when it is expressed in the preferred units. As a European, I always measure length in meters, so when people say they are 1.80m tall, I can easily imagine their height. On the other hand, when Americans tell their height, I do not have an intuition for what 5.0 feet is. 

However, in science, the SI units are often not the preferred ones, especially when things get very large or very small. For instance, measuring cosmological distances in meters is very unpractical, as the distance to the Sun is $\sim1.5\cdot10^{11}$m and the distance to our closest neighbor galaxy Andromeda is $\sim 2.4\cdot10^{22}$m! Instead, we use units like the astronomical unit [a.u.] (should not be confused with atomic units) and light-years. 

For small scales, the situation is similar. For instance, the most probable distance between the nucleus and the electron is the Bohr radius, which is $a_0\approx5.3\cdot10^{-11}$m, which again is very unpractical to work with. Instead, we define so-called natural units where $a_0=1$, and scale the other quantities after that. We will in this chapter present how the quantum dot and atomic Hamiltonian can be scaled in natural units. We will stick to Hartree atomic units, as it gives elegant expressions of the Hamiltonians.  

\section{Quantum dots}
For circular quantum dots, the one-dimensional Hamiltonian in SI units reads
\begin{equation}
\hat{\mathcal{H}}=-\frac{\hbar^2}{2m}\frac{\partial^2}{\partial x^2}+\frac{1}{2}m\omega^2x^2
\label{eq:HamiltonianHO}
\end{equation}
with $\hbar$ as the reduced Planck's constant, $m$ as the electron mass and $\omega$ as the oscillator frequency. The corresponding wave functions read
\begin{equation}
\phi_n(x)=\frac{1}{\sqrt{2^nn!}}\cdot\bigg(\frac{m\omega}{\pi\hbar}\bigg)^{1/4}\exp\Big(-\frac{m\omega}{2\hbar}x^2\Big)H_n\Big(\sqrt{\frac{m\omega}{\hbar}}x\Big).
\end{equation}
We want to get rid of $\hbar$ and $m$ in equation \eqref{eq:HamiltonianHO} to make it dimensionless, which can be accomplished by scaling  $\hat{\mathcal{H}}'= \hat{\mathcal{H}}/\hbar$, such that the Hamiltonian reduces to
\begin{equation}
\hat{\mathcal{H}}'=-\frac{\hbar}{2m}\frac{\partial^2}{\partial x^2}+\frac{1}{2}\frac{m\omega^2}{\hbar}x^2.
\end{equation}
Moreover, we observe that the fraction $\hbar/m$ comes in both terms, which can be avoided by introducing a characteristic length $x'= x/\sqrt{\hbar/m}$. The final Hamiltonian is
\begin{empheq}[box={\mybluebox[5pt]}]{equation}
\hat{\mathcal{H}}=\frac{1}{2}\frac{\partial^2}{\partial x^2}+\frac{1}{2}\omega^2x^2
\end{empheq}
which corresponds to setting $\hbar=m=1$. In natural units, one often sets $\omega=1$ as well by scaling $\hat{\mathcal{H}}'=\hat{\mathcal{H}}/\hbar\omega$, but since we want to keep the $\omega$-dependency, we do it slightly different. This means that the exact wave functions for the one-particle one-dimensional case goes as
\begin{equation}
\phi_n(x)\propto\exp\Big(-\frac{\omega}{2}x^2\Big)H_n(\sqrt{\omega}x)
\end{equation}
where we omit the normalization factor with the Metropolis algorithm in mind. 

\section{Atoms}
The atomic Hamiltonian for an electron in subshell $l$ affected by a nucleus with atomic number $Z$ reads
\begin{equation}
\hat{\mathcal{H}}=-\frac{\hbar^2}{2m_e}\nabla^2-\frac{1}{4\pi\epsilon_0}\frac{Ze^2}{r}+\frac{\hbar^2l(l+1)}{2m_er^2}.
\label{eq:HamiltonianAtomic}
\end{equation}
in SI units where $e$ is the elementary charge and $k_e=1/4\pi\epsilon_0$ is Coulomb's constant. Again we want to get rid of the reduced Planck's constant $\hbar$ and the electron mass $m_e$ in order to make the Hamiltonian dimensionless, which we do by multiplying all terms by $(4\pi\epsilon_0)^2\hbar^2/m_ee^4Z^2$,
\begin{equation}
\hat{\mathcal{H}}\cdot\frac{(4\pi\epsilon_0)^2\hbar^2}{m_ee^4Z^2}=-\frac{(4\pi\epsilon_0)^2\hbar^4}{2m_e^2e^4Z^2}\nabla^2+\frac{4\pi\epsilon_0\hbar^2}{m_ee^2Zr}-\frac{(4\pi\epsilon_0)^2\hbar^4}{m_e^2e^4Z^2}\frac{l(l+1)}{r^2}.
\end{equation}
This might look very chaotic, but by exploiting that the Bohr radius,
\begin{equation}
a_0\equiv\frac{4\pi\epsilon_0\hbar^2}{m_ee^2Z},
\end{equation}
is found in all the right hand side terms, we can simplify the Hamiltonian to be
\begin{equation}
\hat{\mathcal{H}}\cdot\frac{(4\pi\epsilon_0)^2\hbar^2}{m_ee^4Z^2}=-\frac{a_0^2}{2}\nabla^2+a_0\frac{Z}{r}-a_0^2\frac{l(l+1)}{2r^2}.
\end{equation}
We obtain the dimensionless Hamiltonian in Atomic units by scaling  $r'=r/a_0$ and $\hat{\mathcal{H}}'=\hat{\mathcal{H}}/(m_ee^4Z^2/(4\pi\epsilon_0)^2\hbar^2)$, getting
\begin{empheq}[box={\mybluebox[5pt]}]{equation}
\hat{\mathcal{H}}=-\frac{1}{2}\nabla^2-\frac{Z}{r}+\frac{l(l+1)}{2r^2},
\end{empheq}
which again corresponds to setting $\hbar=m_e=k_e=e=1$.