\chapter{Scaling}

\section{Quantum dots - Natural units}
For circular quantum dots, the Hamiltonian is in one dimension given by
\begin{equation}
\hat{\mathcal{H}}=-\frac{\hbar^2}{2m}\frac{\partial^2}{\partial x^2}+\frac{1}{2}m\omega^2x^2
\label{eq:HamiltonianHO}
\end{equation}
with $\hbar$ as the reduced Planck's constant, $m$ as the electron mass and $\omega$ as the oscillator frequency. The corresponding wave functions read
\begin{equation}
\psi_n(x)=\frac{1}{\sqrt{2^nn!}}\cdot\bigg(\frac{m\omega}{\pi\hbar}\bigg)^{1/4}\exp\Big(-\frac{m\omega}{2\hbar}x^2\Big)H_n\Big(\sqrt{\frac{m\omega}{\hbar}}x\Big).
\end{equation}
We want to get rid of $\hbar$ and $m$ in equation \eqref{eq:HamiltonianHO} to make it dimensionless, which can be accomplished by scaling  $\hat{\mathcal{H}}'\equiv \hat{\mathcal{H}}/\hbar$, such that the Hamiltonian reduces to
\begin{equation}
\hat{\mathcal{H}}'=-\frac{\hbar}{2m}\frac{\partial^2}{\partial x^2}+\frac{1}{2}\frac{m\omega^2}{\hbar}x^2.
\end{equation}
Moreover, we observe that the fraction $\hbar/m$ comes in both terms, which can be avoided by introducing a characteristic length $x'\equiv x\cdot\sqrt{m/\hbar}$. The final Hamiltonian is
\begin{equation}
\hat{\mathcal{H}}=\frac{1}{2}\frac{\partial^2}{\partial x^2}+\frac{1}{2}\omega^2x^2
\end{equation}
which corresponds to setting $\hbar=m=1$. In natural units, one often sets $\omega=1$ as well by scaling $\hat{\mathcal{H}}'=\hat{\mathcal{H}}/\hbar\omega$, but since we want to keep the $\omega$-dependency, we do it slightly different. This means that the exact wave functions for the one-particle one-dimensional case goes as
\begin{equation}
\psi_n(x)\propto\exp\Big(-\frac{\omega}{2}x^2\Big)H_n(\sqrt{\omega}x)
\end{equation}
where we omit the normalization factor with the Metropolis algorithm in mind. 

\newpage
\section{Atomic systems - Atomic units}
The atomic Hamiltonian for an electron in subshell $l$ affected by a nucleus with atomic number $Z$ reads
\begin{equation}
\hat{\mathcal{H}}=-\frac{\hbar^2}{2m}\frac{\partial^2}{\partial r^2}-k_e\frac{Ze^2}{r}+\frac{\hbar^2l(l+1)}{2mr^2}.
\label{eq:HamiltonianAtomic}
\end{equation}
where $e$ is the elementary charge and $k_e$ is Coulomb's constant. Again we want to get rid of the reduced Planck's constant $\hbar$ and the electron mass $m$ in order to make the Hamiltonian dimensionless, which we do by dividing all terms by $\hbar^2/m$,
\begin{equation}
\hat{\mathcal{H}}\cdot\frac{m}{\hbar^2}=-\frac{1}{2}\frac{\partial^2}{\partial r^2}+k\frac{m}{\hbar^2}\frac{Ze^2}{r}-\frac{1}{2}\frac{l(l+1)}{r^2}
\end{equation}
and then define $a\equiv mke^2/\hbar^2$. If we then divide all terms by $a^2$, we can write the Hamiltonian as
\begin{equation}
\hat{\mathcal{H}}\cdot\frac{m}{a^2\hbar^2}=-\frac{1}{2a^2}\frac{\partial^2}{\partial r^2}+\frac{Z}{ar}-\frac{1}{2a^2}\frac{l(l+1)}{r^2}
\end{equation}
and obtain a dimensionless equation by scaling $r'=ar$ and $\hat{\mathcal{H}}'=\hat{\mathcal{H}}\cdot m/a^2\hbar^2$. The final Hamiltonian is
\begin{equation}
\hat{\mathcal{H}}=-\frac{1}{2}\frac{\partial^2}{\partial r^2}-\frac{Z}{r}+\frac{l(l+1)}{2r^2}.
\end{equation}

\section{Comparison between natural and atomic units}
As a summary, we will present how the observable are scaled nicely in a table, and how to convert them back to standard units.
\begin{table} [H]
	\caption{Comparison the natural and atomic units presented above.  \vspace{2mm}}
	\begin{tabularx}{\textwidth}{X|XXX} \hline\hline
		\label{tab:energies2P2D}
		Quantity & Symbol & Natural units & Atomic units \\ \hline
		Energy & $E$ & $1/\hbar$ & $\hbar^2/m(ke^2)^2$\\ 
		Length & $r$ & $\sqrt{m/\hbar}$ & $m(ke^2)/\hbar^2$\\
		Reduced Planck's constant & $\hbar$ & 1 & 1 \\
		Elementary charge & $e$ & 1 & $\sqrt{\alpha}$ \\
		Coulomb's constant & $k_e$ & 1 & 1 \\
		Boltzmann's constant & $k_B$ & 1 & 1 \\
		Electron rest mass & $m_e$ & 1 & $511 keV$ \\ \hline\hline
	\end{tabularx}
\end{table}

By a mix of classical and quantum mechanics, Niels Bohr found the quantized energy levels and radii in an atom to be
\begin{equation}
E_n=-\frac{Z^2(ke^2)^2m}{2\hbar^2n^2}\approx-\frac{Z^2}{n^2}13.6\text{ eV}
\end{equation}
and
\begin{equation}
r_n=\frac{n^2\hbar^2}{Zke^2m}\approx\frac{n^2}{Z}5.29\cdot10^{-11}\text{ m}
\end{equation}
respectively. What we observe, is that the energy in atomic units is scaled with respect to $2\cdot E_1$ and $r_1$, which means that
\begin{equation}
1\text{ a.u.} = 2\cdot13.6\text{ eV}\quad\quad\text{and}\quad\quad 1 \text{ a.u.} = 5.29\cdot10^{-11}\text{ m}
\end{equation}