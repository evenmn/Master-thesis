\thispagestyle{empty}
\begin{center} \vspace{1cm}
    \textbf{\Large{\mtitle}}\\ \vspace{0.5cm}
    \small{by}\\ \vspace{0.5cm}
    \large{\mauthor}\\ \vspace{4.4cm}
    \large{THESIS}\\ \vspace{0.3cm}
    \small{for the degree of}\\ \vspace{0.3cm}
    \large{MASTER OF SCIENCE}\\ \vspace{0.7cm}
    \includegraphics[scale=1.0]{Images/UiO_Segl_pms485.eps} \\ \vspace{0.5cm}
    \large{Faculty of Mathematics and Natural Sciences \\ University of Oslo} \\ \vspace{0.5cm}
    \small{\mdate}\\ \vfill
\end{center}

\newpage
\section*{Abstract}
This thesis aims the study of circular quantum dots using machine learning to reduce the need of physical intuition

\thispagestyle{empty}
\clearpage

\thispagestyle{empty}
\clearpage

\section*{Acknowledgements}
5 years at Blindern

Thanks to Morten Hjorth-Jensen for all the help, motivation and believing in me.

Thanks to Håkon Emil Kristiansen and Alocias Mariadason for answering all my questions and for you patience. 

Thanks to my family for all the support, and my friends who always are there when I need to disconnect from the studies. 

Last, but not least, I would like to say thank you to the Computational Physics group...

I have always believed that everything has a reason, which I used to point out whenever people talked about things that could not easily be explained. First when I was introduced to quantum mechanics, I realized that it might not be that simple, which is one of the reasons why this beautiful theory caught my attention. Even though my philosophy has changed to a less deterministic direction, I still like to think that everything can be explained out of some first principles. 
    
\thispagestyle{empty}
\clearpage

{%
    %\microtypesetup{protrusion=false}
    \tableofcontents
    %\microtypesetup{protrusion=true}
    \thispagestyle{empty}
    \clearpage}%

\thispagestyle{empty}
\clearpage

\section*{List of abbreviations}
\begin{table}[H]
    \centering
    \begin{tabular}{lcl}
        \textbf{Letters} & & \textbf{Meaning} \\
        RBM & - & Restricted Boltzmann Machine \\
        MC & - & Monte Carlo \\
        VMC & - & Variational Monte Carlo \\
        DMC & - & Diffusion Monte Carlo \\
        ML & - & Machine Learning \\
        WF & - & Wave Function \\
        SPF & - & Single Particle Function \\

    \end{tabular}
    \caption{List of symbols used with explaination.}
    \label{tab:symbols}
\end{table}

\section*{Numbers used in sums}
\begin{table}[H]
	\centering
	\begin{tabular}{lcl}
		\textbf{Letters} & & \textbf{Meaning} \\
		P & - & Number of particles \\
		D & - & Number of dimensions \\
		F & - & Number of free dimensions \\
		H & - & Number of hidden nodes \\
		M & - & Number of Monte-Carlo cycles \\
		
	\end{tabular}
	\caption{Numbers used in sums.}
	\label{tab:symbols2}
\end{table}
\thispagestyle{empty}
\clearpage

\section*{Source Code}
    The source code is given in \url{https://github.com/evenmn}

\thispagestyle{empty}
\clearpage
