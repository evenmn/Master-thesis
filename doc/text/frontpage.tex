\thispagestyle{empty}
\begin{center} \vspace{1cm}
    \textbf{\Large{\mtitle}}\\ \vspace{0.5cm}
    \small{by}\\ \vspace{0.5cm}
    \large{\mauthor}\\ \vspace{4.4cm}
    \large{THESIS}\\ \vspace{0.3cm}
    \small{for the degree of}\\ \vspace{0.3cm}
    \large{MASTER OF SCIENCE}\\ \vspace{0.7cm}
    \includegraphics[scale=1.0]{Images/UiO_Segl_pms485.eps} \\ \vspace{0.5cm}
    \large{Faculty of Mathematics and Natural Sciences \\ University of Oslo} \\ \vspace{0.5cm}
    \small{\mdate}\\ \vfill
\end{center}
\newpage
\vspace*{\fill}
{\setlength{\parindent}{0cm}
All illustrations in this thesis are created using the TikZ package \cite{tantau_graph_2013} if nothing else is specified. The plots are produced using a controversial combination of Matplotlib \cite{hunter_matplotlib:_2007} and PGFPlots \cite{tantau_graph_2013}.\bigskip

Effort was made to follow the ISO 80000-2:2009 standard for mathematical signs and symbols \cite{iso/tc_12_iso_nodate}, and the ISO 80000-9:2009 standard for quantities and units in physical chemistry \cite{iso/tc_12_iso_nodate-1}.\bigskip

The \LaTeX\, document preparation system was applied for typesetting.
} 
\newpage
\section*{Abstract}
This thesis aims to investigate how machine learning can be used to perform quantum many-body simulations, which is a possible way of reducing the need for physical intuition. This is important as the information required by traditional methods is nonexistent or unavailable for most of the physically interesting systems. Restricted Boltzmann machines (RBM) and variational Monte Carlo (VMC) methods have many common features, and in our work, they were both implemented from scratch in order to compare a new approach to a traditional approach. A simple Jastrow factor and the well-known Padé-Jastrow factor were added to the RBM, abbreviated RBM+SJ and RBM+PJ respectively, to see how the different methods model electron-electron correlations. Our primary focus is on closed-shell circular quantum dots, where we compute the ground-state energy and electron densities for two-dimensional systems with up to 90 electrons and three-dimensional systems with up to 70 electrons. Also, double quantum dots were studied, and in the end, we present a few results on atoms to demonstrate the flexibility and generality of our implemented code.

The energy obtained by the RBM was reasonably close to experimental results, and it got even closer as we added more complex correlation factors gradually. For an RBM+PJ, the energy was found to be lower for small dots compared to the VMC energy. However, the one-body density profile reveals that the RBM gives more distinctly located electrons compared to the VMC method, which can be explained by the way the RBM models the correlations. From the two-body density profile, we also observe that the repulsive interactions get more significant as we add a Jastrow factor. For low-frequency dots, the one-body density undergoes a phase transformation with more distinct radial peaks, which indicates the Wigner crystallization effect.

The computational time consumption was found to be favorable for the RBM for small systems and VMC for large systems, which can be explained by the exploding number of variational parameters in the RBM as the system sizes increase. RBMs with Jastrow factors were significantly more computationally expensive than the other methods, and apparently, there is no point in adding a simple Jastrow factor when we can add a more complicated Jastrow factor. 

\thispagestyle{empty}
\cleardoublepage

\section*{Acknowledgements}
After five exciting years at Blindern my time here has come to an end, and on that occasion I would like to acknowledge some people who have been influential throughout the studies. First, I would like to thank my excellent supervisor, Morten Hjorth-Jensen, whom I luckily got to know three years ago. From day one, you took me under your wing and enthralled me with your eager, massive knowledge and work ethic. Few things make me more motivated than talking with you, be it in real-life at Blindern or through video conversations from whatever place you happen to be at.

I would also like to acknowledge the support I have got from my parents and my sister. Albeit the schedule has been filled up and I have not spent as much time with you as I wanted, you have always been supportive and stand up for me anytime I need some help. Thanks to my friends (you know who you are) who are always ready for a beer (or ten) whenever I seek to disconnect from the studies. Those moments filled with lively discussions and dark humor have kept me motivated throughout the studies, which has been absolutely crucial.

The computational physics group is a funny composition of different people with a shared predilection for physics and programming. There are so many talented guys in the group, and I have really appreciated spending early mornings and late nights with you. Last, but not least, I would also like to thank my high school teacher, Jens Otto Opaker, who with his enthusiasm and dedication, got me hooked on science in the first place.

\iffalse
\textcolor{red}{I have always believed that everything has a reason, which I used to point out whenever people talked about things that could not easily be explained. First when I was introduced to quantum mechanics, I realized that it might not be that simple, which is one of the reasons why this beautiful theory caught my attention. Even though my philosophy has changed to a less deterministic direction, I still like to think that everything can be explained out of some first principles.}
\fi
    
\thispagestyle{empty}
\cleardoublepage

\newpage

{%
    %\microtypesetup{protrusion=false}
    \tableofcontents
    %\microtypesetup{protrusion=true}
    \thispagestyle{empty}
    \clearpage}%

\thispagestyle{empty}
\clearpage


