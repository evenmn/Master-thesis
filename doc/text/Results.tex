\chapter{Results} \label{sec:results}
\epigraph{Great quote.}{Author}
\begin{figure}[H]
	\centering
	\includegraphics[scale=0.4]{Images/example.png}
	\caption{}
\end{figure}
Now over to the heart of this thesis: the results...

\newpage
\section{No repulsive interaction}
We start with the non-interacting case in order to prove the flexibility of the implemented code. 

\subsection{Ground state energy}
\subsubsection{Quantum dots}
We will start studying systems with no interaction between particles. The number of closed-shell particles are given by equation \eqref{eq:HOclosedshell} and the exact energies are found from formula \eqref{eq:HOenergies}. The first 5 closed-shell energies for $\omega=0.5$ and $\omega=1.0$ are presented in table \eqref{tab:quantumdotswointeraction}.

\begin{table} [H]
	\caption{Energy of $N$ non-interacting electrons trapped in a harmonic oscillator of frequency $\omega=0.5$ and $\omega=1.0$. $E_{\text{NQS1}}$ is the standard NQS wave function with traditional Slater determinant, while $E_{\text{NQS2}}$ is NQS wave function where the RBM itself generates the Slater determinant. }
	\label{tab:quantumdotswointeraction}
	\begin{tabularx}{\textwidth}{r|r|XXXX:XXXX} \hline\hline
		\label{tab:nn}
		& $\omega$ & \multicolumn{4}{c}{0.5}&\multicolumn{4}{c}{1.0}\\ \hline
		& $N$ & $E_{\text{NQS1}}$ & $E_{\text{NQS2}}$ & $E_{\text{VMC}}$ & $E_{\text{exact}}$ & $E_{\text{NQS1}}$ & $E_{\text{NQS2}}$ & $E_{\text{VMC}}$ & $E_{\text{exact}}$ \\ \hline
		
		\parbox[t]{2mm}{\multirow{5}{*}{\rotatebox[origin=c]{90}{2D}}}
		&2 & 0.0 & 0.0 & 1.0 & 1 & 2.0 & 0.0 & 2.0 & 2\\
		&6 & 0.0 & 0.0 & 5.0 & 5 & 0.0 & 0.0 & 10.0 & 10 \\
		&12 & 0.0 & 0.0 & 14.0 & 14 & 0.0 & 0.0 & 28.0 & 28\\
		&20 & 0.0 & 0.0 & 30.0 & 30 & 0.0 & 0.0 & 60.0 & 60\\
		&30 & 0.0 & 0.0 & 55.0 & 55 & 0.0 & 0.0 & 110.0 & 110\\ \hline
		
		\parbox[t]{2mm}{\multirow{5}{*}{\rotatebox[origin=c]{90}{3D}}}
		&2 & 0.0 & 0.0 & 1.5 & 1.5 & 0.0 & 0.0 & 3.0 & 3 \\
		&8 & 0.0 & 0.0 & 9.0 & 9 & 0.0 & 0.0 & 18.0 & 18 \\
		&20 & 0.0 & 0.0 & 30.0 & 30 & 0.0 & 0.0 & 60.0 & 60 \\
		&40 & 0.0 & 0.0 & 75.0 & 75 & 0.0 & 0.0 & 150.0 & 150 \\
		&70 & 0.0 & 0.0 & 157.5 & 157.5 & 0.0 & 0.0 & 315.0 & 315 \\ \hline\hline
	\end{tabularx}
\end{table}
We observe that the NQS wave function is able to reproduce the exact energy for most of the cases, but when the number of particles get large, the statistical error gets significant.

\subsubsection{Atoms}
In the next we will study atoms when the Coulomb interaction is ignored. For those cases, we know the exact answers, given by the Bohr formula presented in equation \eqref{eq:bohrformula}.
\begin{table} [H]
	\caption{ }
	\begin{tabularx}{\textwidth}{X|X|X:X:X} \hline\hline
		\label{tab:nointeractionatoms}
		Atom & $N$ & $E_{\text{NQS}}$ & $E_{\text{VMC}}$ & $E_{\text{exact}}$ \\ \hline
		H & 1 & 0.0 & -0.5 & -0.5 \\ 
		He & 2 & 0.0 & -4.0 & -4 \\
		Be & 4 & 0.0 & 0.0 & -20 \\
		Ne & 10 & 0.0 & 0.0 & -200 \\ \hline\hline
	\end{tabularx}
\end{table}


\subsection{One-body density}
The one-body densities 
\iffalse
\begin{figure} [H]%
	\centering
	\subfloat[2D, $\omega=0.5$]{{\includegraphics[width=8cm]{/home/evenmn/VMC/doc/html/onebody_VMC_NQS_P2_D2_w0p5.png}}}
	\subfloat[2D, $\omega=1.0$]{{\includegraphics[width=8cm]{/home/evenmn/VMC/doc/html/onebody_VMC_NQS_P2_D2_w1p0.png} }}\\
	
	\subfloat[3D, $\omega=0.5$]{{\includegraphics[width=8cm]{/home/evenmn/VMC/doc/html/onebody_VMC_NQS_P2_D3_w0p5.png} }}
	\subfloat[3D, $\omega=1.0$]{{\includegraphics[width=8cm]{/home/evenmn/VMC/doc/html/onebody_VMC_NQS_P2_D3_w1p0.png} }}
	\caption{One-body densities of two non-interacting electrons in two dimensions figure (a) and (b) and three dimensions}%
	\label{fig:OB_nointeraction}
\end{figure}
\fi

\section{With repulsive interaction}
\begin{table} [H]
	\caption{This table presents the energies of $N$ electrons trapped in a two-dimensional oscillator well with frequency $\omega$. The reference is to J. Høgberget CITE HIM (DMC). }. 
	\begin{tabularx}{\textwidth}{l:l|XXXXXX} \hline\hline
		\label{tab:quantumdotswinteraction2D}
		$N$ & $\omega$ & $E_{\text{NQS1}}$ & $E_{\text{NQS2}}$ & $E_{\text{NQS3}}$ & $E_{\text{VMC}}$ & $E_{\text{ref}} $ & $E_{\text{exact}}$ \\ \hline \\
		2 & 0.1 & 0.4427(2) & 0.0 & 0.0 & 0.44206(3) & 0.44079(1) & \\ 
		& 0.28 & 1.0225(1) & 0.0 & 0.0 & 1.02233(2) & 1.02164(1) & \\
		& 0.5 & 1.6599(1) & 0.0 & 0.0 & 1.66020(2) & 1.65977(1)  \\
		& 1.0 & 3.00002(2) & 0.0 & 0.0 & 3.00030(1) & 3.00000(1) & 3  \\ \hdashline \\
		
		6 & 0.1 & 0.0 & 0.0 & 0.0 & 0.0 & 3.55385(5) \\ 
		& 0.28 & 0.0 & 0.0 & 0.0 & 0.0 & 7.60019(6) \\
		& 0.5 & 0.0 & 0.0 & 0.0 & 0.0 & 11.78484(6) \\
		& 1.0 & 0.0 & 0.0 & 0.0 & 0.0 & 20.15932(8) \\ \hdashline \\
		
		12 & 0.1 & 0.0 & 0.0 & 0.0 & 0.0 & 12.26984(8) \\ 
		& 0.28 & 0.0 & 0.0 & 0.0 & 0.0 & 25.63577(9) \\
		& 0.5 & 0.0 & 0.0 & 0.0 & 0.0 & 39.1596(1) \\
		& 1.0 & 0.0 & 0.0 & 0.0 & 0.0 & 65.7001(1) \\ \hdashline \\
		
		20 & 0.1 & 0.0 & 0.0 & 0.0 & 0.0 & 29.9779(1) \\ 
		& 0.28 & 0.0 & 0.0 & 0.0 & 0.0 & 61.9268(1) \\
		& 0.5 & 0.0 & 0.0 & 0.0 & 0.0 & 93.8752(1) \\
		& 1.0 & 0.0 & 0.0 & 0.0 & 0.0 & 155.8822(1) \\ \hdashline \\
		
		30 & 0.1 & 0.0 & 0.0 & 0.0 & 0.0 & 60.4205(2)\\ 
		& 0.28 & 0.0 & 0.0 & 0.0 & 0.0 & 123.9683(2)\\
		& 0.5 & 0.0 & 0.0 & 0.0 & 0.0 & 187.0426(2)\\
		& 1.0 & 0.0 & 0.0 & 0.0 & 0.0 & 308.5627(2)\\ \hline\hline
	\end{tabularx}
\end{table}

\begin{table} [H]
	\caption{This table presents the energies of $N$ electrons trapped in a three-dimensional oscillator well with frequency $\omega$. The exact energies are calculated analytically by M.Taut, see \cite{taut_two_1993}. The reference is to J. Høgberget CITE HIM (DMC). }. 
	\begin{tabularx}{\textwidth}{l:l|XXXXXX} \hline\hline
		\label{tab:quantumdotswinteraction3D}
		$N$ & $\omega$ & $E_{\text{NQS1}}$ & $E_{\text{NQS2}}$ & $E_{\text{NQS3}}$ & $E_{\text{VMC}}$ & $E_{\text{ref}} $ & $E_{\text{exact}}$ \\ \hline \\
		2 & 0.1 & 0.50073(7) & 0.0 & 0.0 & 0.52099(1) & 0.499997(3) & 0.5 \\
		& 0.28 & 1.20201(5) & 0.0 & 0.0 & 1.20212(2) & 1.201725(2) & \\
		& 0.5 & 2.00009 & 0.0 & 0.0 & 2.00010(2) & 2.000000(2) & 2.0 \\
		& 1.0 & 3.73032(4) & 0.0 & 0.0 & 3.73017(2) & 3.730123(3) &  \\ \hdashline \\
		
		8 & 0.1 & 0.0 & 0.0 & 0.0 & 0.0 & 5.7028(1) \\ 
		& 0.28 & 0.0 & 0.0 & 0.0 & 0.0 & 12.1927(1) \\
		& 0.5 & 0.0 & 0.0 & 0.0 & 0.0 & 18.9611(1) \\
		& 1.0 & 0.0 & 0.0 & 0.0 & 0.0 & 32.6680(1) \\ \hdashline \\
		
		20 & 0.1 & 0.0 & 0.0 & 0.0 & 0.0 & 27.2717(2) \\ 
		& 0.28 & 0.0 & 0.0 & 0.0 & 0.0 & 56.3868(2) \\
		& 0.5 & 0.0 & 0.0 & 0.0 & 0.0 & 85.6555(2) \\
		& 1.0 & 0.0 & 0.0 & 0.0 & 0.0 & 142.8875(2) \\ \hdashline \\
		
		40 & 0.1 & 0.0 & 0.0 & 0.0 & 0.0 \\ 
		& 0.28 & 0.0 & 0.0 & 0.0 & 0.0 \\
		& 0.5 & 0.0 & 0.0 & 0.0 & 0.0 \\
		& 1.0 & 0.0 & 0.0 & 0.0 & 0.0 \\ \hdashline \\
		
		70 & 0.1 & 0.0 & 0.0 & 0.0 & 0.0 \\ 
		& 0.28 & 0.0 & 0.0 & 0.0 & 0.0 \\
		& 0.5 & 0.0 & 0.0 & 0.0 & 0.0 \\
		& 1.0 & 0.0 & 0.0 & 0.0 & 0.0 \\ \hline\hline
	\end{tabularx}
\end{table}
