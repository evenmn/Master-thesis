\chapter{Distributions of Gaussian-binary restricted Boltzmann machines} \label{app:rbmderive}
In this appendix, we will derive the marginal and conditional distributions of a Gaussian-binary restricted Boltzmann machine with the system energy
\begin{equation}
E(\bs{x},\bs{h})=\sum_{i=1}^{F}\frac{(x_i-a_i)^2}{2\sigma_i^2}-\sum_{j=1}^Hb_jh_j-\sum_{i=1}^F\sum_{j=1}^{H}\frac{x_iw_{ij}h_j}{\sigma_i^2}.
\end{equation}
There are $F$ visible units $x_i$ with related bias weights $a_i$ and $H$ hidden units $h_j$ with related bias weights $b_j$. $w_{ij}$ are the weights connecting the visible units to the hidden units. The joint probability distribution is given by the Boltzmann distribution
\begin{equation}
P(\bs{x},\bs{h})=\frac{1}{Z}\exp(-\beta E(\bs{x},\bs{h})),
\end{equation}
where $Z$ is the partition function,
\begin{equation}
Z=\iint d\bs{x}d\bs{h}P(\bs{x},\bs{h}),
\end{equation}
and $\beta=1/k_BT$ is a factor that will be fixed to 1. As the marginal and conditional distributions are closely related both for the visible and hidden layer, we present the distributions in sections respective for the two layers. 

\section{Distributions of visible units}
The distributions of the visible units are used to find properties related to the visible units. If we recall a restricted Boltzmann machine, the transformation between the visible units and the hidden units is $f_j(\bs{x};\bs{\theta})=b_j+\sum_{i=1}^Fw_{ij}x_i/\sigma_i^2$. By this expression, we can express the joint probability distribution as
\begin{equation}
\begin{aligned}
P(\bs{x,\bs{h}})&=\frac{1}{Z}\exp(-\sum_{i=1}^{F}\frac{(x_i-a_i)^2}{2\sigma_i^2}+\sum_{j=1}^Hb_jh_j+\sum_{i=1}^F\sum_{j=1}^{H}\frac{x_iw_{ij}h_j}{\sigma_i^2}),\\
&=\frac{1}{Z}\exp(-\sum_{i=1}^F \frac{(x_i - a_i)^2}{2\sigma^2}+\sum_{j=1}^H\left[b_jh_j+\frac{\sum_{i=1}^Fh_jw_{ij}x_i}{\sigma_i^2}\right]),\\
&=\frac{1}{Z}\exp(-\sum_{i=1}^F \frac{(x_i - a_i)^2}{2\sigma^2}+\sum_{j=1}^Hh_jf_j(\bs{x};\bs{\theta})),
\end{aligned}
\label{eq:jointvisible}
\end{equation}
which will be the base of the other distributions of the visible units. 

\subsection{Marginal distribution}
The marginal distribution of the visible units is given by the sum
\begin{equation}
P(\bs{h})=\sum_{\{\bs{h}\}} P(\bs{x},\bs{h}),
\end{equation}
as the hidden units can take binary values only. By inserting the expression of the joint probability distribution from equation \eqref{eq:jointvisible}, we obtain
\begin{equation}
\begin{aligned}
P(\bs{x})&=\frac{1}{Z}\sum_{\{\bs{h}\}}\exp(\sum_{i=1}^F \frac{(x_i - a_i)^2}{2\sigma^2}+\sum_{j=1}^Hh_jf_j(\bs{x};\bs{\theta})),\\
&=\frac{1}{Z}\exp(-\sum_{i=1}^F\frac{(x_i - a_i)^2}{2\sigma^2}) \sum_{\{\bs{h}\}}\prod_{j=1}^H\exp(h_jf_j),\\
&=\frac{1}{Z}\exp(-\sum_{i=1}^F \frac{(x_i - a_i)^2}{2\sigma^2})\sum_{h_1=0}^1\sum_{h_2=0}^1\hdots\sum_{h_H=0}^1\exp(h_1f_1)\exp(h_2f_2)\hdots\exp(h_Hf_H),\\
%&=\frac{1}{Z}\exp\Big(\sum_{i=1}^F \frac{(x_i - a_i)^2}{2\sigma^2}\Big)\sum_{h_1=0}^1\sum_{h_2=0}^1\hdots\sum_{h_H=0}^1\exp\Big(h_1f_1\Big)\exp\Big(h_2f_2\Big)\hdots\exp\Big(h_Hf_H\Big)\\
%&=\frac{1}{Z}\exp\Big(\sum_{i=1}^F \frac{(x_i - a_i)^2}{2\sigma^2}\Big)\sum_{h_1=0}^1\exp\Big(h_1f_1\Big)\sum_{h_2=0}^1\exp\Big(h_2f_2\Big)\hdots\sum_{h_H=0}^1\exp\Big(h_Hf_H\Big)\\
&=\frac{1}{Z}\exp(-\sum_{i=1}^F \frac{(x_i - a_i)^2}{2\sigma^2})\prod_{j=1}^H\sum_{h_j=0}^1\exp(h_jf_j),\\
&=\frac{1}{Z}\exp(-\sum_{i=1}^F \frac{(x_i - a_i)^2}{2\sigma^2}) \prod_{j=1}^H \left[1+ \exp(f_j(\bs{x};\bs{\theta}))\right].
\end{aligned}
\label{eq:marginalvisible}
\end{equation}
This is what we will use as the marginal distribution of the visible units. 

\subsection{Conditional distribution}
The conditional distribution of the visible units is given by
\begin{eqnarray}
P(\bs{h}|\bs{x})=\frac{P(\bs{x},\bs{h})}{P(\bs{x})},
\end{eqnarray}
where the numerator is the joint probability distribution given in equation \eqref{eq:jointvisible} and the denominator is the marginal distribution of the visible units given in equation \eqref{eq:marginalvisible}. This gives the conditional probability
\begin{equation}
\begin{aligned}
P(\bs{h}|\bs{x})&=\frac{\frac{1}{Z}\exp(-\sum_{i=1}^F \frac{(x_i - a_i)^2}{2\sigma^2}+\sum_{j=1}^Hh_jf_j(\bs{x};\bs{\theta}))}{\frac{1}{Z}\exp(-\sum_{i=1}^F \frac{(x_i - a_i)^2}{2\sigma^2}) \prod_{j=1}^H \left[1+ \exp(f_j(\bs{x};\bs{\theta}))\right]},\\
&=\prod_{j=1}^H\frac{\exp(h_jf_j(\bs{x};\bs{\theta}))}{1+\exp(f_j(\bs{x};\bs{\theta}))},
\end{aligned}
\end{equation}
where we recall that $f_j(\bs{x};\bs{\theta})=b_j+\sum_{i=1}^Fw_{ij}x_i/\sigma_i^2$ are just the transition going from the visible units to the hidden units.

\section{Distributions of hidden units}
The distributions of the hidden units are used to find properties related to the hidden units. If we recall a restricted Boltzmann machine, the transformation between the hidden units and the visible units is $g_i(\bs{h};\bs{\theta})=a_i+\sum_{j=1}^Hh_jw_{ij}$. By using that expression, we can express the joint probability distribution as
\begin{equation}
\begin{aligned}
P(\bs{x,\bs{h}})&=\frac{1}{Z}\exp(-\sum_{i=1}^{F}\frac{(x_i-a_i)^2}{2\sigma_i^2}+\sum_{j=1}^Hb_jh_j+\sum_{i=1}^F\sum_{j=1}^{H}\frac{x_iw_{ij}h_j}{\sigma_i^2}),\\
&=\frac{1}{Z}\exp(\sum_{j=1}^Hb_jh_j-\sum_{i=1}^F\frac{x_i^2+a_i^2-2a_ix_i-2x_i\sum_{j=1}^Hh_jw_{ij}}{2\sigma_i^2}),\\
&=\frac{1}{Z}\exp(\sum_{j=1}^Hb_jh_j-\sum_{i=1}^F\frac{x_i^2+a_i^2-2x_ig_i(\bs{h};\bs{\theta})}{2\sigma_i^2}),
\end{aligned}
\label{eq:jointhidden}
\end{equation}
which will be the base of the other distributions of the hidden units. 

\subsection{Marginal distribution}
The marginal distribution of the hidden units is given by the integral
\begin{equation}
P(\bs{h})=\int d\bs{x}P(\bs{x},\bs{h}),
\end{equation}
as the visible units can take continuous values. By inserting the expression of the joint probability distribution from equation \eqref{eq:jointhidden}, we obtain
\begin{equation}
\begin{aligned}
P(\bs{h})&=\frac{1}{Z}\int d\bs{x} \exp(\sum_{j=1}^Hb_jh_j-\sum_{i=1}^F\frac{x_i^2+a_i^2-2x_ig_i(\bs{h};\bs{\theta})}{2\sigma_i^2}),\\
&=\frac{1}{Z}\exp(\sum_{j=1}^Hb_jh_j)\prod_{i=1}^F\int dx_i\exp(\frac{x_i^2+a_i^2-2x_ig_i(\bs{h};\bs{\theta})}{2\sigma_i^2}),\\
&=\frac{1}{Z}\exp(\sum_{j=1}^Hb_jh_j)\prod_{i=1}^F\int dx_i\exp(\frac{x_i^2+a_i^2-2x_ig_i(\bs{h};\bs{\theta})+g_i(\bs{h};\bs{\theta})^2-g_i(\bs{h};\bs{\theta})^2}{2\sigma_i^2}),\\
&=\frac{1}{Z}\exp(\sum_{j=1}^Hb_jh_j)\prod_{i=1}^F\int dx_i\exp(\frac{(x_i-g_i(\bs{h};\bs{\theta}))^2+a_i^2-g_i(\bs{h};\bs{\theta})^2}{2\sigma_i^2}),\\
&=\frac{1}{Z}\exp(\sum_{j=1}^Hb_jh_j)\prod_{i=1}^F\exp(\frac{-a_i^2+g_i(\bs{h};\bs{\theta})}{2\sigma_i^2})\underbrace{\int dx_i\exp(\frac{(x_i-g_i(\bs{h};\bs{\theta}))^2}{2\sigma_i^2})}_{\sqrt{2\pi\sigma_i^2}},\\
&=\frac{1}{Z}\exp(\sum_{j=1}^Hb_jh_j)\prod_{i=1}^F\sqrt{2\pi\sigma_i^2}\exp(\frac{-a_i^2+g_i(\bs{h};\bs{\theta})}{2\sigma_i^2}).
\end{aligned}
\label{eq:marginalhidden}
\end{equation}
This is what we have used as the marginal distribution of the hidden units.

\subsection{Conditional distribution}
The conditional distribution of the hidden units is given by
\begin{eqnarray}
P(\bs{x}|\bs{h})=\frac{P(\bs{x},\bs{h})}{P(\bs{h})},
\end{eqnarray}
where the numerator is the joint probability distribution given in equation \eqref{eq:jointhidden} and the denominator is the marginal distribution of the hidden units given in equation \eqref{eq:marginalhidden}. This gives the conditional probability
\begin{equation}
\begin{aligned}
P(\bs{x}|\bs{h})&=\frac{\frac{1}{Z}\exp(\sum_{j=1}^Hb_jh_j-\sum_{i=1}^F\frac{x_i^2+a_i^2-2x_ig_i(\bs{h};\bs{\theta})}{2\sigma_i^2})}{\frac{1}{Z}\exp(\sum_{j=1}^Hb_jh_j)\prod_{i=1}^F\sqrt{2\pi\sigma_i^2}\exp(\frac{-a_i^2+g_i(\bs{h};\bs{\theta})}{2\sigma_i^2})},\\
&=\prod_{i=1}^F\frac{1}{\sqrt{2\pi\sigma_i^2}}\exp(-\frac{x_i^2+a_i^2-2s_ig_i(\bs{h};\bs{\theta})-a_i^2+g_i(\bs{h};\bs{\theta})}{2\sigma_i^2}),\\
&=\prod_{i=1}^F\frac{1}{\sqrt{2\pi\sigma_i^2}}\exp(-\frac{(x_i-g_i(\bs{h};\bs{\theta}))^2}{2\sigma_i^2}),\\
&=\prod_{i=1}^F\mathcal{N}(x_i;g_i(\bs{h};\bs{\theta})),
\end{aligned}
\end{equation}
which is just the joint normal distribution centered around $g_i(\bs{h};\bs{\theta})=a_i+\sum_{j=1}^Hb_jh_j$ with $\sigma_i^2$ as the variance. 