\chapter{Derivation of a Gaussian-binary restricted Boltzmann machine} \label{app:rbmderive}
In this appendix, we will derive the marginal and conditional distributions of a Gaussian-binary restricted Boltzmann machine with the system energy
\begin{equation}
E(\bs{x},\bs{h})=\sum_{i=1}^{F}\frac{(x_i-a_i)^2}{2\sigma_i^2}-\sum_{j=1}^Hb_jh_j-\sum_{i=1}^F\sum_{j=1}^{H}\frac{x_iw_{ij}h_j}{\sigma_i^2}.
\end{equation}
There are $F$ visible units $x_i$ with related bias weights $a_i$ and $H$ hidden units $h_j$ with related bias weights $b_j$. $w_{ij}$ are the weights connecting the visible units to the hidden units. The joint probability distribution is given by the Boltzmann distribution
\begin{equation}
P(\bs{x},\bs{h})=\frac{1}{Z}\exp(-\beta E(\bs{x},\bs{h})),
\end{equation}
where $Z$ is the partition function
\begin{equation}
Z=\iint d\bs{x}d\bs{h}P(\bs{x},\bs{h})
\end{equation}
and $\beta=1/k_BT$ is a factor which will be fixed to 1. As the marginal and conditional distributions are closely related both for the visible and hidden units, the distributions will be presented in the sections related to the two layers respectively. Afterwards, we will also present the gradient of the marginal distributions and the log-likelihood function.

\section{Distributions of visible units}
The distributions of the visible units are used to find properties related to the visible units. If we recall a restricted Boltzmann machine, the transformation between the visible nodes and the hidden nodes is $f_j(\bs{x};\bs{\theta})=b_j+\sum_{i=1}^Fw_{ij}x_i/\sigma_i^2$. By using that expression, we can express the joint probability distribution as
\begin{equation}
\begin{aligned}
P(\bs{x,\bs{h}})&=\frac{1}{Z}\exp(-\sum_{i=1}^{F}\frac{(x_i-a_i)^2}{2\sigma_i^2}+\sum_{j=1}^Hb_jh_j+\sum_{i=1}^F\sum_{j=1}^{H}\frac{x_iw_{ij}h_j}{\sigma_i^2})\\
&=\frac{1}{Z}\exp(-\sum_{i=1}^F \frac{(x_i - a_i)^2}{2\sigma^2}+\sum_{j=1}^H\Big[b_jh_j+\frac{\sum_{i=1}^Fh_jw_{ij}x_i}{\sigma_i^2}\Big])\\
&=\frac{1}{Z}\exp(-\sum_{i=1}^F \frac{(x_i - a_i)^2}{2\sigma^2}+\sum_{j=1}^Hh_jf_j(\bs{x};\bs{\theta}))
\end{aligned}
\label{eq:jointvisible}
\end{equation}
which will be the base of the other distributions of the hidden nodes. 

\subsection{Marginal distribution}
The marginal distribution of the visible nodes is given by the sum
\begin{equation}
P(\bs{h})=\sum_{\{\bs{h}\}} P(\bs{x},\bs{h}),
\end{equation}
as the hidden nodes can take binary values only. By inserting the expression of the joint probability distribution from equation \eqref{eq:jointvisible}, we obtain
\begin{equation}
\begin{aligned}
P(\bs{x})&=\frac{1}{Z}\sum_{\{\bs{h}\}}\exp(\sum_{i=1}^F \frac{(x_i - a_i)^2}{2\sigma^2}+\sum_{j=1}^Hh_jf_j(\bs{x};\bs{\theta}))\\
&=\frac{1}{Z}\exp(-\sum_{i=1}^F\frac{(x_i - a_i)^2}{2\sigma^2}) \sum_{\{\bs{h}\}}\prod_{j=1}^H\exp\Big(h_jf_j\Big)\\
&=\frac{1}{Z}\exp(-\sum_{i=1}^F \frac{(x_i - a_i)^2}{2\sigma^2})\sum_{h_1=0}^1\sum_{h_2=0}^1\hdots\sum_{h_H=0}^1\exp\Big(h_1f_1\Big)\exp\Big(h_2f_2\Big)\hdots\exp\Big(h_Hf_H\Big)\\
%&=\frac{1}{Z}\exp\Big(\sum_{i=1}^F \frac{(x_i - a_i)^2}{2\sigma^2}\Big)\sum_{h_1=0}^1\sum_{h_2=0}^1\hdots\sum_{h_H=0}^1\exp\Big(h_1f_1\Big)\exp\Big(h_2f_2\Big)\hdots\exp\Big(h_Hf_H\Big)\\
%&=\frac{1}{Z}\exp\Big(\sum_{i=1}^F \frac{(x_i - a_i)^2}{2\sigma^2}\Big)\sum_{h_1=0}^1\exp\Big(h_1f_1\Big)\sum_{h_2=0}^1\exp\Big(h_2f_2\Big)\hdots\sum_{h_H=0}^1\exp\Big(h_Hf_H\Big)\\
&=\frac{1}{Z}\exp\Big(-\sum_{i=1}^F \frac{(x_i - a_i)^2}{2\sigma^2}\Big)\prod_{j=1}^H\sum_{h_j=0}^1\exp\Big(h_jf_j\Big)\\
&=\frac{1}{Z}\exp\Big(-\sum_{i=1}^F \frac{(x_i - a_i)^2}{2\sigma^2}\Big) \prod_{j=1}^H \left[1+ \exp(f_j(\bs{x};\bs{\theta}))\right]
\end{aligned}
\label{eq:marginalvisible}
\end{equation}

\subsection{Conditional distribution}
The conditional distribution of the visible units is given by
\begin{eqnarray}
P(\bs{h}|\bs{x})=\frac{P(\bs{x},\bs{h})}{P(\bs{x})}
\end{eqnarray}
where the nominator is the joint probability distribution given in equation \eqref{eq:jointvisible} and the denominator is the marginal distribution of the visible nodes given in equation \eqref{eq:marginalvisible}. This gives the conditional probability
\begin{equation}
\begin{aligned}
P(\bs{h}|\bs{x})&=\frac{\frac{1}{Z}\exp(-\sum_{i=1}^F \frac{(x_i - a_i)^2}{2\sigma^2}+\sum_{j=1}^Hh_jf_j(\bs{x};\bs{\theta}))}{\frac{1}{Z}\exp(-\sum_{i=1}^F \frac{(x_i - a_i)^2}{2\sigma^2}) \prod_{j=1}^H \bigg[1+ \exp(f_j(\bs{x};\bs{\theta}))\bigg]}\\
&=\prod_{j=1}^H\frac{\exp(h_jf_j(\bs{x};\bs{\theta}))}{1+\exp(f_j(\bs{x};\bs{\theta}))}
\end{aligned}
\end{equation}
where we recall that $f_j(\bs{x};\bs{\theta})=b_j+\sum_{i=1}^Fw_{ij}x_i/\sigma_i^2$ are just the transition going from the visible nodes to the hidden nodes.

\section{Distributions of hidden units}
The distributions of the hidden units are used to find properties related to the hidden units. If we recall a restricted Boltzmann machine, the transformation between the hidden nodes and the visible nodes is $g_i(\bs{h};\bs{\theta})=a_i+\sum_{j=1}^Hh_jw_{ij}$. By using that expression, we can express the joint probability distribution as
\begin{equation}
\begin{aligned}
P(\bs{x,\bs{h}})&=\frac{1}{Z}\exp(-\sum_{i=1}^{F}\frac{(x_i-a_i)^2}{2\sigma_i^2}+\sum_{j=1}^Hb_jh_j+\sum_{i=1}^F\sum_{j=1}^{H}\frac{x_iw_{ij}h_j}{\sigma_i^2})\\
&=\frac{1}{Z}\exp(\sum_{j=1}^Hb_jh_j-\sum_{i=1}^F\frac{x_i^2+a_i^2-2a_ix_i-2x_i\sum_{j=1}^Hh_jw_{ij}}{2\sigma_i^2})\\
&=\frac{1}{Z}\exp(\sum_{j=1}^Hb_jh_j-\sum_{i=1}^F\frac{x_i^2+a_i^2-2x_ig_i(\bs{h};\bs{\theta})}{2\sigma_i^2})
\end{aligned}
\label{eq:jointhidden}
\end{equation}
which will be the base of the other distributions of the hidden nodes. 

\subsection{Marginal distribution}
The marginal distribution of the hidden nodes is given by the integral
\begin{equation}
P(\bs{h})=\int d\bs{x}P(\bs{x},\bs{h}),
\end{equation}
as the visible nodes can take continuous values. By inserting the expression of the joint probability distribution from equation \eqref{eq:jointhidden}, we obtain
\begin{equation}
\begin{aligned}
P(\bs{h})&=\frac{1}{Z}\int d\bs{x} \exp(\sum_{j=1}^Hb_jh_j-\sum_{i=1}^F\frac{x_i^2+a_i^2-2x_ig_i(\bs{h};\bs{\theta})}{2\sigma_i^2})\\
&=\frac{1}{Z}\exp(\sum_{j=1}^Hb_jh_j)\prod_{i=1}^F\int dx_i\exp(\frac{x_i^2+a_i^2-2x_ig_i(\bs{h};\bs{\theta})}{2\sigma_i^2})\\
&=\frac{1}{Z}\exp(\sum_{j=1}^Hb_jh_j)\prod_{i=1}^F\int dx_i\exp(\frac{x_i^2+a_i^2-2x_ig_i(\bs{h};\bs{\theta})+g_i(\bs{h};\bs{\theta})^2-g_i(\bs{h};\bs{\theta})^2}{2\sigma_i^2})\\
&=\frac{1}{Z}\exp(\sum_{j=1}^Hb_jh_j)\prod_{i=1}^F\int dx_i\exp(\frac{(x_i-g_i(\bs{h};\bs{\theta}))^2+a_i^2-g_i(\bs{h};\bs{\theta})^2}{2\sigma_i^2})\\
&=\frac{1}{Z}\exp(\sum_{j=1}^Hb_jh_j)\prod_{i=1}^F\exp(\frac{-a_i^2+g_i(\bs{h};\bs{\theta})}{2\sigma_i^2})\underbrace{\int dx_i\exp(\frac{(x_i-g_i(\bs{h};\bs{\theta}))^2}{2\sigma_i^2})}_{\sqrt{2\pi\sigma_i^2}}\\
&=\frac{1}{Z}\exp(\sum_{j=1}^Hb_jh_j)\prod_{i=1}^F\sqrt{2\pi\sigma_i^2}\exp(\frac{-a_i^2+g_i(\bs{h};\bs{\theta})}{2\sigma_i^2})
\end{aligned}
\label{eq:marginalhidden}
\end{equation}

\subsection{Conditional distribution}
The conditional distribution of the hidden units is given by
\begin{eqnarray}
P(\bs{x}|\bs{h})=\frac{P(\bs{x},\bs{h})}{P(\bs{h})}
\end{eqnarray}
where the nominator is the joint probability distribution given in equation \eqref{eq:jointhidden} and the denominator is the marginal distribution of the hidden nodes given in equation \eqref{eq:marginalhidden}. This gives the conditional probability
\begin{equation}
\begin{aligned}
P(\bs{x}|\bs{h})&=\frac{\frac{1}{Z}\exp(\sum_{j=1}^Hb_jh_j-\sum_{i=1}^F\frac{x_i^2+a_i^2-2x_ig_i(\bs{h};\bs{\theta})}{2\sigma_i^2})}{\frac{1}{Z}\exp(\sum_{j=1}^Hb_jh_j)\prod_{i=1}^F\sqrt{2\pi\sigma_i^2}\exp(\frac{-a_i^2+g_i(\bs{h};\bs{\theta})}{2\sigma_i^2})}\\
&=\prod_{i=1}^F\frac{1}{\sqrt{2\pi\sigma_i^2}}\exp(-\frac{x_i^2+a_i^2-2s_ig_i(\bs{h};\bs{\theta})-a_i^2+g_i(\bs{h};\bs{\theta})}{2\sigma_i^2})\\
&=\prod_{i=1}^F\frac{1}{\sqrt{2\pi\sigma_i^2}}\exp(-\frac{(x_i-g_i(\bs{h};\bs{\theta}))^2}{2\sigma_i^2})\\
&=\prod_{i=1}^F\mathcal{N}(x_i;g_i(\bs{h};\bs{\theta}))
\end{aligned}
\end{equation}
which is just the joint normal distribution centered around $g_i(\bs{h};\bs{\theta})=a_i+\sum_{j=1}^Hb_jh_j$ with $\sigma_i^2$ as the variance. 

\iffalse
\section{Closed-form expressions of gradients}\label{sec:derivatives}
By defining the single-particle function as the marginal probability, we have seen that the wave function of a general Gaussian-binary restricted Boltzmann machine takes the form
\begin{equation}
P(\bs{x})=\frac{1}{Z}\exp(-\sum_{i=1}^{F}\frac{(x_i-a_i)^2}{2\sigma^2})\prod_{j=1}^H\Big[1+\exp(f_j(\bs{x};\bs{\theta}))\Big]
\label{eq:GBRBM}
\end{equation}
where $f_j(\bs{x};\bs{\theta})$ is an arbitrary function of the coordinates $\bs{x}$ and the weights $\bs{\theta}$. The Gaussian part is straight-forward to differentiate, so we will keep our attention on the product,
\begin{equation}
\Psi_{\text{rp}}(\bs{x};\bs{\theta})=\prod_{j=1}^H\Big[1+\exp(f_j(\bs{x};\bs{\theta}))\Big].
\end{equation}
We will henceforth no longer specify the arguments $\bs{x}$ and $\bs{\theta}$ of the functions, as all the functions take the same arguments. By introducing the functions
\begin{equation}
p_j\equiv \frac{1}{1+\exp(+f_j)}\quad\wedge\quad n_j\equiv \frac{1}{1+\exp(-f_j)},
\end{equation}
where the last one is the sigmoid function, we find the gradient and Laplacian of $\ln\Psi_{rp}$ to be
\begin{empheq}[box={\mybluebox[5pt]}]{equation}
\nabla_k\ln\Psi_{\text{rp}}=\sum_{j=1}^Hn_j\nabla_k(f_j)
\end{empheq}
and
\begin{empheq}[box={\mybluebox[5pt]}]{equation}
\nabla_k^2\ln\Psi_{\text{rp}}=\sum_{j=1}^Hn_j\big[\nabla_k^2(f_j)+p_j\big(\nabla_k(f_j)\big)^2\big]
\end{empheq}
respectively. Those expressions can be used to find the kinetic energy directly, as the kinetic energy contribution from this specific element is just the sum over the gradients and Laplacians, $T=\sum_{k=1}^F[\nabla_k^2\ln\Psi_{\text{rp}}+(\nabla_k\ln\Psi_{\text{rp}})^2]$. An arbitrary parameter $\theta_i$ can be updated according to the log-likelihood function which turns out to be just the derivative of the log-likelihood function
\begin{empheq}[box={\mybluebox[5pt]}]{equation}
\frac{\partial}{\partial \theta_i}\ln \Psi_{\text{rp}}=\sum_{j=1}^Hn_j\frac{\partial}{\partial\theta_i}(f_j)
\end{empheq}
and the ratio between the new and the old wave function elements can be found by the product
\begin{empheq}[box={\mybluebox[5pt]}]{equation}
\frac{\Psi_{\text{rp}}^{\text{new}}}{\Psi_{\text{rp}}^{\text{old}}}=\prod_{j=1}^H\frac{p_j^{\text{old}}}{p_j^{\text{new}}}.
\end{empheq}
As a conclusion, what we actually need to calculate to find respective expressions for each wave function element is $\nabla_k(f_j)$, $\nabla_k^2(f_j)$ and $\partial_{\theta_i}(f_j)$, which is naturally simpler than differentiating the entire wave function element. This applies to all elements on the form presented in equation \eqref{eq:GBRBM}, including general Gaussian-binary restricted Boltzmann machines and deep Boltzmann machines as long as all the units are Gaussian-binary.
\fi