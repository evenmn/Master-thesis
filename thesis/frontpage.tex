\thispagestyle{empty}
\begin{center} \vspace{1cm}
    \textbf{\Large{\mtitle}}\\ \vspace{0.5cm}
    \small{by}\\ \vspace{0.5cm}
    \large{\mauthor}\\ \vspace{4.4cm}
    \large{THESIS}\\ \vspace{0.3cm}
    \small{for the degree of}\\ \vspace{0.3cm}
    \large{MASTER OF SCIENCE}\\ \vspace{0.7cm}
    \includegraphics[scale=1.0]{../Images/UiO_Segl_pms485.eps} \\ \vspace{0.5cm}
    \large{Faculty of Mathematics and Natural Sciences \\ University of Oslo} \\ \vspace{0.5cm}
    \small{\mdate}\\ \vfill
\end{center}
\newpage
\vspace*{\fill}
{\setlength{\parindent}{0cm}
All illustrations in this thesis are created using the TikZ package \supercite{tantau_graph_2013} if nothing else is specified. The plots are produced using a controversial combination of Matplotlib \supercite{hunter_matplotlib:_2007} and PGFPlots \supercite{tantau_graph_2013}.\bigskip

\iffalse
Effort was made to follow the ISO 80000-2:2009 standard for mathematical signs and symbols \supercite{iso/tc_12_iso_nodate}, and the ISO 80000-9:2009 standard for quantities and units in physical chemistry \supercite{iso/tc_12_iso_nodate-1}.\bigskip\fi

The \LaTeX\, document preparation system was applied for typesetting.}
 
\newpage
\section*{Abstract}
With the ability to solve the many-body Schrödinger equation accurately, in principle all physics and chemistry could be derived from first principles. However, exact wave functions of realistic and interesting systems are in general unavailable because they are non-deterministic polynomial-hard to compute \supercite{troyer_computational_2005}, implying that we need to rely on approximations. The variational Monte Carlo (VMC) method is widely used for ground state studies, but requires a trial wave function ansatz which must trade off between efficiency and accuracy. The method also has many common features with machine learning algorithms, and as neural networks have shown impressive power as function approximators, the idea is to use a neural network as the trial wave function guess. For fermionic systems, like electronic structure systems, the wave function needs to obey Fermi-Dirac statistics, which typically is achieved using a Slater determinant. As a neural network hardly can model this feature, our approach is to replace the single-particle functions in the Slater determinant with restricted Boltzmann machines (RBM). In addition, we add further correlations via so-called Jastrow factors \supercite{drummond_jastrow_2004}.

Our primary focus is on closed-shell circular quantum dots, where we compute the ground state energy and electron density of two-dimensional systems with up to $N=90$ electrons and three-dimensional systems with up to $N=70$ electrons. The energy obtained by the RBM was reasonably close to experimental results, and it gradually became closer as we added more complex correlation factors. For the our most complicated Jastrow factor, the energy was found to be lower than the VMC-energy for small dots, but for larger dots it was slightly higher. However, the one-body density profile reveals that the RBM gives more distinctly located electrons compared to the VMC method, which can be explained by the way the RBM models the correlations. From the two-body density profile, we also observe that the repulsive interactions get more significant as we add a Jastrow factor. Based on the electron densities and the energy distribution between kinetic and potential energy, it is certain that the various methods provide different electron configurations. For low-frequency dots, the electron density becomes more localized with an additional radial peak compared to high-frequency dots. This is reminiscent of what is known as Wigner localization \supercite{ghosal_incipient_2007}.

The computational time consumption was found to be favorable for the RBM for small systems and VMC for large systems, which can be explained by the exploding number of variational parameters in the RBM as the system sizes increase. RBMs with Jastrow factors were significantly more computationally expensive than the other methods, and evidently, there is no reason to add a simple Jastrow factor when we can add a more complicated Jastrow factor instead.

\thispagestyle{empty}
\cleardoublepage

\section*{Acknowledgements}
After five exciting years at Blindern I would like to acknowledge some people who have been important throughout the studies. First, I would like to thank my excellent supervisor, Morten Hjorth-Jensen, whom I luckily got to know three years ago. From day one, you took me under your wing and enthralled me with your eager, work ethic and massive knowledge. Few things make me more motivated than a conversation with you, be it in real-life at Blindern or through video conversations from whatever place you happen to be at.

I would also like to acknowledge the support I have got from my parents and my sister. Albeit the schedule has been filled up and I have not spent as much time with you as I wanted, you have always been supportive and stand up for me anytime I need some help. Thanks to my friends (you know who you are) who are always ready for a beer (or ten) whenever I seek to disconnect from the studies. Those moments filled with lively discussions and dark humor have kept me motivated throughout the studies, which has been absolutely crucial.

The computational physics group is a funny composition of different people with a shared predilection for physics and programming. There are so many talented guys in the group, and I have really appreciated spending early mornings and late nights with you. Last, but not least, I would also like to thank my high school teacher, Jens Otto Opaker, who with his enthusiasm and dedication, got me hooked on science in the first place.

Thanks to Sebastian Gregorious Winther-Larsen, Robert Solli, Marius Jonsson and Kaitlin Rose Preusser for proofreading and valuable annotations.

\iffalse
\textcolor{red}{I have always believed that everything has a reason, which I used to point out whenever people talked about things that could not easily be explained. First when I was introduced to quantum mechanics, I realized that it might not be that simple, which is one of the reasons why this beautiful theory caught my attention. Even though my philosophy has changed to a less deterministic direction, I still like to think that everything can be explained out of some first principles.}
\fi
    
\thispagestyle{empty}
\cleardoublepage

\newpage

{%
    %\microtypesetup{protrusion=false}
    \tableofcontents
    %\microtypesetup{protrusion=true}
    \thispagestyle{empty}
    \clearpage}%

\thispagestyle{empty}
\clearpage


