\chapter{Dirac Formalism} \label{app:dirac}
The Dirac formalism, also called bracket notation, was suggested by \citet{dirac_new_1939} in a 1939 paper to improve the reading ease. The notation unites the integral representation and the matrix representation in an elegant fashion by representing all wave functions as vectors in the \textit{Hilbert-space}. 

The Hilbert-space is a complete linear vector space, which allows length, inner products and angles between vectors to be measured. A column vector in the space is denoted by $\ket{\psi}$, which is called the \textit{ket}, and by taking the Hermitian conjugate of it we obtain the corresponding row vector, $(\ket{\psi})^+=\bra{\psi}$ called the \textit{bra}. As the Hilbert-space is characterized by linearity, it requires that the sum of two element in the space is also an element in the space,
\begin{equation}
\ket{\psi_1}+\ket{\psi_2}=\ket{\psi_1+\psi_2},
\end{equation}
and that two elements always \textit{commute},
\begin{equation}
\ket{\psi_1}+\ket{\psi_2}=\ket{\psi_2}+\ket{\psi_1}.
\end{equation}
Another important property is that an element of the space multiplied with a complex number is also an element of the space,
\begin{equation}
c\ket{\psi}=\ket{c\psi}\quad\quad\forall\quad c\in\mathbb{C}.
\end{equation}

To fully utilize the notation, orthogonality properties need to be taken into account. Assume that we have a orthogonal basis set 
\begin{equation}
\{\psi_1,\psi_2,\cdots,\psi_n\}.
\end{equation}
The inner product between two basis elements is then given by
\begin{equation}
\braket{\psi_i}{\psi_j}=
\begin{cases}
\neq 0\quad&\text{if}\quad i=j\\
=0\quad&\text{otherwise}
\end{cases}
\quad\equiv\delta_{ij},
\end{equation}
where we have introduced the Kronecker delta $\delta_{ij}$. If we further require that our basis is \textit{orthonormal}, the inner product is 1, and we can prove the \textit{completeness relation}. Assume we want to expand a vector $\ket{\chi}$ in our orthonormal basis set,
\begin{align}
\ket{\chi}&=\sum_{i=1}^nc_i\ket{\psi_i}.
\end{align}
By multiplying with one of the basis vectors, $\bra{\psi_j}$, on the left-hand side, the sum collapses, and we are just left with the coefficient $c_j$,
\begin{align}
\braket{\psi_j}{\chi} &= \sum_{i=1}^nc_i\braket{\psi_j}{\psi_i}=c_j.
\end{align}
If we now again insert this into the expansion, we obtain
\begin{equation}
\ket{\chi}=\sum_{i=1}^n\underbrace{\braket{\psi_i}{\chi}}_{c_i}\ket{\psi_i}=\bigg[\sum_{i=1}^n\ket{\psi_i}\bra{\psi_i}\bigg]\ket{\chi},
\end{equation}
which implies that the outer product is equal to 1 (vectorized equal to the identity matrix),
\begin{equation}
\sum_{i=1}^n\ket{\psi_i}\bra{\psi_i}=1.
\end{equation}

We can use these properties to demonstrate how the normalization constant can be obtained without explicitly solving any integrals. Consider two spin-1/2 particles, for example the electron and the proton in the ground state of hydrogen. The system can be found in four different states: the \textit{triplet} with $s=1$ and the \textit{singlet} with $s=0$. The latter has a wave function which can be expressed as \supercite{griffiths_introduction_2005}
\begin{equation}
\ket{00}=A\big(\ket{\uparrow\downarrow}-\ket{\downarrow\uparrow}\big),
\end{equation}
which is associated with the bra
\begin{equation}
\bra{00}=\big(\ket{00}\big)^+=A\big(\bra{\uparrow\downarrow}-\bra{\downarrow\uparrow}\big).
\end{equation}
The inner product is then given by
\begin{equation}
\begin{aligned}
\braket{00}{00}&=A\big(\bra{\uparrow\downarrow}-\bra{\downarrow\uparrow}\big)A\big(\ket{\uparrow\downarrow}-\ket{\downarrow\uparrow}\big),\\
&=A^2\big(\braket{\uparrow\downarrow}{\uparrow\downarrow}-\braket{\uparrow\downarrow}{\downarrow\uparrow}-\braket{\downarrow\uparrow}{\uparrow\downarrow}+\braket{\downarrow\uparrow}{\downarrow\uparrow}\big),\\
&=A^2\big(1-0-0+1\big),\\
&=2A^2=1,
\end{aligned}
\end{equation}
where we have assumed that also the states $\ket{\uparrow\downarrow}$ and $\ket{\downarrow\uparrow}$ are orthonormal. From this, we can see that the normalization constant $A$ must be equal to $1/\sqrt{2}$.

Further, we also use the Dirac notation as a short-hand notation for the Slater determinant, discussed in section \ref{sec:slater}. Instead of writing out the entire determinant, it is common to write it as
\begin{equation}
\ket{\psi_1\psi_2,\cdots,\psi_N}\equiv\frac{1}{\sqrt{N!}}
\begin{vmatrix}
\psi_1(\boldsymbol{r}_1,\sigma_1) & \psi_2(\boldsymbol{r}_1,\sigma_1) & \cdots & \psi_n(\boldsymbol{r}_1,\sigma_1)\\
\psi_1(\boldsymbol{r}_2,\sigma_2) & \psi_2(\boldsymbol{r}_2,\sigma_2) & \cdots & \psi_N(\boldsymbol{r}_2,\sigma_2)\\
\vdots & \vdots & \ddots & \vdots \\
\psi_1(\boldsymbol{r}_N,\sigma_N) & \psi_2(\boldsymbol{r}_N,\sigma_N) & \cdots & \psi_N(\boldsymbol{r}_N,\sigma_N)
\end{vmatrix},
\end{equation}
which is extensively used in for instance second quantization. This should not be confused with the Hartree product. 