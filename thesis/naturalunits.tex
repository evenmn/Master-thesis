\chapter{Natural Units} \label{app:units}
In everyday life, we usually stick to the standard SI units when measuring or expressing quantities like distance, energy, mass and time. A standardized unit system is important because it is common for people around the world and simplifies communication across borders. Moreover, we gradually develop intuitions about units when they are used frequently. We then immediately observe if a number makes sense or not when it is expressed in the preferred units. As a European, I use the metric system. When a person is 1.90 meters tall, I can easily imagine their height. On the other hand, when Americans tell their height, I do not have an intuition for how tall 6 feet 3 inches is. 

In science, the SI units are often not the preferred ones, especially not on very large or small scales. For instance, measuring cosmological distances in meters is very unpractical, as the distance to the Sun is $\sim1.5\cdot10^{11}$ meters and the distance to our closest neighbor galaxy, Andromeda, is $\sim 2.4\cdot10^{22}$ meters. Instead, we use units like the astronomical unit [a.u.] (should not be confused with atomic units) and light-years. 

For small scales, the situation is similar. For instance, the most probable distance between the nucleus and the electron in the hydrogen atom is the Bohr radius, which is $a_0\approx5.3\cdot10^{-11}$ meters. These scales are unpractical to work with in SI units. Instead, we define so-called natural units, which is a common term for units based on universal physical constants. The hydrogen atom is usually treated in a variant of atomic units with $a_0=1$, and in section \ref{sec:atomicunits} we will discuss the Hartree atomic units. The quantum dots will be scaled differently, with units named just natural units.

\section{Quantum dots}
For quantum dots, the one-dimensional Hamiltonian in SI units reads
\begin{equation}
\hat{\mathcal{H}}=-\frac{\hbar^2}{2m}\frac{\partial^2}{\partial x^2}+\frac{1}{2}m\omega^2x^2,
\label{eq:HamiltonianHO}
\end{equation}
with $\hbar$ as the reduced Planck's constant, $m$ as the electron mass and $\omega$ as the oscillator frequency. The corresponding wave functions read
\begin{equation}
\phi_n(x)=\frac{1}{\sqrt{2^nn!}}\cdot\left(\frac{m\omega}{\pi\hbar}\right)^{1/4}\exp(-\frac{m\omega}{2\hbar}x^2)H_n\left(\sqrt{\frac{m\omega}{\hbar}}x\right),
\end{equation}
where $H_n(x)$ are the Hermite polynomials. We want to get rid of $\hbar$ and $m$ in equation \eqref{eq:HamiltonianHO} to make it dimensionless. This can be accomplished by scaling  $\hat{\mathcal{H}}'= \hat{\mathcal{H}}/\hbar$, such that the Hamiltonian reduces to
\begin{equation}
\hat{\mathcal{H}}'=-\frac{\hbar}{2m}\frac{\partial^2}{\partial x^2}+\frac{1}{2}\frac{m\omega^2}{\hbar}x^2.
\end{equation}
We now observe that the fraction $\hbar/m$ appears in both terms, so we can avoid the constants by introducing a characteristic length, $x'= x/\sqrt{\hbar/m}$. This results in the Hamiltonian 
\begin{equation}
\hat{\mathcal{H}}=\frac{1}{2}\frac{\partial^2}{\partial x^2}+\frac{1}{2}\omega^2x^2,
\end{equation}
which corresponds to setting $\hbar=m=1$. In natural units, one often sets $\omega=1$ as well by scaling $\hat{\mathcal{H}}'=\hat{\mathcal{H}}/\hbar\omega$, but since we want to keep the $\omega$-dependency, we do it slightly different. This means that the exact wave functions for the one-particle one-dimensional case is given by
\begin{equation}
\phi_n(x)=\frac{1}{\sqrt{2^nn!}}\cdot\left(\frac{\omega}{\pi}\right)^{1/4}\exp(-\frac{\omega}{2}x^2)H_n(\sqrt{\omega}x)
\end{equation} 
in natural units.

\section{Atoms} \label{sec:atomicunits}
The atomic Hamiltonian for an electron in subshell $l$ affected by a nucleus with atomic number $Z$ reads
\begin{equation}
\hat{\mathcal{H}}=-\frac{\hbar^2}{2m_e}\nabla^2-\frac{1}{4\pi\epsilon_0}\frac{Ze^2}{r}+\frac{\hbar^2l(l+1)}{2m_er^2}
\label{eq:HamiltonianAtomic}
\end{equation}
in SI units, where $e$ is the elementary charge and $k_e=1/4\pi\epsilon_0$ is Coulomb's constant. Again we want to get rid of the reduced Planck's constant, $\hbar$, and the electron mass, $m_e$, in order to make the Hamiltonian dimensionless. We can do this by multiplying all terms by $(4\pi\epsilon_0)^2\hbar^2/m_ee^4Z^2$,
\begin{equation}
\hat{\mathcal{H}}\cdot\frac{(4\pi\epsilon_0)^2\hbar^2}{m_ee^4Z^2}=-\frac{(4\pi\epsilon_0)^2\hbar^4}{2m_e^2e^4Z^2}\nabla^2+\frac{4\pi\epsilon_0\hbar^2}{m_ee^2Zr}-\frac{(4\pi\epsilon_0)^2\hbar^4}{m_e^2e^4Z^2}\frac{l(l+1)}{r^2}.
\end{equation}
This might look very chaotic, but by exploiting that the Bohr radius,
\begin{equation}
a_0\equiv\frac{4\pi\epsilon_0\hbar^2}{m_ee^2Z},
\end{equation}
is found in all the right-hand side terms, the Hamiltonian reduces to
\begin{equation}
\hat{\mathcal{H}}\cdot\frac{(4\pi\epsilon_0)^2\hbar^2}{m_ee^4Z^2}=-\frac{a_0^2}{2}\nabla^2+a_0\frac{Z}{r}-a_0^2\frac{l(l+1)}{2r^2}.
\end{equation}
We obtain the dimensionless Hamiltonian in Hartree atomic units by introduce the length scaling $r'=r/a_0$ and the energy scaling $\hat{\mathcal{H}}'=\hat{\mathcal{H}}/(m_ee^4Z^2/(4\pi\epsilon_0)^2\hbar^2)$,
\begin{equation}
\hat{\mathcal{H}}=-\frac{1}{2}\nabla^2-\frac{Z}{r}+\frac{l(l+1)}{2r^2}.
\end{equation}
Similar to the natural units discussed for the quantum dots, this corresponds to setting $\hbar=m_e=k_e=e=1$.