\chapter{Retrospect and Future Work} \label{sec:conclusion}
In this chapter, we will make an attempt to compress the relatively comprehensive discussion in the previous chapter down to a more tangible conclusion. Thereafter, we address some possible extensions of our work, how our work can contribute to solving the quantum many-body puzzle and why it is important. 

% Conclude ground state energy results
We have seen that a VMC simulation with the trial wave function ansatz determined by the marginal distributions of a Gaussian-binary restricted Boltzmann machine (RBM) is capable of producing reasonable ground state energy estimates, and when we add more intuition in the form of Jastrow factors of different complexities the energy drops further towards the diffusion Monte Carlo (DMC) energy. Most notably, a VMC simulations with RBM with the Padé-Jastrow factor (RBM+PJ) provides lower ground state energy and statistical error than the energies and errors of a variational Monte Carlo simulation with a standard Slater-Jastrow wave function (VMC) for the smallest dots. This indicates that the method can provide a wave function closer to the exact one than standard VMC. However, for larger quantum dots, the RBM+PJ gives a slightly higher energy than the VMC, but we suspect this is a consequence of a large number of variational parameters as we consequently set the number of hidden units, $H$, equal to the number of electrons in the dot, $N$. In machine learning terms, we use a too complex model for our problem. We decided to do this because \citet{nordhagen_computational_2018} found $H=N$ to be optimal for small quantum dots, but it could be different for larger dots. \citet{carleo_solving_2017} operate with a hidden variable density $\alpha=H/F$ with $F$ as the degrees of freedom (number of visible units), which they set to an integer number and thus end up with more variational parameters than we do. A conclusion is that the number of hidden units might not be optimal, and thus some more investigation is needed. We also observe that all the methods more or less give the same ratio between kinetic and potential energy for all system sizes and all frequencies, which means the electron configuration is fundamentally different for the different methods.

% Conclude electron density results
Throughout the results, we had a thorough discussion of the electron density provided by the various methods, which revealed some significant differences between the methods that cannot be seen just from the ground state energy. The most notable difference is found for the one-body density produced using VMC and RBM, where RBM tends to exaggerate the fluctuations compared to VMC. As discussed, this difference is probably caused by how the two methods model the electron-electron correlations. The same effect was found in the two-body density plots, where the difference between the various correlation models is even more significant. In general, the RBM+PJ and VMC give a more significant electron-electron repulsion than the fellow methods RBM and RBM with a simple Jastrow factor (RBM+SJ). This allows the conclusion that the energy estimates are not necessarily the best way to compare various trial wave function ansätze, other observable are potentially more crucial. The RBM+SJ is an excellent example of this, as it provides energy estimates similar to the VMC, but the two-body density plots exploit that the correlations were somewhat weaker. In general, we believe that the Padé-Jastrow factor works better than the simple Jastrow factor as it provides a lower energy and is constructed to model the electron-electron cusp correctly. As the simple Jastrow factor is more or less as computationally expensive as the Padé-Jastrow factor, we see no reason to select RBM+SJ instead of RBM+PJ.

% Conclude RBM
Based on the discussions above, the RBM provides exciting results, but at its current version it is not able to compete with the existing many-body methods neither when it comes to performance, nor computational cost. However, we see the outcome of this work as a step in the right direction, and with some more investigation we believe that the RBM can be an alternative to traditional methods. More precisely, the plain RBM has some properties that makes it able to estimate the ground state energy at a lower cost than the VMC, and other RBM structures, for instance based on spherical coordinates, might enhance the performance at the same cost. We also see a bright future for the RBM+PJ, which for some systems give a lower energy than VMC, and our thought is that it can outperform the VMC if the cost is reduced.

% Go back to the goals announced in introduction
The overall goal of this work was to develop and investigate a method that required less physical intuition. To achieve this, we had to develop a flexible VMC framework from scratch and implement the restricted Boltzmann machines as trial wave function ansätze. Throughout the results in the previous chapter, we have carefully validated the framework, which seems to give consistent results with references. It is hard to validate the trial wave functions based on RBMs as no previous research has done anything comparable. However, when comparing to the results obtained using other methods, we are confident that also this implementation is correct. Furthermore, we managed to produce a large number of ground state results of the quantum dots using various trial wave function ansätze with and without adding a Jastrow factor. These results were compared extensively, resulting in a thorough evaluation of the trial wave functions. Selected atoms were studied using the hydrogen-like orbitals in the trial wave function, but we did not have time or manage to study them using the RBMs. We put some effort in trying to model the atoms using the same RBMs as for the quantum dots, but even with a large number of hidden units, these Gaussian-binary unit RBMs were not flexible enough to capture the properties of the atoms. Other possible attempts include expanding a Hartree-Fock basis in a set of RBMs, or simply choose another RBM structure which is not based on the Gaussian mapping. This is something that can and should be tried, and is one among many things that are desirable for future work.

\section*{Future work}
The use of machine learning for solving the many-body problem is just in the starting block, and there are millions of aspects to be investigated. Using the same approach as we did, there are plenty of network architectures, hyper-parameter setting and initial conditions that we did not have time to explore. RBMs with a smaller number of hidden units is something that requires future investigations. Also, writing an RBM code in spherical coordinates, instead of Cartesian coordinates, could be interesting as it might be easier to model the correlations in that coordinate system.

\citet{pfau2019abinitio} created a trial wave function ansatz based on Slater determinants, where the single-particle functions were determined by neural networks. Both the relative distances between all the particles and the collective positions where passed into the neural network. Using a standard VMC framework, they obtained lower ground state energies of atoms than provided by traditional methods. In future studies, these calculations could be repeated using RBMs as the single-particle functions.

Initially, the idea behind using the RBM as the trial wave function ansatz, was that it can be used to investigate systems with unavailable wave functions. Future studies should therefore consider more complicated systems. In the first place, multi-quantum dots (multiple quantum dots with intern connections) are good candidates as there exist comparable experiments \supercite{marzin_photoluminescence_1994,brunner_sharp-line_1994}. As the RBM is able to model two-body correlations, it is also imaginable that it can model three-body correlations and therefore be used to simulate nuclear systems where the three-body component in the wave function is not properly understood \supercite{sauer_three-nucleon_2014}.

In section \ref{sec:dmc}, we discussed the sign-problem plaguing the diffusion Monte Carlo method, and that shadow wave functions can be used to bypass this problem. There are many similarities between the shadow wave functions and wave functions based on restricted Boltzmann machines, and it would be interesting to investigate this link further.

As the quantum simulations are costly, one should always try to find the bottleneck and optimize that part of the code in order to simulate larger systems. For RBMs, the bottleneck might be the neural networks, which can be evaluated extremely fast on a GPU. However, the remaining framework is probably faster to evaluate on CPUs, so a hybrid of GPU and CPU would might be optimal.