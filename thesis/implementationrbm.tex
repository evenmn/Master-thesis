\chapter{Implementation: Restricted Boltzmann Machines} \label{chp:rbmimplementation}
In chapter \ref{chp:WFE}, we described common optimization procedures for a standard variational Monte Carlo (VMC) implementation, and we also presented implementation examples taken from the code. In this section, we will do the same, but for the restricted Boltzmann machines (RBM). As we have pointed out before, the same sampling methods and optimization algorithms can be used both for the VMC implementation and the RBM implementation. This implies that much of the VMC framework can be reused. The already described parts of the code will naturally not be described again, and for that reason, this chapter will more or less exclusively deal with the RBM wave function elements.

The main goal of this work is to reduce the physical intuition needed when doing quantum computations, and that is the task of the restricted Boltzmann machines. The idea is to use a flexible basis set based on RBMs, which are implemented as the elements of the Slater matrix, as first seen in section \ref{sec:slater}. Recall that the Slater determinant can be decomposed in a spin-up part and a spin down-part,
\begin{equation}
|\hat{S}(\bs{R})|=\frac{1}{\sqrt{N!}}|\hat{S}_{\uparrow}(\bs{R}^{\uparrow})|\cdot |\hat{S}_{\downarrow}(\bs{R}^{\downarrow})|,
\end{equation}
with $N$ as the number of electrons and $\bs{R}=\{\bs{r}_1,\bs{r}_2,\cdots,\bs{r}_N\}$ as the collective spatial coordinates. For quantum dot systems, the spin-$\sigma$ determinant can be split in a factor $G(\bs{R})$ and a determinant containing the Hermite polynomials,
\begin{equation}
|\hat{S}(\bs{R})|\equiv|\hat{S}_{\sigma}(\bs{R})|=G(\bs{R})
\begin{vmatrix}
H_1(\boldsymbol{r}_1) & H_2(\boldsymbol{r}_1) & \cdots & H_N(\boldsymbol{r}_1)\\
H_1(\boldsymbol{r}_2) & H_2(\boldsymbol{r}_2) & \cdots & H_N(\boldsymbol{r}_2)\\
\vdots & \vdots & \ddots & \vdots \\
H_1(\boldsymbol{r}_N) & H_2(\boldsymbol{r}_N) & \cdots & H_N(\boldsymbol{r}_N)
\end{vmatrix},
\end{equation}
for $\sigma\in\{\uparrow, \downarrow\}$. For a standard VMC Slater determinant, the factor $G(\bs{R})$ is just the Gaussian function, $G(\bs{R})=\exp(-\frac{1}{2}\alpha\omega\sum_{i=1}^Fx_i^2)$, but as suggested by \citet{carleo_solving_2017}, this can be replaced by the marginal distribution of the visible units of a Gaussian-binary RBM in order to introduce more variational parameters to the trial wave function. This will be our RBM trial wave function ansatz. We then get the determinant
\begin{equation}
|\hat{S}(\bs{R})|=P(\bs{R})
\underbrace{
	\begin{vmatrix}
	H_1(\bs{r}_1) & H_2(\bs{r}_1) & \cdots & H_N(\bs{r}_1) \\
	H_1(\bs{r}_2) & H_2(\bs{r}_2) & \cdots & H_N(\bs{r}_2) \\
	\vdots & \vdots & \ddots & \vdots \\
	H_1(\bs{r}_N) & H_2(\bs{r}_N) & \cdots & H_N(\bs{r}_N)
	\end{vmatrix},
}_{|\hat{D}(\bs{R})|}
\end{equation}
with
\begin{equation}
P(\bs{R})=\frac{1}{Z}\exp(-\sum_{i=1}^F\frac{(x_i-a_i)^2}{2\sigma_i^2})\prod_{j=1}^H\left[1+\exp(b_j+\sum_{i=1}^F\frac{x_iw_{ij}}{2\sigma_i^2})\right]
\end{equation}
as the marginal distribution, taken from section \ref{sec:gaussianbinary} where all the symbols are defined. This can be treated as a separate wave function element, as discussed in section \ref{sec:factorizing}, and we can reuse the evaluation of the determinant $|\hat{D}(\bs{R})|$ discussed in section \ref{sec:slaterdeterminant}. If we denote this element by "rbm" and avoid the normalization factor $Z$, the total element is given by
\begin{equation}
\Psi_{\text{rbm}}(\bs{x};\bs{a},\bs{b},\bs{w})=\exp(-\sum_{i=1}^{F}\frac{(x_i-a_i)^2}{2\sigma_i^2})\prod_{j=1}^H\left[1+\exp(b_j+\sum_{i=1}^{F}\frac{x_iw_{ij}}{\sigma_i^2})\right],
\label{eq:rbmelement}
\end{equation}
which can naturally be split further into a Gaussian part and a product part. This splitting is performed in order to simplify the equations needed when computing the kinetic energy. They will also be implemented as separate wave function elements, as this will reduce the complexity of the derivatives associated with each element. The first part will henceforth be denoted by "RBM-Gaussian", while the last part will be denoted by "RBM-product". 

\section{RBM-Gaussian}
The RBM-Gaussian is just the first part of equation \eqref{eq:rbmelement} and reads
\begin{equation}
\Psi_{\text{rg}}(\bs{x};\bs{a})=\exp(-\sum_{i=1}^{F}\frac{(x_i-a_i)^2}{2\sigma_i^2}),
\end{equation}
which is really similar to the simple Gaussian presented in section \ref{sec:simplegaussian}. Also the gradient, Laplacian and the gradient with respect to the variational parameters become similar, and we will for that reason just list them up:
\begin{empheq}[box={\mybluebox[5pt]}]{equation}
\label{eq:NQSGaussian}
\begin{aligned}
\frac{|\Psi_{\text{rg}}(\bs{x}_{\text{new}})|^2}{|\Psi_{\text{rg}}(\bs{x}_{\text{old}})|^2}&=\exp(\frac{(x_i^{\text{old}}-a_i)^2-(x_i^{\text{new}}-a_i)^2}{2\sigma_i^2}),\\
\nabla_k\ln\Psi_{\text{rg}} &= -\frac{x_k-a_k}{\sigma_i^2},\\
\nabla_k^2\ln\Psi_{\text{rg}}&=-\frac{1}{\sigma_i^2},\\
\nabla_{\alpha_l}\ln\Psi_{\text{rg}} &= \frac{x_l-a_l}{\sigma_i^2}.
\end{aligned}
\end{empheq}
Further, the frequency of the quantum dots should be inversely proportional to the Gaussian sampling width from the Gaussian-binary RBM, $\sigma_i^2$, such that we can set 
\begin{equation}
\omega = \frac{1}{\sigma_i^2} \quad\quad\forall\quad\quad i\in\{1,\cdots,F\},
\end{equation}
for the RBMs to account for the oscillator frequency. An obvious optimization concerning this element, is that we can introduce a vector $\bs{xa}\equiv\bs{x}-\bs{a}$, which we deal with instead of the position vector $\bs{x}$ and the parameter vector $\bs{a}$. We update the arrays using the virtual function \lstinline|updateArrays|, which looks like
\begin{lstlisting}
void RBMGaussian::updateArrays(const Eigen::VectorXd positions,
                               const Eigen::VectorXd radialVector,
                               const Eigen::MatrixXd distanceMatrix,
                               const int i)
{
    m_positions = positions;
    m_Xa = positions - m_a;
    double sqrdDiff = m_XaOld(i) * m_XaOld(i) - m_Xa(i) * m_Xa(i);
    m_probabilityRatio = exp(sqrdDiff / (2 * m_sigma_iSqrd));
}
\end{lstlisting}
We see that the vector \lstinline|m_Xa|, corresponding to $\bs{xa}$, is declared globally such that it can be used in the gradients in equation \eqref{eq:NQSGaussian}. The updated coordinate is again denoted by \lstinline|i|. 

\section{RBM-product}
The RBM product is the last part of \eqref{eq:RBMWF2}, and is thus given by
\begin{equation}
\Psi_{\text{rp}}(\bs{x};\bs{b},\bs{w})=\prod_{j=1}^H\left[1+\exp(b_j+\sum_{i=1}^{F}\frac{x_iw_{ij}}{\sigma_i^2})\right].
\end{equation}
For a general Gaussian-binary RBM product in the form of
\begin{equation}
\Psi(\bs{x};\bs{\theta})=\prod_{j=1}^H\bigg[1+\exp\Big(f_j(\bs{x};\bs{\theta})\Big)\bigg],
\end{equation}
it can be shown that the gradient and the Laplacian are
\begin{equation}
\nabla_k\ln\Psi_{\text{rp}}=\sum_{j=1}^Hn_j\nabla_k(f_j),
\end{equation}
and
\begin{equation}
\nabla_k^2\ln\Psi_{\text{rp}}=\sum_{j=1}^Hn_j\big[\nabla_k^2(f_j)+p_j\big(\nabla_k(f_j)\big)^2\big],
\end{equation}
respectively, with $p_j$ and $n_j$ as the sigmoid and its counterpart,
\begin{equation}
n_j\equiv \frac{1}{1+\exp(-f_j)}\quad\wedge\quad p_j\equiv \frac{1}{1+\exp(+f_j)}.
\end{equation}
Those expressions can be used to find the kinetic energy directly, as the kinetic energy contribution from this specific element is just the sum over the gradients and Laplacians, $T=\sum_{k=1}^F[\nabla_k^2\ln\Psi_{\text{rp}}+(\nabla_k\ln\Psi_{\text{rp}})^2]$. An arbitrary parameter $\theta_i$ can be updated according to the log-likelihood function which is just the derivative of the log-likelihood function
\begin{equation}
\frac{\partial}{\partial \theta_i}\ln \Psi_{\text{rp}}=\sum_{j=1}^Hn_j\frac{\partial}{\partial\theta_i}(f_j),
\end{equation}
and the ratio between the new and the old wave function elements can be found by the product
\begin{equation}
\frac{\Psi_{\text{rp}}^{\text{new}}}{\Psi_{\text{rp}}^{\text{old}}}=\prod_{j=1}^H\frac{p_j^{\text{old}}}{p_j^{\text{new}}}.
\end{equation}
As a conclusion, what we actually need to calculate to find respective expressions for each wave function element is $\nabla_k(f_j)$, $\nabla_k^2(f_j)$ and $\partial_{\theta_i}(f_j)$ for all variational parameters $\theta_i$, which is naturally simpler than differentiating the entire wave function element. For our particular expression of $f_j=b_j+\bs{w}_j^T\bs{x}/\sigma_i^2$, they can be found to be 
\begin{equation}
\begin{aligned}
\nabla_k(f_j)&=\frac{w_{kj}}{\sigma_i^2},\\
\nabla_k^2(f_j)&=0,\\
\partial_{b_l}(f_j)&=\delta_{lj},\\
\partial _{w_{ml}}(f_j)&=\frac{x_m}{\sigma_i^2}\delta_{lj},
\end{aligned}
\end{equation}
where $\delta_{lj}$ is the Kronecker delta. Finally, we obtain the needed equations
\begin{empheq}[box={\mybluebox[5pt]}]{equation}
\begin{aligned}
\frac{|\Psi_{\text{rp}}(\bs{x}_{\text{new}})|^2}{|\Psi_{\text{rp}}(\bs{x}_{\text{old}})|^2}&=\prod_{j=1}^H\frac{p_j(\bs{x}_{\text{old}})^2}{p_j(\bs{x}_{\text{new}})^2},\\
\nabla_k\ln\Psi_{\text{rp}} &=\sum_{j=1}^H\frac{w_{kj}}{\sigma_i^2}n_j,\\
\nabla_k^2\ln\Psi_{\text{rp}} &= \sum_{j=1}^H\frac{w_{kj}^2}{\sigma_i^4}p_jn_j,\\
\nabla_{b_l}\ln\Psi_{\text{rp}}&=n_l,\\
\nabla_{w_{ml}}\ln\Psi_{\text{rp}}&=\frac{x_mn_l}{\sigma_i^2}.
\end{aligned}
\end{empheq}
This is the same result as obtained for the gradient of the log-likelihood function presented in chapter \ref{chp:machinelearning}. In this element, there are plenty of optimization possibilities. For the RBM-Gaussian, we saw that the distribution width, $\sigma_i$, was set such that $\omega=1/\sigma_i^2$, but for this product the value of $\sigma_i$ has no impact on the outcome since it is always multiplied with the weights which are adjusted freely. By further revealing that some sums are vector products, we can get a significant speed-up. Firstly, we will define a vector 
\begin{equation}
\bs{v}=\bs{b}+\bs{w}^T\bs{x},
\end{equation}
which is the transformation for going from the hidden units to the visible units in Gaussian-binary restricted Boltzmann machine, and what we above have called $\bs{f}(\bs{x};\bs{\theta})$. Thereafter, we define the vectors $\bs{n}$ and $\bs{p}$ as described above. These vectors are declared as \lstinline|m_v|, \lstinline|m_n| and \lstinline|m_p| respectively, and are initialized and updated using the function \lstinline|updateVectors| in the following way

\begin{lstlisting}
void RBMProduct::updateVectors()
{
    m_v = m_b + m_W.transpose() * m_positions;
    Eigen::VectorXd e = m_v.array().exp();
    m_p = (e + Eigen::VectorXd::Ones(m_numberOfHiddenNodes)).cwiseInverse();
    m_n = e.cwiseProduct(m_p);
}
\end{lstlisting}
One can see that all the operations are vectorized, which makes the operations quite affordable. 

\section{Partly restricted Boltzmann machine}
For the partly restricted Boltzmann machine given in equation \eqref{eq:PRBMWF}, we observe that the only difference from a standard Boltzmann machine is the factor 
\begin{equation}
\Psi_{\text{pr}}=\exp\Big(\sum_{i=1}^{F}\sum_{j=1}^{F}x_ic_{ij}x_j\Big),
\end{equation}
which we also can threat separately. To run a computation with the partly restricted Boltzmann machine, we thus need to add the elements \lstinline|RBMGaussian|, \lstinline|RBMProduct| and \lstinline|PartlyRestricted| in a similar way as in the example \ref{lst:qd}. When differentiating, we end up with the expressions
\begin{empheq}[box={\mybluebox[5pt]}]{equation}
\begin{aligned}
\frac{|\Psi_{\text{pr}}(\bs{R}_{\text{new}})|^2}{|\Psi_{\text{pr}}(\bs{R}_{\text{old}})|^2}&=\exp\Big(2\sum_{j=1}^{F}c_{ij}x_j(x_i^{\text{new}}-x_i^{\text{old}})\Big),\\
\nabla_k\ln\Psi_{\text{pr}} &=2\sum_{j=1}^{F}c_{kj}x_j,\\
\nabla_k^2\ln\Psi_{\text{pr}} &= 2c_{kk},\\
\nabla_{c_{ml}}\ln\Psi_{\text{pr}}&=x_mx_l,
\end{aligned}
\end{empheq}
where $x_i$ is the changed coordinated. Also here can we use vectorization to speed-up the computations, most elegantly shown by the \lstinline|computeParameterGradient|,
\begin{lstlisting}
Eigen::VectorXd PartlyRestricted::computeParameterGradient()
{
    Eigen::MatrixXd out = m_positions * m_positions.transpose();
    m_gradients.head(out.size()) = WaveFunction::flatten(out);
    return m_gradients;
}
\end{lstlisting}
where we use that the parameter gradient $\nabla_{c_{ml}}\ln\Psi_{\text{pr}}$ is given by the outer product between the coordinate vectors, as already hinted in equation \eqref{eq:partlyparty}.